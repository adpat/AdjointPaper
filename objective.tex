%!TEX root =article.tex
% This part should be short. Talk about modeling the optimization problem, with dynamics as the constraints. 

% Then, discuss the extensibility of the formulation to other problems

% \begin{itemize}
%     \item State estimation
%     \item Variable Speed limits
%     \item Rerouting (probably cite some of my previous work)
% \end{itemize}


Given a network of junctions $\jns$ and links $\links$, a dynamical system $\sysMat$, and a set of control parameters $\control$, we wish to find an control parameter $\control^*$ that minimizes some cost function $\costMat$. For instance, in the state estimation problem discussed in Section~\ref{sub:control_parameters_for_ramp_metering}, an appropriate cost function would penalize for the difference between a known final density state $\bar{\density}$ and the final estimated density $\density'$.

For the problem of ramp metering, we seek to reduce congestion on a freeway/ramp system caused by excess ramp flow. One metric that captures congestion is \emph{total travel time}, or the total time spent on the network by every vehicle that enters the network. We define the cost function for total travel time:

\begin{equation}
    \label{eqn:total-travel-time}
    \cost\left(\state,\control\right) = \sum_i,k \idx{\density}{i}{k}\length_i + \idx{\queue}{i}{k}
\end{equation}

Given some initial and boundary conditions that specify a time horizon over which to optimize, then this problem can be phrased as a constrained, finite-horizon optimal control problem:
\begin{equation}
\begin{aligned}
\label{eqn:optimal-control-problem}
\underset{u}{\text{minimize}} & \quad \cost\left(\state,\control\right) \\
\text{such that:} & \\
& \sysMat \left(\state, \control\right) = 0
\end{aligned}
\end{equation}

The rest of the article details our solution methods and modifications of Problem~\eqref{eqn:optimal-control-problem}.