% A sample chapter file- it contains a lot of 
% environments, including tabulars, align, etc
%   
% Don't try and compile this file using pdflatex etc, just 
% compare the *format* of it to the format of the
% sampleAFTER.tex
%
% In particular, compare the tabular and align-type
% environments before and after running the script

\section{Polynomial functions}
\reformatstepslist{P} % the steps list should be P1, P2, \ldots
In your previous mathematics classes you have studied \emph{linear} and 
\emph{quadratic} functions. The most general forms of these types of 
functions can be represented (respectively) by  the functions $f$ 
and $g$ that have formulas
\begin{equation}\label{poly:eq:linquad}
	f(x)=mx+b, \qquad g(x)=ax^2+bx+c
\end{equation}
We know that $m$ is the slope of $f$, and that $a$ is the \emph{leading coefficient} 
of $g$. We also know that the \emph{signs} of $m$ and $a$ completely 
determine the behavior of the functions $f$ and $g$. For example, if $m>0$
then $f$ is an \emph{increasing} function, and if $m<0$ then $f$ is 
a \emph{decreasing} function.  Similarly, if $a>0$ then $g$ is 
\emph{concave up} and if $a<0$ then $g$ is \emph{concave down}. Graphical 
representations of these statements are given in \cref{poly:fig:linquad}.

\begin{figure}[!htb]
	\setlength{\figurewidth}{.2\textwidth}
	\begin{subfigure}{\figurewidth}
		\begin{tikzpicture}
			\begin{axis}[
			   framed,
			   xmin=-10,xmax=10,
			   ymin=-10,ymax=10,
			   width=\textwidth,
			   xtick={-11},
			   ytick={-11},
			   ]
			   \addplot expression[domain=-10:8]{(x+2)};
			\end{axis}
		\end{tikzpicture}
		\caption{$m>0$}
	\end{subfigure}
	\hfill
	\begin{subfigure}{\figurewidth}
		\begin{tikzpicture}
			\begin{axis}[
			   framed,
			   xmin=-10,xmax=10,
			   ymin=-10,ymax=10,
			   width=\textwidth,
			   xtick={-11},
			   ytick={-11},
			   ]
			   \addplot expression[domain=-10:8]{-(x+2)};
			\end{axis}
		\end{tikzpicture}
		\caption{$m<0$}
	\end{subfigure}
	\hfill
	\begin{subfigure}{\figurewidth}
		\begin{tikzpicture}
			\begin{axis}[
			   framed,
			   xmin=-10,xmax=10,
			   ymin=-10,ymax=10,
			   width=\textwidth,
			   xtick={-11},
			   ytick={-11},
			   ]
			   \addplot expression[domain=-4:4]{(x^2-6)};
			\end{axis}
		\end{tikzpicture}
		\caption{$a>0$}
	\end{subfigure}
	\hfill
	\begin{subfigure}{\figurewidth}
		\begin{tikzpicture}
			\begin{axis}[
			   framed,
			   xmin=-10,xmax=10,
			   ymin=-10,ymax=10,
			   width=\textwidth,
			   xtick={-11},
			   ytick={-11},
			   ]
			   \addplot expression[domain=-4:4]{-(x^2-6)};
			\end{axis}
		\end{tikzpicture}
		\caption{$a<0$}
	\end{subfigure}
	\caption{Typical graphs of linear and quadratic functions.}
	\label{poly:fig:linquad}
\end{figure}

Let's look a little more closely at the formulas for $f$ and $g$ in 
\cref{poly:eq:linquad}. Note that the \emph{degree} 
of $f$ is $1$ since the highest power of $x$ that is present in the 
formula for $f(x)$ is $1$. Similarly, the degree of $g$ is $2$ since
the highest power of $x$ that is present in the formula for $g(x)$ 
is $2$.

In this section we will build upon our knowledge of these elementary
functions. In particular, we will generalize the functions $f$ and $g$ to a function $p$ that has 
any degree that we wish.

%===================================
%   Author: Hughes
%   Date:   March 2012
%===================================
\begin{essentialskills}
	%===================================
	%   Author: Hughes
	%   Date:   March 2012
	%===================================
	\begin{problem}[Quadratic functions]
	Every quadratic function has the form $y=ax^2+bx+c$; state the value 
	of $a$ for each of the following functions, and hence decide if the 
	parabola that represents the function opens upward or downward.
	\begin{multicols}{2}
		\begin{subproblem}
			$F(x)=x^2+3$ 
			\begin{shortsolution}
				$a=1$; the parabola opens upward. 
			\end{shortsolution}
		\end{subproblem}
		\begin{subproblem}
			$G(t)=4-5t^2$ 
			\begin{shortsolution}
				$a=-5$; the parabola opens downward. 
			\end{shortsolution}
		\end{subproblem}
		\begin{subproblem}
			$H(y)=4y^2-96y+8$ 
			\begin{shortsolution}
				$a=4$; the parabola opens upward. 
			\end{shortsolution}
		\end{subproblem}
		\begin{subproblem}
			$K(z)=-19z^2$ 
			\begin{shortsolution}
				$m=-19$; the parabola opens downward. 
			\end{shortsolution}
		\end{subproblem}
	\end{multicols}
	Now let's generalize our findings for the most general quadratic function $g$
	that has formula $g(x)=a_2x^2+a_1x+a_0$. Complete the following sentences.
	\begin{subproblem}
		When $a_2>0$, the parabola that represents $y=g(x)$ opens $\ldots$ 
		\begin{shortsolution}
			When $a_2>0$, the parabola that represents the function opens upward.
		\end{shortsolution}
	\end{subproblem}
	\begin{subproblem}
		When $a_2<0$, the parabola that represents $y=g(x)$ opens $\ldots$ 
		\begin{shortsolution}
			When $a_2<0$, the parabola that represents the function opens downward.
		\end{shortsolution}
	\end{subproblem}
	\end{problem}
\end{essentialskills}

\subsection*{Power functions with positive exponents}
The study of polynomials will rely upon a good knowledge 
of power functions| you may reasonably ask, what is a power function?
\begin{pccdefinition}[Power functions]
Power functions have the form
\[
	f(x) = a_n x^n
\]
where $n$ can be any real number.

Note that for this section we will only be concerned with the 
case when $n$ is a positive integer.
\end{pccdefinition}

You may find assurance in the fact that you are already very comfortable 
with power functions that have $n=1$ (linear) and $n=2$ (quadratic). Let's 
explore some power functions that you might not be so familiar with.
As you read \cref{poly:ex:oddpow,poly:ex:evenpow}, try and spot 
as many patterns and similarities as you can.

%===================================
%   Author: Hughes
%   Date:   March 2012
%===================================
\begin{pccexample}[Power functions with odd positive exponents]
\label{poly:ex:oddpow}
Graph each of the following functions, state their domain, and their 
long-run behavior as $x\rightarrow\pm\infty$
\[
	f(x)=x^3,   \qquad  g(x)=x^5, \qquad h(x)=x^7
\]
\begin{pccsolution}
The functions $f$, $g$, and $h$ are plotted in \cref{poly:fig:oddpow}.
The domain of each of the functions $f$, $g$, and $h$ is $(-\infty,\infty)$. Note that 
the long-run behavior of each of the functions is the same, and in particular
\begin{align*}
	f(x)\rightarrow\infty                           & \text{ as } x\rightarrow\infty  \\
	\mathllap{\text{and }}   f(x)\rightarrow-\infty & \text{ as } x\rightarrow-\infty 
\end{align*}
The same results hold for $g$ and $h$.
\end{pccsolution}
\end{pccexample}

\begin{figure}[!htb]
	\begin{minipage}{.45\textwidth}
		\begin{tikzpicture}
			\begin{axis}[
			   framed,
			   xmin=-1.5,xmax=1.5,
			   ymin=-5,ymax=5,
			   xtick={-1.0,-0.5,...,1.0},
			   minor ytick={-3,-1,...,3},
			   grid=both,
			   width=\textwidth,
			   legend pos=north west,
			   ]
			   \addplot expression[domain=-1.5:1.5]{x^3};
			   \addplot expression[domain=-1.379:1.379]{x^5};
			   \addplot expression[domain=-1.258:1.258]{x^7};
			   \addplot[soldot]coordinates{(-1,-1)} node[axisnode,anchor=north west]{$(-1,-1)$};
			   \addplot[soldot]coordinates{(1,1)} node[axisnode,anchor=south east]{$(1,1)$};
			   \legend{$f$,$g$,$h$}
			\end{axis}
		\end{tikzpicture}
		\caption{Odd power functions}
		\label{poly:fig:oddpow}
	\end{minipage}%
	\hfill
	\begin{minipage}{.45\textwidth}
		\begin{tikzpicture}
			\begin{axis}[
			   framed,
			   xmin=-2.5,xmax=2.5,
			   ymin=-5,ymax=5,
			   xtick={-2.0,-1.5,...,2.0},
			   minor ytick={-3,-1,...,3},
			   grid=both,
			   width=\textwidth,
			   legend pos=south east,
			   ]
			   \addplot expression[domain=-2.236:2.236]{x^2};
			   \addplot expression[domain=-1.495:1.495]{x^4};
			   \addplot expression[domain=-1.307:1.307]{x^6};
			   \addplot[soldot]coordinates{(-1,1)} node[axisnode,anchor=east]{$(-1,1)$};
			   \addplot[soldot]coordinates{(1,1)} node[axisnode,anchor=west]{$(1,1)$};
			   \legend{$F$,$G$,$H$}
			\end{axis}
		\end{tikzpicture}
		\caption{Even power functions}
		\label{poly:fig:evenpow}
	\end{minipage}%
\end{figure}

%===================================
%   Author: Hughes
%   Date:   March 2012
%===================================
\begin{pccexample}[Power functions with even positive exponents]\label{poly:ex:evenpow}%
Graph each of the following functions, state their domain, and their 
long-run behavior as $x\rightarrow\pm\infty$
\[
	F(x)=x^2, \qquad G(x)=x^4, \qquad H(x)=x^6
\]
\begin{pccsolution}
The functions $F$, $G$, and $H$ are plotted in \cref{poly:fig:evenpow}. The domain
of each of the functions is $(-\infty,\infty)$. Note that the long-run behavior 
of each of the functions is the same, and in particular
\begin{align*}
	F(x)\rightarrow\infty                          & \text{ as } x\rightarrow\infty  \\
	\mathllap{\text{and }}   F(x)\rightarrow\infty & \text{ as } x\rightarrow-\infty 
\end{align*}
The same result holds for $G$ and $H$.
\end{pccsolution}
\end{pccexample}

\begin{doyouunderstand}
	\begin{problem}
	Repeat \cref{poly:ex:oddpow,poly:ex:evenpow} using (respectively)
	\begin{subproblem}
		$f(x)=-x^3,   \qquad  g(x)=-x^5, \qquad h(x)=-x^7$
		\begin{shortsolution}
			The functions $f$, $g$, and $h$ have domain $(-\infty,\infty)$ and 
			are graphed below.
			
			\begin{tikzpicture}
				\begin{axis}[
				   framed,
				   xmin=-1.5,xmax=1.5,
				   ymin=-5,ymax=5,
				   xtick={-1.0,-0.5,...,0.5},
				   minor ytick={-3,-1,...,3},
				   grid=both,
				   width=\solutionfigurewidth,
				   legend pos=north east,
				   ]
				   \addplot expression[domain=-1.5:1.5]{-x^3};
				   \addplot expression[domain=-1.379:1.379]{-x^5};
				   \addplot expression[domain=-1.258:1.258]{-x^7};
				   \legend{$f$,$g$,$h$}
				\end{axis}
			\end{tikzpicture}
			
			Note that
			\begin{align*}
				f(x)\rightarrow-\infty                         & \text{ as } x\rightarrow\infty  \\
				\mathllap{\text{and }}   f(x)\rightarrow\infty & \text{ as } x\rightarrow-\infty 
			\end{align*}
			The same is true for $g$ and $h$.
		\end{shortsolution}
	\end{subproblem}
	\begin{subproblem}
		$F(x)=-x^2,   \qquad  G(x)=-x^4, \qquad H(x)=-x^6$
		\begin{shortsolution}
			The functions $F$, $G$, and $H$ have domain $(-\infty,\infty)$ and 
			are graphed below.
			
			\begin{tikzpicture}
				\begin{axis}[
				   framed,
				   xmin=-2.5,xmax=2.5,
				   ymin=-5,ymax=5,
				   xtick={-1.0,-0.5,...,0.5},
				   minor ytick={-3,-1,...,3},
				   grid=both,
				   width=\solutionfigurewidth,
				   legend pos=north east,
				   ]
				   \addplot expression[domain=-2.236:2.236]{-x^2};
				   \addplot expression[domain=-1.495:1.495]{-x^4};
				   \addplot expression[domain=-1.307:1.307]{-x^6};
				   \legend{$F$,$G$,$H$}
				\end{axis}
			\end{tikzpicture}
			
			Note that
			\begin{align*}
				F(x)\rightarrow-\infty                          & \text{ as } x\rightarrow\infty  \\
				\mathllap{\text{and }}   F(x)\rightarrow-\infty & \text{ as } x\rightarrow-\infty 
			\end{align*}
			The same is true for $G$ and $H$.
		\end{shortsolution}
	\end{subproblem}
	\end{problem}
\end{doyouunderstand}

\subsection*{Polynomial functions}
Now that we have a little more familiarity with power functions, 
we can define polynomial functions. Provided that you were comfortable
with our opening discussion about linear and quadratic functions (see 
$f$ and $g$ in \cref{poly:eq:linquad}) then there is every chance 
that you'll be able to master polynomial functions as well; just remember
that polynomial functions are a natural generalization of linear
and quadratic functions. Once you've studied the examples and problems
in this section, you'll hopefully agree that polynomial functions
are remarkably predictable.

%===================================
%   Author: Hughes
%   Date:   May 2011
%===================================
\begin{pccdefinition}[Polynomial functions]
Polynomial functions have the form
\[
	p(x)=a_nx^n+a_{n-1}x^{n-1}+\ldots+a_1x+a_0
\]
where $a_n$, $a_{n-1}$, $a_{n-2}$, \ldots, $a_0$ are real numbers.
\begin{itemize}
	\item We call $n$ the degree of the polynomial, and require that $n$
	is a non-negative integer;
	\item $a_n$, $a_{n-1}$, $a_{n-2}$, \ldots, $a_0$ are called the coefficients;
	\item We typically write polynomial functions in descending powers of $x$.
\end{itemize}
In particular, we call $a_n$ the \emph{leading} coefficient, and $a_nx^n$ the 
\emph{leading term}.

Note that if a polynomial is given in factored form, then the degree can be found 
by counting the number of linear factors.
\end{pccdefinition}

%===================================
%   Author: Hughes
%   Date:   March 2012
%===================================
\begin{pccexample}[Polynomial or not]
Identify the following functions as polynomial or not; if the function 
is a polynomial, state its degree.
\begin{multicols}{3}
	\begin{enumerate}
		\item $p(x)=x^2-3$     
		\item $q(x)=-4x^{\nicefrac{1}{2}}+10$     
		\item $r(x)=10x^5$
		\item $s(x)=x^{-2}+x^{23}$
		\item $f(x)=-8$
		\item $g(x)=3^x$
		\item $h(x)=\sqrt[3]{x^7}-x^2+x$
		\item $k(x)=4x(x+2)(x-3)$
		\item $j(x)=x^2(x-4)(5-x)$
	\end{enumerate}
\end{multicols}
\begin{pccsolution}
\begin{enumerate}
	\item $p$ is a polynomial, and its degree is $2$.
	\item $q$ is \emph{not} a polynomial, because $\frac{1}{2}$ is not an integer.
	\item $r$ is a polynomial, and its degree is $5$.
	\item $s$ is \emph{not} a polynomial, because $-2$ is not a positive integer.
	\item $f$ is a polynomial, and its degree is $0$.
	\item $g$ is \emph{not} a polynomial, because the independent 
	variable, $x$, is in the exponent.
	\item $h$ is \emph{not} a polynomial, because $\frac{7}{3}$ is not an integer.
	\item $k$ is a polynomial, and its degree is $3$.
	\item $j$ is a polynomial, and its degree is $4$.
\end{enumerate}
\end{pccsolution}
\end{pccexample}

%===================================
%   Author: Hughes
%   Date:   March 2012
%===================================
\begin{pccexample}[Typical graphs]\label{poly:ex:typical}
\Cref{poly:fig:typical} shows graphs of some polynomial functions;
the ticks have deliberately been left off the axis to allow us to concentrate
on the features of each graph. Note in particular that:
\begin{itemize}
	\item \cref{poly:fig:typical1} shows a degree-$1$ polynomial (you might also 
	classify the function as linear) whose leading coefficient, $a_1$, is positive.
	\item \cref{poly:fig:typical2} shows a degree-$2$ polynomial (you might also
	classify the function as quadratic) whose leading coefficient, $a_2$, is positive.
	\item \cref{poly:fig:typical3} shows a degree-$3$ polynomial whose leading coefficient, $a_3$,
	is positive| compare its overall
	shape and long-run behavior to the functions described in \cref{poly:ex:oddpow}.
	\item \cref{poly:fig:typical4} shows a degree-$4$ polynomial whose leading coefficient, $a_4$,
	is positive|compare its overall shape and long-run behavior to the functions described in \cref{poly:ex:evenpow}.
	\item \cref{poly:fig:typical5} shows a degree-$5$ polynomial whose leading coefficient, $a_5$,
	is positive| compare its overall
	shape and long-run behavior to the functions described in \cref{poly:ex:oddpow}.
\end{itemize}
\end{pccexample}

%===================================
%   Author: Hughes
%   Date:   May 2011
%===================================
\begin{figure}[!htb]
	\begin{widepage}
	\setlength{\figurewidth}{\textwidth/6}
	\begin{subfigure}{\figurewidth}
		\begin{tikzpicture}
			\begin{axis}[
			   framed,
			   xmin=-10,xmax=10,
			   ymin=-10,ymax=10,
			   width=\textwidth,
			   xtick={-11},
			   ytick={-11},
			   ]
			   \addplot expression[domain=-10:8]{(x+2)};
			\end{axis}
		\end{tikzpicture}
		\caption{$a_1>0$}
		\label{poly:fig:typical1}
	\end{subfigure}
	\hfill
	\begin{subfigure}{\figurewidth}
		\begin{tikzpicture}
			\begin{axis}[
			   framed,
			   xmin=-10,xmax=10,
			   ymin=-10,ymax=10,
			   width=\textwidth,
			   xtick={-11},
			   ytick={-11},
			   ]
			   \addplot expression[domain=-4:4]{(x^2-6)};
			\end{axis}
		\end{tikzpicture}
		\caption{$a_2>0$}
		\label{poly:fig:typical2}
	\end{subfigure}
	\hfill
	\begin{subfigure}{\figurewidth}
		\begin{tikzpicture}
			\begin{axis}[
			   framed,
			   xmin=-10,xmax=10,
			   ymin=-10,ymax=10,
			   width=\textwidth,
			   xtick={-11},
			   ytick={-11},
			   ]
			   \addplot expression[domain=-7.5:7.5]{0.05*(x+6)*x*(x-6)};
			\end{axis}
		\end{tikzpicture}
		\caption{$a_3>0$}
		\label{poly:fig:typical3}
	\end{subfigure}
	\hfill
	\begin{subfigure}{\figurewidth}
		\begin{tikzpicture}
			\begin{axis}[
			   framed,
			   xmin=-10,xmax=10,
			   ymin=-10,ymax=10,
			   width=\textwidth,
			   xtick={-11},
			   ytick={-11},
			   ]
			   \addplot expression[domain=-2.35:5.35,samples=100]{0.2*(x-5)*x*(x-3)*(x+2)};
			\end{axis}
		\end{tikzpicture}
		\caption{$a_4>0$}
		\label{poly:fig:typical4}
	\end{subfigure}
	\hfill
	\begin{subfigure}{\figurewidth}
		\begin{tikzpicture}
			\begin{axis}[
			   framed,
			   xmin=-10,xmax=10,
			   ymin=-10,ymax=10,
			   width=\textwidth,
			   xtick={-11},
			   ytick={-11},
			   ]
			   \addplot expression[domain=-5.5:6.3,samples=100]{0.01*(x+2)*x*(x-3)*(x+5)*(x-6)};
			\end{axis}
		\end{tikzpicture}
		\caption{$a_5>0$}
		\label{poly:fig:typical5}
	\end{subfigure}
	\end{widepage}
	\caption{Graphs to illustrate typical curves of polynomial functions.}
	\label{poly:fig:typical}
\end{figure}

%===================================
%   Author: Hughes
%   Date:   March 2012
%===================================
\begin{doyouunderstand}
	\begin{problem}
	Use \cref{poly:ex:typical} and \cref{poly:fig:typical} to help you sketch 
	the graphs of polynomial functions that have negative leading coefficients| note
	that there are many ways to do this! The intention with this problem
	is to use your knowledge of transformations- in particular, \emph{reflections}- 
	to guide you.
	\begin{shortsolution}
		$a_1<0$:
		
		\begin{tikzpicture}
			\begin{axis}[
			   framed,
			   xmin=-10,xmax=10,
			   ymin=-10,ymax=10,
			   width=\solutionfigurewidth,
			   xtick={-11},
			   ytick={-11},
			   ]
			   \addplot expression[domain=-10:8]{-(x+2)};
			\end{axis}
		\end{tikzpicture}
		
		$a_2<0$
		
		\begin{tikzpicture}
			\begin{axis}[
			   framed,
			   xmin=-10,xmax=10,
			   ymin=-10,ymax=10,
			   width=\solutionfigurewidth,
			   xtick={-11},
			   ytick={-11},
			   ]
			   \addplot expression[domain=-4:4]{-(x^2-6)};
			\end{axis}
		\end{tikzpicture}
		
		$a_3<0$
		
		\begin{tikzpicture}
			\begin{axis}[
			   framed,
			   xmin=-10,xmax=10,
			   ymin=-10,ymax=10,
			   width=\solutionfigurewidth,
			   xtick={-11},
			   ytick={-11},
			   ]
			   \addplot expression[domain=-7.5:7.5]{-0.05*(x+6)*x*(x-6)};
			\end{axis}
		\end{tikzpicture}
		
		$a_4<0$
		
		\begin{tikzpicture}
			\begin{axis}[
			   framed,
			   xmin=-10,xmax=10,
			   ymin=-10,ymax=10,
			   width=\solutionfigurewidth,
			   xtick={-11},
			   ytick={-11},
			   ]
			   \addplot expression[domain=-2.35:5.35,samples=100]{-0.2*(x-5)*x*(x-3)*(x+2)};
			\end{axis}
		\end{tikzpicture}
		
		$a_5<0$
		
		\begin{tikzpicture}
			\begin{axis}[
			   framed,
			   xmin=-10,xmax=10,
			   ymin=-10,ymax=10,
			   width=\solutionfigurewidth,
			   xtick={-11},
			   ytick={-11},
			   ]
			   \addplot expression[domain=-5.5:6.3,samples=100]{-0.01*(x+2)*x*(x-3)*(x+5)*(x-6)};
			\end{axis}
		\end{tikzpicture}
	\end{shortsolution}
	\end{problem}
\end{doyouunderstand}

\fixthis{poly: Need a more basic example here- it can have a similar
format to the multiple zeros example, but just keep it simple; it should
be halfway between the 2 examples surrounding it}

%===================================
%   Author: Hughes
%   Date:   May 2011
%===================================
\begin{pccexample}[Multiple zeros]
Consider the polynomial functions $p$, $q$, and $r$ which are 
graphed in \cref{poly:fig:moremultiple}. 
The formulas for $p$, $q$, and $r$ are as follows
\begin{align*}
	p(x) & =(x-3)^2(x+4)^2       \\
	q(x) & =x(x+2)^2(x-1)^2(x-3) \\
	r(x) & =x(x-3)^3(x+1)^2      
\end{align*}
Find the degree of $p$, $q$, and $r$, and decide if the functions bounce off or cut 
through the horizontal axis at each of their zeros.
\begin{pccsolution}
The degree of $p$ is 4. Referring to \cref{poly:fig:bouncep}, 
the curve bounces off the horizontal axis at both zeros, $3$ and $4$.

The degree of $q$ is 6. Referring to \cref{poly:fig:bounceq},
the curve bounces off the horizontal axis at $-2$ and $1$, and cuts 
through the horizontal axis at $0$ and $3$.

The degree of $r$ is 6. Referring to \cref{poly:fig:bouncer},
the curve bounces off the horizontal axis at $-1$, and cuts through 
the horizontal axis at $0$ and at $3$, although is flattened immediately to the left and right of $3$.
\end{pccsolution}
\end{pccexample}

\setlength{\figurewidth}{0.25\textwidth}
\begin{figure}[!htb]
	\begin{subfigure}{\figurewidth}
		\begin{tikzpicture}
			\begin{axis}[
			   xmin=-6,xmax=5,
			   ymin=-30,ymax=200,
			   xtick={-4,-2,...,4},
			   width=\textwidth,
			   ]
			   \addplot expression[domain=-5.63733:4.63733,samples=50]{(x-3)^2*(x+4)^2};
			   \addplot[soldot]coordinates{(3,0)(-4,0)};
			\end{axis}
		\end{tikzpicture}
		\caption{$y=p(x)$}
		\label{poly:fig:bouncep}
	\end{subfigure}
	\hfill
	\begin{subfigure}{\figurewidth}
		\begin{tikzpicture}
			\begin{axis}[
			   xmin=-3,xmax=4,
			   xtick={-2,...,3},
			   ymin=-60,ymax=40,
			   width=\textwidth,
			   ]
			   \addplot+[samples=50] expression[domain=-2.49011:3.11054]{x*(x+2)^2*(x-1)^2*(x-3)};
			   \addplot[soldot]coordinates{(-2,0)(0,0)(1,0)(3,0)};
			\end{axis}
		\end{tikzpicture}
		\caption{$y=q(x)$}
		\label{poly:fig:bounceq}
	\end{subfigure}
	\hfill
	\begin{subfigure}{\figurewidth}
		\begin{tikzpicture}
			\begin{axis}[
			   xmin=-2,xmax=4,
			   xtick={-1,...,3},
			   ymin=-40,ymax=40,
			   width=\textwidth,
			   ]
			   \addplot expression[domain=-1.53024:3.77464,samples=50]{x*(x-3)^3*(x+1)^2};
			   \addplot[soldot]coordinates{(-1,0)(0,0)(3,0)};
			\end{axis}
		\end{tikzpicture}
		\caption{$y=r(x)$}
		\label{poly:fig:bouncer}
	\end{subfigure}
	\caption{}
	\label{poly:fig:moremultiple}
\end{figure}

\begin{pccdefinition}[Multiple zeros]\label{poly:def:multzero}
Let $p$ be a polynomial that has a repeated linear factor $(x-a)^n$. Then we say
that $p$ has a multiple zero at $a$ of multiplicity $n$ and 
\begin{itemize}
	\item if the factor $(x-a)$ is repeated an even number of times, the graph of $y=p(x)$ does not
	cross the $x$ axis at $a$, but `bounces' off the horizontal axis at $a$.
	\item if the factor $(x-a)$ is repeated an odd number of times, the graph of $y=p(x)$ crosses the
	horizontal axis at $a$, but it looks `flattened' there
\end{itemize}
If $n=1$, then we say that $p$ has a \emph{simple} zero at $a$.
\end{pccdefinition}

%===================================
%   Author: Hughes
%   Date:   May 2011
%===================================
\begin{pccexample}[Find a formula]
Find formulas for the polynomial functions, $p$ and $q$, graphed in \cref{poly:fig:findformulademoboth}.
\begin{figure}[!htb]
	\begin{subfigure}{.45\textwidth}
		\begin{tikzpicture}
			\begin{axis}[framed,
			   xmin=-5,xmax=5,
			   ymin=-10,ymax=10,
			   xtick={-4,-2,...,4},
			   minor xtick={-3,-1,...,3},
			   ytick={-8,-6,...,8},
			   width=\textwidth,
			   grid=both]
			   \addplot expression[domain=-3.25842:2.25842,samples=50]{-x*(x-2)*(x+3)*(x+1)};
			   \addplot[soldot]coordinates{(1,8)}node[axisnode,inner sep=.35cm,anchor=west]{$(1,8)$};
			   \addplot[soldot]coordinates{(-3,0)(-1,0)(0,0)(2,0)};
			\end{axis}
		\end{tikzpicture}
		\caption{$p$}
		\label{poly:fig:findformulademo}
	\end{subfigure}
	\hfill
	\begin{subfigure}{.45\textwidth}
		\begin{tikzpicture}
			            \begin{axis}[framed,
			               	xmin=-5,xmax=5,
			               	ymin=-10,ymax=10,
			               	xtick={-4,-2,...,4},
			   minor xtick={-3,-1,...,3},
			               	ytick={-8,-6,...,8},
			   width=\textwidth,
			   grid=both]
			   \addplot expression[domain=-4.33:4.08152]{-.25*(x+2)^2*(x-3)};
			   \addplot[soldot]coordinates{(2,4)}node[axisnode,anchor=south west]{$(2,4)$};
			   \addplot[soldot]coordinates{(-2,0)(3,0)};
			            \end{axis}
		\end{tikzpicture}
		\caption{$q$}
		\label{poly:fig:findformulademo1}
	\end{subfigure}
	\caption{}
	\label{poly:fig:findformulademoboth}
\end{figure}
\begin{pccsolution}
\begin{enumerate}
	\item We begin by noting that the horizontal intercepts of $p$ are $(-3,0)$, $(-1,0)$, $(0,0)$ and $(2,0)$. 
	We also note that each zero is simple (multiplicity $1$).
	If we assume that $p$ has no other zeros, then we can start by writing
	\begin{align*}
		p(x) & =(x+3)(x+1)(x-0)(x-2) \\
		     & =x(x+3)(x+1)(x-2)     \\
	\end{align*}
	According to \cref{poly:fig:findformulademo}, the point $(1,8)$ lies 
	on the curve $y=p(x)$.
	Let's check if the formula we have written satisfies this requirement
	\begin{align*}
		p(1) & = (1)(4)(2)(-1) \\
		     & = -8            
	\end{align*}
	which is clearly not correct| it is close though. We can correct this by 
	multiplying $p$ by a constant $k$; so let's assume that
	\[
		p(x)=kx(x+3)(x+1)(x-2)
	\]
	Then $p(1)=-8k$, and if this is to equal $8$, then $k=-1$. Therefore
	the formula for $p(x)$ is
	\[
		p(x)=-x(x+3)(x+1)(x-2)
	\]
	\item The function $q$ has a zero at $-2$ of multiplicity $2$, and zero of 
	multiplicity $1$ at $3$ (so $3$ is a simple zero of $q$); we can therefore assume that $q$ has the form
	\[
		q(x)=k(x+2)^2(x-3)
	\]
	where $k$ is some real number. In order to find $k$, we use the given ordered pair, $(2,4)$, and
	evaluate $p(2)$
	\begin{align*}
		p(2) & =k(4)^2(-1) \\
		     & =-16k       
	\end{align*}
	We solve the equation $4=-8k$ and obtain $k=-\frac{1}{4}$ and conclude that the 
	formula for $q(x)$ is
	\[
		q(x)=-\frac{1}{4}(x+2)^2(x-3)
	\]
\end{enumerate}
\end{pccsolution}
\end{pccexample}


\fixthis{Chris: need sketching polynomial problems}
\begin{pccspecialcomment}[Steps to follow when sketching polynomial functions]
	\begin{steps}
		\item \label{poly:step:first} Determine the degree of the polynomial, 
		its leading term and leading coefficient, and hence determine
		the long-run behavior of the polynomial| does it behave like $\pm x^2$ or $\pm x^3$ 
		as $x\rightarrow\pm\infty$?
		\item Determine the zeros and their multiplicity. Mark all zeros 
		and the vertical intercept on the graph using solid circles $\bullet$.
		\item \label{poly:step:last}  Deduce the overall shape of the curve, and sketch it. If there isn't
		enough information from the previous steps, then construct a table of values.
	\end{steps}
	Remember that until we have the tools of calculus, we won't be able to 
	find the exact coordinates of local minimums, local maximums, and points
	of inflection.
\end{pccspecialcomment}
Before we demonstrate some examples, it is important to remember the following:
\begin{itemize}
	\item our sketches will give a good representation of the overall 
	shape of the graph, but until we have the tools of calculus (from MTH 251)
	we can not find local minimums, local maximums, and inflection points algebraically. This
	means that we will make our best guess as to where these points are.
	\item we will not concern ourselves too much with the vertical scale (because of 
	our previous point)| we will, however, mark the vertical intercept (assuming there is one), 
	and any horizontal asymptotes.
\end{itemize}
%===================================
%   Author: Hughes
%   Date:   May 2012
%===================================
\begin{pccexample}\label{poly:ex:simplecubic}
Use \crefrange{poly:step:first}{poly:step:last} to sketch a graph of the function $p$ 
that has formula
\[
	p(x)=\frac{1}{2}(x-4)(x-1)(x+3)
\]
\begin{pccsolution}
\begin{steps}
	\item $p$ has degree $3$. The leading term of $p$ is $\frac{1}{2}x^3$, so the leading coefficient of $p$ 
	is $\frac{1}{2}$. The long-run behavior of $p$ is therefore similar to that of $x^3$.
	\item The zeros of $p$ are $-3$, $1$, and $4$; each zero is simple (i.e, it has multiplicity $1$).
	This means that the curve of $p$ cuts the horizontal axis at each zero. The vertical 
	intercept of $p$ is $(0,6)$.
	\item We draw the details we have obtained so far on \cref{poly:fig:simplecubicp1}. Given 
	that the curve of $p$ looks like the curve of $x^3$ in the long-run, we are able to complete a sketch of the 
	graph of $p$ in \cref{poly:fig:simplecubicp2}.
	
	Note that we can not find the coordinates of the local minimums, local maximums, and inflection
	points| for the moment we make reasonable guesses as to where these points are (you'll find how
	to do this in calculus).
\end{steps}

\begin{figure}[!htbp]
	\begin{subfigure}{.45\textwidth}
		\begin{tikzpicture}
			\begin{axis}[
			   xmin=-10,xmax=10,
			   ymin=-10,ymax=15,
			   xtick={-8,-6,...,8},
			   ytick={-5,5},
			   width=\textwidth,
			   ]
			   \addplot[soldot] coordinates{(-3,0)(1,0)(4,0)(0,6)}node[axisnode,anchor=south west]{$(0,6)$};
			\end{axis}
		\end{tikzpicture}
		\caption{}
		\label{poly:fig:simplecubicp1}
	\end{subfigure}%
	\hfill
	\begin{subfigure}{.45\textwidth}
		\begin{tikzpicture}
			\begin{axis}[
			   xmin=-10,xmax=10,
			   ymin=-10,ymax=15,
			   xtick={-8,-6,...,8},
			   ytick={-5,5},
			   width=\textwidth,
			   ]
			   \addplot[soldot] coordinates{(-3,0)(1,0)(4,0)(0,6)}node[axisnode,anchor=south west]{$(0,6)$};
			   \addplot[pccplot] expression[domain=-3.57675:4.95392,samples=100]{.5*(x-4)*(x-1)*(x+3)};
			\end{axis}
		\end{tikzpicture}
		\caption{}
		\label{poly:fig:simplecubicp2}
	\end{subfigure}%
	\caption{$y=\dfrac{1}{2}(x-4)(x-1)(x+3)$}
	\label{poly:fig:simplecubic}
\end{figure}
\end{pccsolution}
\end{pccexample}

%===================================
%   Author: Hughes
%   Date:   May 2012
%===================================
\begin{pccexample}\label{poly:ex:degree5}
Use \crefrange{poly:step:first}{poly:step:last} to sketch a graph of the function $q$ 
that has formula
\[
	q(x)=\frac{1}{200}(x+7)^2(2-x)(x-6)^2
\]
\begin{pccsolution}
\begin{steps}
	\item $q$ has degree $4$. The leading term of $q$ is 
	\[
		-\frac{1}{200}x^5
	\]
	so the leading coefficient of $q$ is $-\frac{1}{200}$. The long-run behavior of $q$
	is therefore similar to that of $-x^5$.
	\item The zeros of $q$ are $-7$ (multiplicity 2), $2$ (simple), and $6$ (multiplicity $2$).
	The curve of $q$ bounces off the horizontal axis at the zeros with multiplicity $2$ and
	cuts the horizontal axis at the simple zeros. The vertical intercept of $q$ is $\left( 0,\frac{441}{25} \right)$.
	\item We mark the details we have found so far on \cref{poly:fig:degree5p1}. Given that 
	the curve of $q$ looks like the curve of $-x^5$ in the long-run, we can complete \cref{poly:fig:degree5p2}.
\end{steps}

\begin{figure}[!htbp]
	\begin{subfigure}{.45\textwidth}
		\begin{tikzpicture}
			\begin{axis}[
			   xmin=-10,xmax=10,
			   ymin=-10,ymax=40,
			   xtick={-8,-6,...,8},
			   ytick={-5,0,...,35},
			   width=\textwidth,
			   ]
			   \addplot[soldot] coordinates{(-7,0)(2,0)(6,0)(0,441/25)}node[axisnode,anchor=south west]{$\left( 0, \frac{441}{25} \right)$};
			\end{axis}
		\end{tikzpicture}
		\caption{}
		\label{poly:fig:degree5p1}
	\end{subfigure}%
	\hfill
	\begin{subfigure}{.45\textwidth}
		\begin{tikzpicture}
			\begin{axis}[
			   xmin=-10,xmax=10,
			   ymin=-10,ymax=40,
			   xtick={-8,-6,...,8},
			   ytick={-5,0,...,35},
			   width=\textwidth,
			   ]
			   \addplot[soldot] coordinates{(-7,0)(2,0)(6,0)(0,441/25)}node[axisnode,anchor=south west]{$\left( 0, \frac{441}{25} \right)$};
			   \addplot[pccplot] expression[domain=-8.83223:7.34784,samples=50]{1/200*(x+7)^2*(2-x)*(x-6)^2};
			\end{axis}
		\end{tikzpicture}
		\caption{}
		\label{poly:fig:degree5p2}
	\end{subfigure}%
	\caption{$y=\dfrac{1}{200}(x+7)^2(2-x)(x-6)^2$}
	\label{poly:fig:degree5}
\end{figure}
\end{pccsolution}
\end{pccexample}

%===================================
%   Author: Hughes
%   Date:   May 2012
%===================================
\begin{pccexample}
Use \crefrange{poly:step:first}{poly:step:last} to sketch a graph of the function $r$ 
that has formula
\[
	r(x)=\frac{1}{100}x^3(x+4)(x-4)(x-6)
\]
\begin{pccsolution}
\begin{steps}
	\item $r$ has degree $6$. The leading term of $r$ is 
	\[
		\frac{1}{100}x^6
	\]
	so the leading coefficient of $r$ is $\frac{1}{100}$. The long-run behavior of $r$
	is therefore similar to that of $x^6$.
	\item The zeros of $r$ are $-4$ (simple), $0$ (multiplicity $3$), $4$ (simple), 
	and $6$ (simple). The vertical intercept of $r$ is $(0,0)$. The curve of $r$
	cuts the horizontal axis at the simple zeros, and goes through the axis
	at $(0,0)$, but does so in a flattened way.
	\item We mark the zeros and vertical intercept on \cref{poly:fig:degree6p1}. Given that
	the curve of $r$ looks like the curve of $x^6$ in the long-run, we complete the graph
	of $r$ in \cref{poly:fig:degree6p2}.
\end{steps}

\begin{figure}[!htbp]
	\begin{subfigure}{.45\textwidth}
		\begin{tikzpicture}
			\begin{axis}[
			   xmin=-5,xmax=10,
			   ymin=-20,ymax=10,
			   xtick={-4,-2,...,8},
			   ytick={-15,-10,...,5},
			   width=\textwidth,
			   ]
			   \addplot[soldot] coordinates{(-4,0)(0,0)(4,0)(6,0)};
			\end{axis}
		\end{tikzpicture}
		\caption{}
		\label{poly:fig:degree6p1}
	\end{subfigure}%
	\hfill
	\begin{subfigure}{.45\textwidth}
		\begin{tikzpicture}
			\begin{axis}[
			   xmin=-5,xmax=10,
			   ymin=-20,ymax=10,
			   xtick={-4,-2,...,8},
			   ytick={-15,-10,...,5},
			   width=\textwidth,
			   ]
			   \addplot[soldot] coordinates{(-4,0)(0,0)(4,0)(6,0)};
			   \addplot[pccplot] expression[domain=-4.16652:6.18911,samples=100]{1/100*(x+4)*x^3*(x-4)*(x-6)};
			\end{axis}
		\end{tikzpicture}
		\caption{}
		\label{poly:fig:degree6p2}
	\end{subfigure}%
	\caption{$y=\dfrac{1}{100}(x+4)x^3(x-4)(x-6)$}
\end{figure}
\end{pccsolution}
\end{pccexample}

%===================================
%   Author: Hughes
%   Date:   March 2012
%===================================
\begin{pccexample}[An open-topped box] 
A cardboard company makes open-topped boxes for their clients. The specifications
dictate that the box must have a square base, and that it must be open-topped. 
The company uses sheets of cardboard that are $\unit[1200]{cm^2}$. Assuming that 
the base of each box has side $x$ (measured in cm), it can be shown that the volume of each box, $V(x)$,
has formula
\[
	V(x)=\frac{x}{4}(1200-x^2)
\]
Find the dimensions of the box that maximize the volume.
\begin{pccsolution}
We graph $y=V(x)$ in \cref{poly:fig:opentoppedbox}. Note that because 
$x$ represents the length of a side, and $V(x)$ represents the volume
of the box, we necessarily require both values to be positive; we illustrate
the part of the curve that applies to this problem using a solid line. 

\begin{figure}[!htb]
	\centering
	\begin{tikzpicture}
		\begin{axis}[framed,
		   xmin=-50,xmax=50,
		   ymin=-5000,ymax=5000,
		   xtick={-40,-30,...,40},
		   minor xtick={-45,-35,...,45},
		   minor ytick={-3000,-1000,1000,3000},
		   width=.75\textwidth,
		   height=.5\textwidth,
		   grid=both]
		   \addplot[pccplot,dashed,<-] expression[domain=-40:0,samples=50]{x/4*(1200-x^2)};
		   \addplot[pccplot,-] expression[domain=0:34.64,samples=50]{x/4*(1200-x^2)};
		   \addplot[pccplot,dashed,->] expression[domain=34.64:40,samples=50]{x/4*(1200-x^2)};
		   \addplot[soldot] coordinates{(20,4000)};
		\end{axis}
	\end{tikzpicture}
	\caption{$y=V(x)$}
	\label{poly:fig:opentoppedbox}
\end{figure}

According to \cref{poly:fig:opentoppedbox}, the maximum volume of such a box is 
approximately $\unit[4000]{cm^2}$, and we achieve it using a base of length 
approximately $\unit[20]{cm}$. Since the base is square and each sheet of cardboard
is $\unit[1200]{cm^2}$, we conclude that the dimensions of each box are $\unit[20]{cm}\times\unit[20]{cm}\times\unit[30]{cm}$.
\end{pccsolution}
\end{pccexample}

\subsection*{Complex zeros}
There has been a pattern to all of the examples that we have seen so far|
the degree of the polynomial has dictated the number of \emph{real} zeros that the 
polynomial has. For example, the function $p$ in \cref{poly:ex:simplecubic}
has degree $3$, and $p$ has $3$ real zeros; the function $q$ in \cref{poly:ex:degree5}
has degree $5$ and $q$ has $5$ real zeros.

You may wonder if this result can be generalized| does every polynomial that
has degree $n$ have $n$ real zeros? Before we tackle the general result, 
let's consider an example that may help motivate it.
%===================================
%   Author: Hughes
%   Date:   June 2012
%===================================
\begin{pccexample}\label{poly:ex:complx}
Consider the polynomial function $c$ that has formula
\[
	c(x)=x(x^2+1)
\]
It is clear that $c$ has degree $3$, and that $c$ has a (simple) zero at $0$. Does
$c$ have any other zeros, i.e, can we find any values of $x$ that satisfy the equation
\begin{equation}\label{poly:eq:complx}
	x^2+1=0
\end{equation}
The solutions to \cref{poly:eq:complx} are $\pm i$.

We conclude that $c$ has $3$ zeros: $0$ and $\pm i$; we note that \emph{not 
all of them are real}.
\end{pccexample}
\Cref{poly:ex:complx} shows that not every degree-$3$ polynomial has $3$
\emph{real} zeros; however, if we are prepared to venture into the complex numbers, 
then we can state the following theorem.
%===================================
%   Author: Hughes
%   Date:   June 2012
%===================================
\begin{pccspecialcomment}[The fundamental theorem of algebra]
	Every polynomial function of degree $n$ has $n$ roots, some of which may 
	be complex, and some may be repeated.
\end{pccspecialcomment}
\fixthis{Fundamental theorem of algebra: is this wording ok? do we want
it as a theorem?}
%===================================
%   Author: Hughes
%   Date:   June 2012
%===================================
\begin{pccexample}
Find all the zeros of the polynomial function $p$ that has formula
\[
	p(x)=x^4-2x^3+5x^2
\]
\begin{pccsolution}
We begin by factoring $p$
\begin{align*}
	p(x) & =x^4-2x^3+5x^2 \\
	     & =x^2(x^2-2x+5) 
\end{align*}
We note that $0$ is a zero of $p$ with multiplicity $2$. The other zeros of $p$ 
can be found by solving the equation
\[
	x^2-2x+5=0
\]
This equation can not be factored, so we use the quadratic formula
\begin{align*}
	x & =\frac{2\pm\sqrt{(-2)^2}-20}{2(1)} \\
	  & =\frac{2\pm\sqrt{-16}}{2}          \\
	  & =1\pm 2i                           
\end{align*}
We conclude that $p$ has $4$ zeros: $0$ (multiplicity $2$), and $1\pm 2i$ (simple).
\end{pccsolution}
\end{pccexample}
%===================================
%   Author: Hughes
%   Date:   June 2012
%===================================
\begin{pccexample}
Find a polynomial that has zeros at $2\pm i\sqrt{2}$. 
\begin{pccsolution}
We know that the zeros of a polynomial can be found by analyzing the linear
factors. We are given the zeros, and have to work backwards to find the 
linear factors. 

We begin by assuming that $p$ has the form
\begin{align*}
	p(x) & =(x-(2-i\sqrt{2}))(x-(2+i\sqrt{2}))                           \\
	     & =x^2-x(2+i\sqrt{2})-x(2-i\sqrt{2})+(2-i\sqrt{2})(2+i\sqrt{2}) \\
	     & =x^2-4x+(4-2i^2)                                              \\
	     & =x^2-4x+6                                                     
\end{align*}
We conclude that a possible formula for a polynomial function, $p$, 
that has zeros at $2\pm i\sqrt{2}$ is
\[
	p(x)=x^2-4x+6
\]
Note that we could multiply $p$ by any real number and still ensure
that $p$ has the same zeros.
\end{pccsolution}
\end{pccexample}
\investigation*{}
%===================================
%   Author: Hughes
%   Date:   May 2011
%===================================
\begin{problem}[Find a formula from a graph]
For each of the polynomials in \cref{poly:fig:findformula}
\begin{enumerate}
	\item count the number of times the curve turns round, and cuts/bounces off the $x$ axis;
	\item approximate the degree of the polynomial;
	\item use your information to find the linear factors of each polynomial, and therefore write a possible formula for each;
	\item make sure your polynomial goes through the given ordered pair.
\end{enumerate}
\begin{shortsolution}
	\Vref{poly:fig:findformdeg2}: 
	\begin{enumerate}
		\item the curve turns round once;
		\item the degree could be 2;
		\item based on the zeros, the linear factors are $(x+5)$ and $(x-3)$; since the 
		graph opens downwards, we will assume the leading coefficient is negative: $p(x)=-k(x+5)(x-3)$;
		\item $p$ goes through $(2,2)$, so we need to solve $2=-k(7)(-1)$ and therefore $k=\nicefrac{2}{7}$, so
		\[
			p(x)=-\frac{2}{7}(x+5)(x-3)
		\]
	\end{enumerate}
	\Vref{poly:fig:findformdeg3}:
	\begin{enumerate}
		\item the curve turns around twice;
		\item the degree could be 3;
		\item based on the zeros, the linear factors are $(x+2)^2$, and $(x-1)$;
		based on the behavior of $p$, we assume that the leading coefficient is positive, and try $p(x)=k(x+2)^2(x-1)$;
		\item $p$ goes through $(0,-2)$, so we need to solve $-2=k(4)(-1)$ and therefore $k=\nicefrac{1}{2}$, so
		\[
			p(x)=\frac{1}{2}(x+2)^2(x-1)
		\]
	\end{enumerate}
	\Vref{poly:fig:findformdeg5}:
	\begin{enumerate}
		\item the curve turns around 4 times;
		\item the degree could be 5;
		\item based on the zeros, the linear factors are $(x+5)^2$, $(x+1)$, $(x-2)$, $(x-3)$;
		based on the behavior of $p$, we assume that the leading coefficient is positive, and try $p(x)=k(x+5)^2(x+1)(x-2)(x-3)$;
		\item $p$ goes through $(-3,-50)$, so we need to solve $-50=k(64)(-2)(-5)(-6)$ and therefore $k=\nicefrac{5}{384}$, so
		\[
			p(x)=\frac{5}{384}(x+5)^2(x+1)(x-2)(x-3)
		\]
	\end{enumerate}
\end{shortsolution}
\end{problem}


\begin{figure}[!htb]
	\setlength{\figurewidth}{0.3\textwidth}
	\begin{subfigure}{\figurewidth}
		\begin{tikzpicture}
			\begin{axis}[
			   xmin=-5,xmax=5,
			   ymin=-2,ymax=5,
			   width=\textwidth,
			   ]
			   \addplot expression[domain=-4.5:3.75]{-1/3*(x+4)*(x-3)};
			   \addplot[soldot] coordinates{(-4,0)(3,0)(2,2)} node[axisnode,above right]{$(2,2)$};
			\end{axis}
		\end{tikzpicture}
		\caption{}
		\label{poly:fig:findformdeg2}
	\end{subfigure}
	\hfill
	\begin{subfigure}{\figurewidth}
		\begin{tikzpicture}
			\begin{axis}[
			   xmin=-3,xmax=2,
			   ymin=-2,ymax=4,
			   xtick={-2,...,1},
			   width=\textwidth,
			   ]
			   \addplot expression[domain=-2.95:1.75]{1/3*(x+2)^2*(x-1)};
			   \addplot[soldot]coordinates{(-2,0)(1,0)(0,-1.33)}node[axisnode,anchor=north west]{$(0,-2)$};
			\end{axis}
		\end{tikzpicture}
		\caption{}
		\label{poly:fig:findformdeg3}
	\end{subfigure}
	\hfill
	\begin{subfigure}{\figurewidth}
		\begin{tikzpicture}
			\begin{axis}[
			   xmin=-5,xmax=5,
			   ymin=-100,ymax=150,
			   width=\textwidth,
			   ]
			   \addplot expression[domain=-4.5:3.4,samples=50]{(x+4)^2*(x+1)*(x-2)*(x-3)};
			   \addplot[soldot]coordinates{(-4,0)(-1,0)(2,0)(3,0)(-3,-60)}node[axisnode,anchor=north]{$(-3,-50)$};
			\end{axis}
		\end{tikzpicture}
		\caption{}
		\label{poly:fig:findformdeg5}
	\end{subfigure}
	\caption{}
	\label{poly:fig:findformula}
\end{figure}




\begin{exercises}
%===================================
%   Author: Hughes
%   Date:   March 2012
%===================================
\begin{problem}[Prerequisite classifacation skills]
Decide if each of the following functions are linear or quadratic.
\begin{multicols}{3}
	\begin{subproblem}
		$f(x)=2x+3$ 
		\begin{shortsolution}
			$f$ is linear. 
		\end{shortsolution}
	\end{subproblem}
	\begin{subproblem}
		$g(x)=10-7x$ 
		\begin{shortsolution}
			$g$ is linear 
		\end{shortsolution}
	\end{subproblem}
	\begin{subproblem}
		$h(x)=-x^2+3x-9$ 
		\begin{shortsolution}
			$h$ is quadratic. 
		\end{shortsolution}
	\end{subproblem}
	\begin{subproblem}
		$k(x)=-17$ 
		\begin{shortsolution}
			$k$ is linear.
		\end{shortsolution}
	\end{subproblem}
	\begin{subproblem}
		$l(x)=-82x^2-4$
		\begin{shortsolution}
			$l$ is quadratic 
		\end{shortsolution}
	\end{subproblem}
	\begin{subproblem}
		$m(x)=6^2x-8$ 
		\begin{shortsolution}
			$m$ is linear. 
		\end{shortsolution}
	\end{subproblem}
\end{multicols}
\end{problem}
%===================================
%   Author: Hughes
%   Date:   March 2012
%===================================
\begin{problem}[Prerequisite slope identification]
State the slope of each of the following linear functions, and 
hence decide if each function is increasing or decreasing.
\begin{multicols}{4}
	\begin{subproblem}
		$\alpha(x)=4x+1$ 
		\begin{shortsolution}
			$m=4$; $\alpha$ is increasing. 
		\end{shortsolution}
	\end{subproblem}
	\begin{subproblem}
		$\beta(x)=-9x$ 
		\begin{shortsolution}
			$m=-9$; $\beta$ is decreasing. 
		\end{shortsolution}
	\end{subproblem}
	\begin{subproblem}
		$\gamma(t)=18t+100$ 
		\begin{shortsolution}
			$m=18$; $\gamma$ is increasing.
		\end{shortsolution}
	\end{subproblem}
	\begin{subproblem}
		$\delta(y)=23-y$ 
		\begin{shortsolution}
			$m=-1$; $\delta$ is decreasing. 
		\end{shortsolution}
	\end{subproblem}
\end{multicols}
Now let's generalize our findings for the most general linear function $f$
that has formula $f(x)=mx+b$. Complete the following sentences.
\begin{subproblem}
	When $m>0$, the function $f$ is $\ldots$  
	\begin{shortsolution}
		When $m>0$, the function $f$ is $\ldots$  \emph{increasing}.
	\end{shortsolution}
\end{subproblem}
\begin{subproblem}
	When $m<0$, the function $f$ is $\ldots$  
	\begin{shortsolution}
		When $m<0$, the function $f$ is $\ldots$  \emph{decreasing}.
	\end{shortsolution}
\end{subproblem}
\end{problem}
%===================================
%   Author: Hughes
%   Date:   May 2011
%===================================
\begin{problem}[Polynomial or not?]
Identify whether each of the following functions is a polynomial or not. 
If the function is a polynomial, state its degree.
\begin{multicols}{3}
	\begin{subproblem}
		$p(x)=2x+1$
		\begin{shortsolution}
			$p$ is a polynomial (you might also describe $p$ as linear). The degree of $p$ is 1.
		\end{shortsolution}
	\end{subproblem}
	\begin{subproblem}
		$p(x)=7x^2+4x$
		\begin{shortsolution}
			$p$ is a polynomial (you might also describe $p$ as quadratic). The degree of $p$ is 2.
		\end{shortsolution}
	\end{subproblem}
	\begin{subproblem}
		$p(x)=\sqrt{x}+2x+1$
		\begin{shortsolution}
			$p$ is not a polynomial; we require the powers of $x$ to be integer values.
		\end{shortsolution}
	\end{subproblem}
	\begin{subproblem}
		$p(x)=2^x-45$
		\begin{shortsolution}
			$p$ is not a polynomial; the $2^x$ term is exponential.
		\end{shortsolution}
	\end{subproblem}
	\begin{subproblem}
		$p(x)=6x^4-5x^3+9$
		\begin{shortsolution}
			$p$ is a polynomial, and the degree of $p$ is $6$.
		\end{shortsolution}
	\end{subproblem}
	\begin{subproblem}
		$p(x)=-5x^{17}+9x+2$
		\begin{shortsolution}
			$p$ is a polynomial, and the degree of $p$ is 17.
		\end{shortsolution}
	\end{subproblem}
	\begin{subproblem}
		$p(x)=4x(x+7)^2(x-3)^3$ 
		\begin{shortsolution}
			$p$ is a polynomial, and the degree of $p$ is $6$.     
		\end{shortsolution}
	\end{subproblem}
	\begin{subproblem}
		$p(x)=4x^{-5}-x^2+x$ 
		\begin{shortsolution}
			$p$ is not a polynomial because $-5$ is not a positive integer. 
		\end{shortsolution}
	\end{subproblem}
	\begin{subproblem}
		$p(x)=-x^6(x^2+1)(x^3-2)$ 
		\begin{shortsolution}
			$p$ is a polynomial, and the degree of $p$ is $11$.
		\end{shortsolution}
	\end{subproblem}
\end{multicols}
\end{problem}
%===================================
%   Author: Hughes
%   Date:   May 2011
%===================================
\begin{problem}[Polynomial graphs]
Three polynomial functions $p$, $m$, and $n$ are shown in \crefrange{poly:fig:functionp}{poly:fig:functionn}.
The functions have the following formulas
\begin{align*}
	p(x) & = (x-1)(x+2)(x-3)           \\ 
	m(x) & = -(x-1)(x+2)(x-3)          \\
	n(x) & = (x-1)(x+2)(x-3)(x+1)(x+4) 
\end{align*}
Note that for our present purposes we are not concerned with the vertical scale of the graphs.
\begin{subproblem}
	Identify both on the graph {\em and} algebraically, the zeros of each polynomial.
	\begin{shortsolution}
		$y=p(x)$ is shown below.
		
		\begin{tikzpicture}
			\begin{axis}[
			   xmin=-5,xmax=5,
			   ymin=-10,ymax=10,
			   width=\solutionfigurewidth,
			   ]
			   \addplot expression[domain=-2.5:3.5,samples=50]{(x-1)*(x+2)*(x-3)};
			   \addplot[soldot] coordinates{(-2,0)(1,0)(3,0)};
			\end{axis}
		\end{tikzpicture}
		
		$y=m(x)$ is shown below.
		
		\begin{tikzpicture}
			\begin{axis}[
			   xmin=-5,xmax=5,
			   ymin=-10,ymax=10,
			   width=\solutionfigurewidth,
			   ]
			   \addplot expression[domain=-2.5:3.5,samples=50]{-1*(x-1)*(x+2)*(x-3)};
			   \addplot[soldot] coordinates{(-2,0)(1,0)(3,0)};
			\end{axis}
		\end{tikzpicture}
		
		$y=n(x)$ is shown below.
		
		\begin{tikzpicture}
			\begin{axis}[
			   xmin=-5,xmax=5,
			   ymin=-90,ymax=70,
			   width=\solutionfigurewidth,
			   ]
			   \addplot expression[domain=-4.15:3.15,samples=50]{(x-1)*(x+2)*(x-3)*(x+1)*(x+4)};
			   \addplot[soldot] coordinates{(-4,0)(-2,0)(-1,0)(1,0)(3,0)};
			\end{axis}
		\end{tikzpicture}
		
		The zeros of $p$ are $-2$, $1$, and $3$; the zeros of $m$ are $-2$, $1$, and $3$; the zeros of $n$ are 
		$-4$, $-2$, $-1$, and $3$.
	\end{shortsolution}
\end{subproblem}
\begin{subproblem}
	Write down the degree, how many times the curve of each function `turns around', 
	and how many zeros it has
	\begin{shortsolution}
		\begin{itemize}
			\item The degree of $p$ is 3, and the curve $y=p(x)$ turns around twice. 
			\item The degree of $q$ is also 3, and the curve $y=q(x)$ turns around twice. 
			\item The degree of $n$ is $5$, and the curve $y=n(x)$ turns around 4 times.
		\end{itemize}
	\end{shortsolution}
\end{subproblem}
\end{problem}

\begin{figure}[!htb]
	\begin{widepage}
	\setlength{\figurewidth}{0.3\textwidth}
	\begin{subfigure}{\figurewidth}
		\begin{tikzpicture}
			\begin{axis}[
			   xmin=-5,xmax=5,
			   ymin=-10,ymax=10,
			   ytick={-5,5},
			   width=\textwidth,
			   ]
			   \addplot expression[domain=-2.5:3.5,samples=50]{(x-1)*(x+2)*(x-3)};
			   \addplot[soldot]coordinates{(-2,0)(1,0)(3,0)};
			\end{axis}
		\end{tikzpicture}
		    \caption{$y=p(x)$}
		\label{poly:fig:functionp}
	\end{subfigure}
	\hfill
	\begin{subfigure}{\figurewidth}
		\begin{tikzpicture}
			\begin{axis}[
			   xmin=-5,xmax=5,
			   ymin=-10,ymax=10,
			   ytick={-5,5},
			   width=\textwidth,
			   ]
			   \addplot expression[domain=-2.5:3.5,samples=50]{-1*(x-1)*(x+2)*(x-3)};
			   \addplot[soldot]coordinates{(-2,0)(1,0)(3,0)};
			\end{axis}
		\end{tikzpicture}
		\caption{$y=m(x)$}
		\label{poly:fig:functionm}
	\end{subfigure}
	\hfill
	\begin{subfigure}{\figurewidth}
		\begin{tikzpicture}
			\begin{axis}[
			   xmin=-5,xmax=5,
			   ymin=-90,ymax=70,
			   width=\textwidth,
			   ]
			   \addplot expression[domain=-4.15:3.15,samples=100]{(x-1)*(x+2)*(x-3)*(x+1)*(x+4)};
			   \addplot[soldot]coordinates{(-4,0)(-2,0)(-1,0)(1,0)(3,0)};
			\end{axis}
		\end{tikzpicture}
		        \caption{$y=n(x)$}
		\label{poly:fig:functionn}
	\end{subfigure}
	\caption{}
	\end{widepage}
\end{figure}
%===================================
%   Author: Hughes
%   Date:   May 2011
%===================================
\begin{problem}[Horizontal intercepts]\label{poly:prob:matchpolys}%
State the horizontal intercepts (as ordered pairs) of the following polynomials.
\begin{multicols}{2}
	\begin{subproblem}\label{poly:prob:degree5}
		$p(x)=(x-1)(x+2)(x-3)(x+1)(x+4)$
		\begin{shortsolution}
			$(-4,0)$, $(-2,0)$, $(-1,0)$, $(1,0)$, $(3,0)$
		\end{shortsolution}
	\end{subproblem}
	\begin{subproblem}
		$q(x)=-(x-1)(x+2)(x-3)$
		\begin{shortsolution}
			$(-2,0)$, $(1,0)$, $(3,0)$
		\end{shortsolution}
	\end{subproblem}
	\begin{subproblem}
		$r(x)=(x-1)(x+2)(x-3)$
		\begin{shortsolution}
			$(-2,0)$, $(1,0)$, $(3,0)$
		\end{shortsolution}
	\end{subproblem}
	\begin{subproblem}\label{poly:prob:degree2}
		$s(x)=(x-2)(x+2)$
		\begin{shortsolution}
			$(-2,0)$, $(2,0)$
		\end{shortsolution}
	\end{subproblem}
\end{multicols}
\end{problem}
%===================================
%   Author: Hughes
%   Date:   March 2012
%===================================
\begin{problem}[Minimums, maximums, and concavity]\label{poly:prob:incdec}
Four polynomial functions are graphed in \cref{poly:fig:incdec}. The formulas
for these functions are (not respectively)
\begin{gather*}
	p(x)=\frac{x^3}{6}-\frac{x^2}{4}-3x, \qquad q(x)=\frac{x^4}{20}+\frac{x^3}{15}-\frac{6}{5}x^2+1\\
	r(x)=-\frac{x^5}{50}-\frac{x^4}{40}+\frac{2x^3}{5}+6, \qquad s(x)=-\frac{x^6}{6000}-\frac{x^5}{2500}+\frac{67x^4}{4000}+\frac{17x^3}{750}-\frac{42x^2}{125}
\end{gather*}
\begin{figure}[!htb]
	\begin{widepage}
	\setlength{\figurewidth}{.23\textwidth}
	\centering
	\begin{subfigure}{\figurewidth}
		\begin{tikzpicture}
			\begin{axis}[
			   framed,
			   width=\textwidth,
			   xmin=-10,xmax=10,
			   ymin=-10,ymax=10,
			   xtick={-8,-6,...,8},
			   ytick={-8,-6,...,8},
			   grid=major,
			   ]
			   \addplot expression[domain=-5.28:4.68,samples=50]{-x^5/50-x^4/40+2*x^3/5+6};
			\end{axis}
		\end{tikzpicture}
		\caption{}
		\label{poly:fig:incdec3}
	\end{subfigure}
	\hfill
	\begin{subfigure}{\figurewidth}
		\begin{tikzpicture}
			\begin{axis}[
			   framed,
			   width=\textwidth,
			   xmin=-10,xmax=10,ymin=-10,ymax=10,
			   xtick={-8,-6,...,8},
			   ytick={-8,-6,...,8},
			   grid=major,
			   ]
			   \addplot expression[domain=-6.08:4.967,samples=50]{x^4/20+x^3/15-6/5*x^2+1};
			\end{axis}
		\end{tikzpicture}
		\caption{}
		\label{poly:fig:incdec2}
	\end{subfigure}
	\hfill
	\begin{subfigure}{\figurewidth}
		\begin{tikzpicture}
			\begin{axis}[
			   framed,
			   width=\textwidth,
			   xmin=-6,xmax=8,ymin=-10,ymax=10,
			   xtick={-4,-2,...,6},
			   ytick={-8,-4,4,8},
			   minor ytick={-6,-2,...,6},
			   grid=both,
			   ]
			   \addplot expression[domain=-4.818:6.081,samples=50]{x^3/6-x^2/4-3*x};
			\end{axis}
		\end{tikzpicture}
		\caption{}
		\label{poly:fig:incdec1}
	\end{subfigure}
	\hfill
	\begin{subfigure}{\figurewidth}
		\begin{tikzpicture}
			\begin{axis}[
			   framed,
			   width=\textwidth,
			   xmin=-10,xmax=10,ymin=-10,ymax=10,
			   xtick={-8,-4,4,8},
			   ytick={-8,-4,4,8},
			   minor xtick={-6,-2,...,6},
			   minor ytick={-6,-2,...,6},
			   grid=both,
			   ]
			   \addplot expression[domain=-9.77:8.866,samples=50]{-x^6/6000-x^5/2500+67*x^4/4000+17/750*x^3-42/125*x^2};
			\end{axis}
		\end{tikzpicture}
		\caption{}
		\label{poly:fig:incdec4}
	\end{subfigure}
	\caption{Graphs for \cref{poly:prob:incdec}.}
	\label{poly:fig:incdec}
	\end{widepage}
\end{figure}
\begin{subproblem}
	Match each of the formulas with one of the given graphs.
	\begin{shortsolution}
		\begin{itemize}
			\item $p$ is graphed in \vref{poly:fig:incdec1};
			\item $q$ is graphed in \vref{poly:fig:incdec2};
			\item $r$ is graphed in \vref{poly:fig:incdec3};
			\item $s$ is graphed in \vref{poly:fig:incdec4}.
		\end{itemize}
	\end{shortsolution}
\end{subproblem}
\begin{subproblem}
	Approximate the zeros of each function using the appropriate graph.
	\begin{shortsolution}
		\begin{itemize}
			\item $p$ has simple zeros at about $-3.8$, $0$, and $5$.
			\item $q$ has simple zeros at about $-5.9$, $-1$, $1$, and $4$.
			\item $r$ has simple zeros at about $-5$, $-2.9$, and $4.1$.
			\item $s$ has simple zeros at about $-9$, $-6$, $4.2$, $8.1$, and a zero of multiplicity $2$ at $0$.
		\end{itemize}
	\end{shortsolution}
\end{subproblem}
\begin{subproblem}
	Approximate the local maximums and minimums of each of the functions.
	\begin{shortsolution}
		\begin{itemize}
			\item $p$ has a local maximum of approximately $3.9$ at $-2$, and a local minimum of approximately $-6.5$ at $3$.
			\item $q$ has a local minimum of approximately $-10$ at $-4$, and $-4$ at $3$; $q$ has a local maximum of approximately $1$ at $0$.
			\item $r$ has a local minimum of approximately $-5.5$ at $-4$, and a local maximum of approximately $10$ at $3$.
			\item $s$ has a local maximum of approximately $5$ at $-8$, $0$ at $0$, and $5$ at  $7$; $s$ has local minimums 
			of approximately $-3$ at $-4$, and $-1$ at $3$.
		\end{itemize}
	\end{shortsolution}
\end{subproblem}
\begin{subproblem}
	Approximate the global maximums and minimums of each of the functions.
	\begin{shortsolution}
		\begin{itemize}
			\item $p$ does not have a global maximum, nor a global minimum.
			\item $q$ has a global minimum of approximately $-10$; it does not have a global maximum.
			\item $r$ does not have a global maximum, nor a global minimum.
			\item $s$ has a global maximum of approximately $5$; it does not have a global minimum.
		\end{itemize}
	\end{shortsolution}
\end{subproblem}
\begin{subproblem}
	Approximate the intervals on which each function is increasing and decreasing.
	\begin{shortsolution}
		\begin{itemize}
			\item $p$ is increasing on $(-\infty,-2)\cup (3,\infty)$, and decreasing on $(-2,3)$.
			\item $q$ is increasing on $(-4,0)\cup (3,\infty)$, and decreasing on $(-\infty,-4)\cup (0,3)$.
			\item $r$ is increasing on $(-4,3)$, and decreasing on $(-\infty,-4)\cup (3,\infty)$.
			\item $s$ is increasing on $(-\infty,-8)\cup (-4,0)\cup (3,5)$, and decreasing on $(-8,-4)\cup (0,3)\cup (5,\infty)$.
		\end{itemize}
	\end{shortsolution}
\end{subproblem}
\begin{subproblem}
	Approximate the intervals on which each function is concave up and concave down.
	\begin{shortsolution}
		\begin{itemize}
			\item $p$ is concave up on  $(1,\infty)$, and concave down on  $(-\infty,1)$.
			\item $q$ is concave up on $(-\infty,-1)\cup (1,\infty)$, and concave down on $(-1,1)$.
			\item $r$ is concave up on $(-\infty,-3)\cup (0,2)$, and concave down on $(-3,0)\cup (2,\infty)$.
			\item $s$ is concave up on $(-6,-2)\cup (2,5)$, and concave down on $(-\infty,-6)\cup (-2,2)\cup (5,\infty)$.
		\end{itemize}
	\end{shortsolution}
\end{subproblem}
\begin{subproblem}
	The degree of $q$ is $5$. Assuming that all of the real zeros of $q$ are
	shown in its graph, how many complex zeros does $q$ have?
	\begin{shortsolution}
		\Vref{poly:fig:incdec2} shows that $q$ has $3$ real zeros 
		since the curve of $q$ cuts the horizontal axis $3$ times. 
		Since $q$ has degree $5$, $q$ must have $2$ complex zeros.
	\end{shortsolution}
\end{subproblem}
\end{problem}

%===================================
%   Author: Hughes
%   Date:   May 2011
%===================================
\begin{problem}[Long-run behaviour of polynomials]
Describe the long-run behavior of each of polynomial functions in 
\crefrange{poly:prob:degree5}{poly:prob:degree2}.
\begin{shortsolution}
	$\dd\lim_{x\rightarrow-\infty}p(x)=-\infty$,
	$\dd\lim_{x\rightarrow\infty}p(x)=\infty$,
	$\dd\lim_{x\rightarrow-\infty}q(x)=\infty$,
	$\dd\lim_{x\rightarrow\infty}q(x)=-\infty$,
	$\dd\lim_{x\rightarrow-\infty}r(x)=-\infty$,
	$\dd\lim_{x\rightarrow\infty}r(x)=\infty$,
	$\dd\lim_{x\rightarrow-\infty}s(x)=\infty$,
	$\dd\lim_{x\rightarrow\infty}s(x)=\infty$,
\end{shortsolution}
\end{problem}

%===================================
%   Author: Hughes
%   Date:   May 2011
%===================================
\begin{problem}[True of false?]
Let $p$ be a polynomial function. 
Label each of the following statements as true (T) or false (F); if they are false, 
provide an example that supports your answer.
\begin{subproblem}
	If $p$ has degree $3$, then $p$ has $3$ distinct zeros.
	\begin{shortsolution}
		False. Consider $p(x)=x^2(x+1)$ which has only 2 distinct zeros.
	\end{shortsolution}
\end{subproblem}
\begin{subproblem}
	If $p$ has degree $4$, then $\dd\lim_{x\rightarrow-\infty}p(x)=\infty$ and $\dd\lim_{x\rightarrow\infty}p(x)=\infty$.
	\begin{shortsolution}
		False. Consider $p(x)=-x^4$.
	\end{shortsolution}
\end{subproblem}
\begin{subproblem}
	If $p$ has even degree, then it is possible that $p$ can have no real zeros.
	\begin{shortsolution}
		True.
	\end{shortsolution}
\end{subproblem}
\begin{subproblem}
	If $p$ has odd degree, then it is possible that $p$ can have no real zeros.
	\begin{shortsolution}
		False. All odd degree polynomials will cut the horizontal axis at least once.
	\end{shortsolution}
\end{subproblem}
\end{problem}
%===================================
%   Author: Hughes
%   Date:   May 2011
%===================================
\begin{problem}[Find a formula from a description]
In each of the following problems, give a possible formula for a polynomial 
function that has the specified properties.
\begin{subproblem}
	 Degree 2 and has zeros at $4$ and $5$.
	\begin{shortsolution}
		Possible option: $p(x)=(x-4)(x-5)$. Note we could multiply $p$ by any real number, and still meet the requirements.
	\end{shortsolution}
\end{subproblem}
\begin{subproblem}
	 Degree 3 and has zeros at $4$,$5$ and $-3$.
	\begin{shortsolution}
		Possible option: $p(x)=(x-4)(x-5)(x+3)$. Note we could multiply $p$ by any real number, and still meet the requirements.
	\end{shortsolution}
\end{subproblem}
\begin{subproblem}
	 Degree 4 and has zeros at $0$, $4$, $5$, $-3$.
	\begin{shortsolution}
		Possible option: $p(x)=x(x-4)(x-5)(x+3)$. Note we could multiply $p$ by any real number, and still meet the requirements.
	\end{shortsolution}
\end{subproblem}
\begin{subproblem}
	Degree 4, with zeros that make the graph cut at $2$, $-5$, and a zero that makes the graph touch at $-2$;
	\begin{shortsolution}
		Possible option: $p(x)=(x-2)(x+5)(x+2)^2$. Note we could multiply $p$ by any real number, and still meet the requirements.
	\end{shortsolution}
\end{subproblem}
\begin{subproblem}
	Degree 3, with only one zero at $-1$.
	\begin{shortsolution}
		Possible option: $p(x)=(x+1)^3$. Note we could multiply $p$ by any real number, and still meet the requirements.
	\end{shortsolution}
\end{subproblem}
\end{problem}
%===================================
%   Author: Hughes
%   Date:   June 2012
%===================================
\begin{problem}[\Cref{poly:step:last}]
\pccname{Saheed} is graphing a polynomial function, $p$. 
He is following \crefrange{poly:step:first}{poly:step:last} and has so far
marked the zeros of $p$ on \cref{poly:fig:optionsp1}. Saheed tells you that 
$p$ has degree $3$, but does \emph{not} say if the leading coefficient 
of $p$ is positive or negative.
\begin{figure}[!htbp]
	\begin{widepage}
	\begin{subfigure}{.45\textwidth}
		\begin{tikzpicture}
			\begin{axis}[
			   xmin=-10,xmax=10,
			   ymin=-10,ymax=10,
			   xtick={-8,-6,...,8},
			   ytick={-15},
			   width=\textwidth,
			   height=.5\textwidth,
			   ]
			   \addplot[soldot] coordinates{(-5,0)(2,0)(6,0)};
			\end{axis}
		\end{tikzpicture}
		\caption{}
		\label{poly:fig:optionsp1}
	\end{subfigure}%
	\hfill
	\begin{subfigure}{.45\textwidth}
		\begin{tikzpicture}
			\begin{axis}[
			   xmin=-10,xmax=10,
			   ymin=-10,ymax=10,
			   xtick={-8,-6,...,8},
			   ytick={-15},
			   width=\textwidth,
			   height=.5\textwidth,
			   ]
			   \addplot[soldot] coordinates{(-5,0)(6,0)};
			\end{axis}
		\end{tikzpicture}
		\caption{}
		\label{poly:fig:optionsp2}
	\end{subfigure}%
	\caption{}
	\end{widepage}
\end{figure}
\begin{subproblem}
	Use the information in \cref{poly:fig:optionsp1} to help sketch $p$, assuming that the leading coefficient
	is positive.
	\begin{shortsolution}
		Assuming that $a_3>0$:
		
		\begin{tikzpicture}
			\begin{axis}[
			   xmin=-10,xmax=10,
			   ymin=-10,ymax=10,
			   xtick={-8,-6,...,8},
			   ytick={-15},
			   width=\solutionfigurewidth,
			   ]
			   \addplot expression[domain=-6.78179:8.35598,samples=50]{1/20*(x+5)*(x-2)*(x-6)};
			   \addplot[soldot] coordinates{(-5,0)(2,0)(6,0)};
			\end{axis}
		\end{tikzpicture}
	\end{shortsolution}
\end{subproblem}
\begin{subproblem}
	Use the information in \cref{poly:fig:optionsp1} to help sketch $p$, assuming that the leading coefficient
	is negative.
	\begin{shortsolution}
		Assuming that $a_3<0$:
		
		\begin{tikzpicture}
			\begin{axis}[
			   xmin=-10,xmax=10,
			   ymin=-10,ymax=10,
			   xtick={-8,-6,...,8},
			   ytick={-15},
			   width=\solutionfigurewidth,
			   ]
			   \addplot expression[domain=-6.78179:8.35598,samples=50]{-1/20*(x+5)*(x-2)*(x-6)};
			   \addplot[soldot] coordinates{(-5,0)(2,0)(6,0)};
			\end{axis}
		\end{tikzpicture}
	\end{shortsolution}
\end{subproblem}
Saheed now turns his attention to another polynomial function, $q$. He finds
the zeros of $q$ (there are only $2$) and marks them on \cref{poly:fig:optionsp2}. 
Saheed knows that $q$ has degree $3$, but doesn't know if the leading 
coefficient is positive or negative.
\begin{subproblem}
	Use the information in \cref{poly:fig:optionsp2} to help sketch $q$, assuming that the leading 
	coefficient of $q$ is positive. Hint: only one of the zeros is simple.
	\begin{shortsolution}
		Assuming that $a_4>0$ there are $2$ different options:
		
		\begin{tikzpicture}
			\begin{axis}[
			   xmin=-10,xmax=10,
			   ymin=-10,ymax=10,
			   xtick={-8,-6,...,8},
			   ytick={-15},
			   width=\solutionfigurewidth,
			   ]
			   \addplot expression[domain=-8.68983:7.31809,samples=50]{1/20*(x+5)^2*(x-6)};
			   \addplot expression[domain=-6.31809:9.68893,samples=50]{1/20*(x+5)*(x-6)^2};
			   \addplot[soldot] coordinates{(-5,0)(6,0)};
			\end{axis}
		\end{tikzpicture}
	\end{shortsolution}
\end{subproblem}
\begin{subproblem}
	Use the information in \cref{poly:fig:optionsp2} to help sketch $q$, assuming that the leading 
	coefficient of $q$ is negative.
	\begin{shortsolution}
		Assuming that $a_4<0$ there are $2$ different options:
		
		\begin{tikzpicture}
			\begin{axis}[
			   xmin=-10,xmax=10,
			   ymin=-10,ymax=10,
			   xtick={-8,-6,...,8},
			   ytick={-15},
			   width=\solutionfigurewidth,
			   ]
			   \addplot expression[domain=-8.68983:7.31809,samples=50]{-1/20*(x+5)^2*(x-6)};
			   \addplot expression[domain=-6.31809:9.68893,samples=50]{-1/20*(x+5)*(x-6)^2};
			   \addplot[soldot] coordinates{(-5,0)(6,0)};
			\end{axis}
		\end{tikzpicture}
	\end{shortsolution}
\end{subproblem}
\end{problem}
%===================================
%   Author: Hughes
%   Date:   June 2012
%===================================
\begin{problem}[Zeros]
Find all zeros of each of the following polynomial functions, making 
sure to detail their multiplicity. Note that
you may need to use factoring, or the quadratic formula, or both! Also note 
that some zeros may be repeated, and some may be complex.
\begin{multicols}{3}
	\begin{subproblem}
		$p(x)=x^2+1$ 
		\begin{shortsolution}
			$\pm i$ (simple).
		\end{shortsolution}
	\end{subproblem}
	\begin{subproblem}
		$q(y)=(y^2-9)(y^2-7)$ 
		\begin{shortsolution}
			$\pm 3$, $\pm \sqrt{7}$ (all are simple).
		\end{shortsolution}
	\end{subproblem}
	\begin{subproblem}
		$r(z)=-4z^3(z^2+3)(z^2+64)$ 
		\begin{shortsolution}
			$0$ (multiplicity $3$), $\pm\sqrt{3}$ (simple), $\pm\sqrt{8}$ (simple).
		\end{shortsolution}
	\end{subproblem}
	\begin{subproblem}
		$a(x)=x^4-81$ 
		\begin{shortsolution}
			$\pm 3$, $\pm 3i$ (all are simple).
		\end{shortsolution}
	\end{subproblem}
	\begin{subproblem}
		$b(y)=y^3-8$ 
		\begin{shortsolution}
			$2$, $-1\pm i\sqrt{3}$ (all are simple).
		\end{shortsolution}
	\end{subproblem}
	\begin{subproblem}
		$c(m)=m^3-m^2$
		\begin{shortsolution}
			$0$ (multiplicity $2$), $1$ (simple).
		\end{shortsolution}
	\end{subproblem}
	\begin{subproblem}
		$h(n)=(n+1)(n^2+4)$ 
		\begin{shortsolution}
			$-1$, $\pm 2i$ (all are simple).
		\end{shortsolution}
	\end{subproblem}
	\begin{subproblem}
		$f(\alpha)=(\alpha^2-16)(\alpha^2-5\alpha+4)$ 
		\begin{shortsolution}
			$-4$ (simple), $4$ (multiplicity $2$), $1$ (simple).
		\end{shortsolution}
	\end{subproblem}
	\begin{subproblem}
		$g(\beta)=(\beta^2-25)(\beta^2-5\beta-4)$ 
		\begin{shortsolution}
			$\pm 5$, $\dfrac{5\pm\sqrt{41}}{2}$ (all are simple).
		\end{shortsolution}
	\end{subproblem}
\end{multicols}
\end{problem}
%===================================
%   Author: Hughes
%   Date:   June 2012
%===================================
\begin{problem}[Given zeros, find a formula]
In each of the following problems you are given the zeros of a polynomial. 
Write a possible formula for each polynomial| you may leave your 
answer in factored form, but it may not contain complex numbers. Unless
otherwise stated, assume that the zeros are simple.
\begin{multicols}{3}
	\begin{subproblem}
		$1$, $2$ 
		\begin{shortsolution}
			$p(x)=(x-1)(x-2)$
		\end{shortsolution}
	\end{subproblem}
	\begin{subproblem}
		$0$, $5$, $13$ 
		\begin{shortsolution}
			$p(x)=x(x-5)(x-13)$
		\end{shortsolution}
	\end{subproblem}
	\begin{subproblem}
		$-7$, $2$ (multiplicity $3$), $5$ 
		\begin{shortsolution}
			$p(x)=(x+7)(x-2)^3(x-5)$
		\end{shortsolution}
	\end{subproblem}
	\begin{subproblem}
		$0$, $\pm i$ 
		\begin{shortsolution}
			$p(x)=x(x^2+1)$
		\end{shortsolution}
	\end{subproblem}
	\begin{subproblem}
		$\pm 2i$, $\pm 7$ 
		\begin{shortsolution}
			$p(x)=(x^2+4)(x^2-49)$
		\end{shortsolution}
	\end{subproblem}
	\begin{subproblem}
		$-2\pm i\sqrt{6}$ 
	\end{subproblem}
\end{multicols}
\end{problem}
%===================================
%   Author: Hughes
%   Date:   June 2012
%===================================
\begin{problem}[Composition of polynomials]
Let $p$ and $q$ be polynomial functions that have formulas
\[
	p(x)=(x+1)(x+2)(x+5), \qquad q(x)=3-x^4
\]
Evaluate each of the following.
\begin{multicols}{4}
	\begin{subproblem}
		$(p\circ q)(0)$ 
		\begin{shortsolution}
			$160$
		\end{shortsolution}
	\end{subproblem}
	\begin{subproblem}
		$(q\circ p)(0)$ 
		\begin{shortsolution}
			$-9997$
		\end{shortsolution}
	\end{subproblem}
	\begin{subproblem}
		$(p\circ q)(1)$ 
		\begin{shortsolution}
			$84$
		\end{shortsolution}
	\end{subproblem}
	\begin{subproblem}
		$(p\circ p)(0)$ 
		\begin{shortsolution}
			$1980$
		\end{shortsolution}
	\end{subproblem}
\end{multicols}
\end{problem}
%===================================
%   Author: Hughes
%   Date:   June 2012
%===================================
\begin{problem}[Piecewise polynomial functions]
Let $P$ be the piecewise-defined function with formula
\[
	P(x)=\begin{cases}
	(1-x)(2x+5)(x^2+1), &   x\leq -3\\
	4-x^2,              &   -3<x < 4\\
	x^3                 &   x\geq 4
\end{cases}
\]
Evaluate each of the following 
\begin{multicols}{5}
	\begin{subproblem}
		$P(-4)$ 
		\begin{shortsolution}
			$-255$
		\end{shortsolution}
	\end{subproblem}
	\begin{subproblem}
		$P(0)$ 
		\begin{shortsolution}
			$4$
		\end{shortsolution}
	\end{subproblem}
	\begin{subproblem}
		$P(4)$ 
		\begin{shortsolution}
			$64$
		\end{shortsolution}
	\end{subproblem}
	\begin{subproblem}
		$P(-3)$ 
		\begin{shortsolution}
			$-40$
		\end{shortsolution}
	\end{subproblem}
	\begin{subproblem}
		$(P\circ P)(0)$ 
		\begin{shortsolution}
			$64$
		\end{shortsolution}
	\end{subproblem}
\end{multicols}
\end{problem}

%===================================
%   Author: Hughes
%   Date:   July 2012
%===================================
\begin{problem}[Function algebra]
Let $p$ and $q$ be the polynomial functions that have formulas
\[
	p(x)=x(x+1)(x-3)^2, \qquad q(x)=7-x^2
\]
Evaluate each of the following (if possible).
\begin{multicols}{4}
	\begin{subproblem}
		$(p+q)(1)$ 
		\begin{shortsolution}
			$14$ 
		\end{shortsolution}
	\end{subproblem}
	\begin{subproblem}
		$(p-q)(0)$ 
		\begin{shortsolution}
			$7$ 
		\end{shortsolution}
	\end{subproblem}
	\begin{subproblem}
		$(p\cdot q)(\sqrt{7})$ 
		\begin{shortsolution}
			$0$ 
		\end{shortsolution}
	\end{subproblem}
	\begin{subproblem}
		$\left( \frac{q}{p} \right)(1)$ 
		\begin{shortsolution}
			$\frac{3}{4}$ 
		\end{shortsolution}
	\end{subproblem}
\end{multicols}
\begin{subproblem}
	What is the domain of the function $\frac{q}{p}$?
	\begin{shortsolution}
		$(-\infty,-1)\cup (-1,0)\cup (0,3)\cup (3,\infty)$ 
	\end{shortsolution}
\end{subproblem}
\end{problem}

%===================================
%   Author: Hughes
%   Date:   July 2012
%===================================
\begin{problem}[Transformations: given the transformation, find the formula]
Let $p$ be the polynomial function that has formula.
\[
	p(x)=4x(x^2-1)(x+3)
\]
In each of the following 
problems apply the given transformation to the function $p$ and 
write a formula for the transformed version of $p$.
\begin{multicols}{2}
	\begin{subproblem}
		Shift $p$ to the right by $5$ units. 
		\begin{shortsolution}
			$p(x-5)=4(x-5)(x-2)(x^2-10x+24)$
		\end{shortsolution}
	\end{subproblem}
	\begin{subproblem}
		Shift $p$ to the left by $6$ units. 
		\begin{shortsolution}
			$p(x+6)=4(x+6)(x+9)(x^2+12x+35)$
		\end{shortsolution}
	\end{subproblem}
	\begin{subproblem}
		Shift $p$ up by $12$ units. 
		\begin{shortsolution}
			$p(x)+12=4x(x^2-1)(x+3)+12$
		\end{shortsolution}
	\end{subproblem}
	\begin{subproblem}
		Shift $p$ down by $2$ units. 
		\begin{shortsolution}
			$p(x)-2=4x(x^2-1)(x+3)-2$
		\end{shortsolution}
	\end{subproblem}
	\begin{subproblem}
		Reflect $p$ over the horizontal axis.
		\begin{shortsolution}
			$-p(x)=-4x(x^2-1)(x+3)$
		\end{shortsolution}
	\end{subproblem}
	\begin{subproblem}
		Reflect $p$ over the vertical axis.
		\begin{shortsolution}
			$p(-x)=-4x(x^2-1)(3-x)$ 
		\end{shortsolution}
	\end{subproblem}
\end{multicols}
\end{problem}

%===================================
%   Author: Hughes
%   Date:   May 2011
%===================================
\begin{problem}[Find a formula from a table]\label{poly:prob:findformula}
\Crefrange{poly:tab:findformulap}{poly:tab:findformulas} show values of polynomial functions, $p$, $q$, 
$r$, and $s$.

\begin{table}[!htb]
	\centering
	\begin{widepage}
	\caption{Tables for \cref{poly:prob:findformula}}
	\label{poly:tab:findformula}
	\begin{subtable}{.2\textwidth}
		\centering
		\caption{$y=p(x)$}
		\label{poly:tab:findformulap}
		\begin{tabular}{rr}
			\beforeheading
			\heading{$x$} & \heading{$y$} \\ 
			\afterheading
			$-4$          & $-56$         \\\normalline
			$-3$          & $-18$         \\\normalline
			$-2$          & $0$           \\\normalline
			$-1$          & $4$           \\\normalline
			$0$           & $0$           \\\normalline
			$1$           & $-6$          \\\normalline
			$2$           & $-8$          \\\normalline
			$3$           & $0$           \\\normalline
			$4$           & $24$          \\\lastline
		\end{tabular}
	\end{subtable}
	\hfill
	\begin{subtable}{.2\textwidth}
		\centering
		\caption{$y=q(x)$}
		\label{poly:tab:findformulaq}
		\begin{tabular}{rr}
			\beforeheading
			\heading{$x$} & \heading{$y$} \\ \afterheading
			$-4$          & $-16$         \\\normalline
			$-3$          & $-3$          \\\normalline
			$-2$          & $0$           \\\normalline
			$-1$          & $-1$          \\\normalline
			$0$           & $0$           \\\normalline
			$1$           & $9$           \\\normalline
			$2$           & $32$          \\\normalline
			$3$           & $75$          \\\normalline
			$4$           & $144$         \\\lastline
		\end{tabular}
	\end{subtable}
	\hfill
	\begin{subtable}{.2\textwidth}
		\centering
		\caption{$y=r(x)$}
		\label{poly:tab:findformular}
		\begin{tabular}{rr}
			\beforeheading
			\heading{$x$} & \heading{$y$} \\ \afterheading
			$-4$          & $105$         \\\normalline
			$-3$          & $0$           \\\normalline
			$-2$          & $-15$         \\\normalline
			$-1$          & $0$           \\\normalline
			$0$           & $9$           \\\normalline
			$1$           & $0$           \\\normalline
			$2$           & $-15$         \\\normalline
			$3$           & $0$           \\\normalline
			$4$           & $105$         \\\lastline
		\end{tabular}
	\end{subtable}
	\hfill
	\begin{subtable}{.2\textwidth}
		\centering
		\caption{$y=s(x)$}
		\label{poly:tab:findformulas}
		\begin{tabular}{rr}
			\beforeheading
			\heading{$x$} & \heading{$y$} \\ \afterheading
			$-4$          & $75$          \\\normalline
			$-3$          & $0$           \\\normalline
			$-2$          & $-9$          \\\normalline
			$-1$          & $0$           \\\normalline
			$0$           & $3$           \\\normalline
			$1$           & $0$           \\\normalline
			$2$           & $15$          \\\normalline
			$3$           & $96$          \\\normalline
			$4$           & $760$         \\\lastline
		\end{tabular}
	\end{subtable}
	\end{widepage}
\end{table}

\begin{subproblem}
	Assuming that all of the zeros of $p$ are shown (in \cref{poly:tab:findformulap}), how many zeros does $p$ have? 
	\begin{shortsolution}
		$p$ has 3 zeros.
	\end{shortsolution}
\end{subproblem}
\begin{subproblem}
	What is the degree of $p$?
	\begin{shortsolution}
		$p$ is degree 3.
	\end{shortsolution}
\end{subproblem}
\begin{subproblem}
	Write a formula for $p(x)$.
	\begin{shortsolution}
		$p(x)=x(x+2)(x-3)$
	\end{shortsolution}
\end{subproblem}
\begin{subproblem}
	Assuming that all of the zeros of $q$ are shown (in \cref{poly:tab:findformulaq}), how many zeros does $q$ have?
	\begin{shortsolution}
		$q$ has 2 zeros.
	\end{shortsolution}
\end{subproblem}
\begin{subproblem}
	Describe the difference in behavior of $p$ and $q$ at $-2$.
	\begin{shortsolution}
		$p$ changes sign at $-2$, and $q$ does not change sign at $-2$.
	\end{shortsolution}
\end{subproblem}
\begin{subproblem}
	Given that $q$ is a degree-$3$ polynomial, write a formula for $q(x)$.
	\begin{shortsolution}
		$q(x)=x(x+2)^2$
	\end{shortsolution}
\end{subproblem}
\begin{subproblem}
	Assuming that all of the zeros of $r$ are shown (in \cref{poly:tab:findformular}), find a formula for $r(x)$.
	\begin{shortsolution}
		$r(x)=(x+3)(x+1)(x-1)(x-3)$ 
	\end{shortsolution}
\end{subproblem}
\begin{subproblem}
	Assuming that all of the zeros of $s$ are shown (in \cref{poly:tab:findformulas}), find a formula for $s(x)$.
	\begin{shortsolution}
		$s(x)=(x+3)(x+1)(x-1)^2$ 
	\end{shortsolution}
\end{subproblem}
\end{problem}
\end{exercises}

\section{Rational functions}
\subsection*{Power functions with negative exponents}
The study of rational functions will rely upon a good knowledge 
of power functions with negative exponents. \Cref{rat:ex:oddpow,rat:ex:evenpow} are 
simple but fundamental to understanding the behavior of rational functions.
%===================================
%   Author: Hughes
%   Date:   May 2011
%===================================
\begin{pccexample}[Power functions with odd negative exponents]\label{rat:ex:oddpow}
Graph each of the following functions on your calculator, state their domain in interval notation, and their 
behavior as $x\rightarrow 0^-$ and $x\rightarrow 0^+$.
\[
	f(x)=\frac{1}{x},\qquad g(x)=\dfrac{1}{x^3},\qquad h(x)=\dfrac{1}{x^5}
\]
\begin{pccsolution}
The functions $f$, $g$, and $k$ are plotted in \cref{rat:fig:oddpow}.
The domain of each of the functions $f$, $g$, and $h$ is $(-\infty,0)\cup (0,\infty)$. Note that 
the long-run behavior of each of the functions is the same, and in particular
\begin{align*}
	f(x)\rightarrow 0                         & \text{ as } x\rightarrow\infty  \\
	\mathllap{\text{and }}  f(x)\rightarrow 0 & \text{ as } x\rightarrow-\infty 
\end{align*}
The same results hold for $g$ and $h$. Note also that each of the functions
has a \emph{vertical asymptote} at $0$. We see that
\begin{align*}
	f(x)\rightarrow -\infty                         & \text{ as } x\rightarrow 0^- \\
	\mathllap{\text{and }}   f(x)\rightarrow \infty & \text{ as } x\rightarrow 0^+ 
\end{align*}
The same results hold for $g$ and $h$.

The curve of a function that has a vertical asymptote is necessarily separated 
into \emph{branches}| each of the functions $f$, $g$, and $h$ have $2$ branches.
\end{pccsolution}
\end{pccexample}

\begin{figure}[!htb]
	\begin{minipage}{.45\textwidth}
		\begin{tikzpicture}
			\begin{axis}[
			   framed,
			   xmin=-3,xmax=3,
			   ymin=-5,ymax=5,
			   xtick={-2,-1,...,2},
			   minor ytick={-3,-1,...,3},
			   grid=both,
			   width=\textwidth,
			   legend pos=north west,
			   ]
			   \addplot expression[domain=-3:-0.2]{1/x};
			   \addplot expression[domain=-3:-0.584]{1/x^3};
			   \addplot expression[domain=-3:-0.724]{1/x^5};
			   \addplot expression[domain=0.2:3]{1/x};
			   \addplot expression[domain=0.584:3]{1/x^3};
			   \addplot expression[domain=0.724:3]{1/x^5};
			   \addplot[soldot]coordinates{(-1,-1)}node[axisnode,anchor=north east]{$(-1,-1)$};
			   \addplot[soldot]coordinates{(1,1)}node[axisnode,anchor=south west]{$(1,1)$};
			   \legend{$f$,$g$,$h$}
			\end{axis}
		\end{tikzpicture}
		\caption{}
		\label{rat:fig:oddpow}
	\end{minipage}%
	\hfill
	\begin{minipage}{.45\textwidth}
		\begin{tikzpicture}
			\begin{axis}[
			   framed,
			   xmin=-3,xmax=3,
			   ymin=-5,ymax=5,
			   xtick={-2,-1,...,2},
			   minor ytick={-3,-1,...,3},
			   grid=both,
			   width=\textwidth,
			   legend pos=south east,
			   ]
			   \addplot expression[domain=-3:-0.447]{1/x^2};
			   \addplot expression[domain=-3:-0.668]{1/x^4};
			   \addplot expression[domain=-3:-0.764]{1/x^6};
			   \addplot expression[domain=0.447:3]{1/x^2};
			   \addplot expression[domain=0.668:3]{1/x^4};
			   \addplot expression[domain=0.764:3]{1/x^6};
			   \addplot[soldot]coordinates{(-1,1)}node[axisnode,anchor=south east]{$(-1,1)$};
			   \addplot[soldot]coordinates{(1,1)}node[axisnode,anchor=south west]{$(1,1)$};
			   \legend{$F$,$G$,$H$}
			\end{axis}
		\end{tikzpicture}
		\caption{}
		\label{rat:fig:evenpow}
	\end{minipage}%
\end{figure}


%===================================
%   Author: Hughes
%   Date:   May 2011
%===================================
\begin{pccexample}[Power functions with even negative exponents]\label{rat:ex:evenpow}%
Graph each of the following functions, state their domain, and their 
behavior as $x\rightarrow 0^-$ and $x\rightarrow 0^+$.
\[
	f(x)=\frac{1}{x^2},\qquad g(x)=\frac{1}{x^4},\qquad h(x)=\frac{1}{x^6}
\]
\begin{pccsolution}
The functions $F$, $G$, and $H$ are plotted in \cref{rat:fig:evenpow}.
The domain of each of the functions $F$, $G$, and $H$ is $(-\infty,0)\cup (0,\infty)$. Note that 
the long-run behavior of each of the functions is the same, and in particular
\begin{align*}
	F(x)\rightarrow 0                           & \text{ as } x\rightarrow\infty  \\
	\mathllap{\text{and }}    f(x)\rightarrow 0 & \text{ as } x\rightarrow-\infty 
\end{align*}
As in \cref{rat:ex:oddpow}, $F$ has a horizontal asymptote  that 
has equation $y=0$.
The same results hold for $G$ and $H$. Note also that each of the functions
has a \emph{vertical asymptote} at $0$. We see that
\begin{align*}
	F(x)\rightarrow \infty                          & \text{ as } x\rightarrow 0^- \\
	\mathllap{\text{and }}   F(x)\rightarrow \infty & \text{ as } x\rightarrow 0^+ 
\end{align*}
The same results hold for $G$ and $H$. Each of the functions $F$, $G$, and $H$ 
have $2$ branches.
\end{pccsolution}
\end{pccexample}
%===================================
%   Author: Hughes
%   Date:   March 2012
%===================================
\begin{doyouunderstand}
	\begin{problem}
	Repeat \cref{rat:ex:oddpow,rat:ex:evenpow} using (respectively)
	\begin{subproblem}
		$k(x)=-\dfrac{1}{x}$, $ m(x)=-\dfrac{1}{x^3}$, $ n(x)=-\dfrac{1}{x^5}$
		\begin{shortsolution}
			The functions $k$, $m$, and $n$ have domain $(-\infty,0)\cup (0,\infty)$, and 
			are graphed below.
			
			\begin{tikzpicture}
				\begin{axis}[
				   framed,
				   xmin=-3,xmax=3,
				   ymin=-5,ymax=5,
				   xtick={-2,-1,...,2},
				   minor ytick={-3,-1,...,3},
				   grid=both,
				   width=\solutionfigurewidth,
				   legend pos=north east,
				   ]
				   \addplot expression[domain=-3:-0.2]{-1/x};
				   \addplot expression[domain=-3:-0.584]{-1/x^3};
				   \addplot expression[domain=-3:-0.724]{-1/x^5};
				   \addplot expression[domain=0.2:3]{-1/x};
				   \addplot expression[domain=0.584:3]{-1/x^3};
				   \addplot expression[domain=0.724:3]{-1/x^5};
				   \legend{$k$,$m$,$n$}
				\end{axis}
			\end{tikzpicture}
			
			Note that
			\begin{align*}
				k(x)\rightarrow 0                                & \text{ as } x\rightarrow\infty  \\
				\mathllap{\text{and }}    k(x)\rightarrow 0      & \text{ as } x\rightarrow-\infty \\
				\intertext{and also}
				k(x)\rightarrow \infty                           & \text{ as } x\rightarrow 0^-    \\
				\mathllap{\text{and }}   k(x)\rightarrow -\infty & \text{ as } x\rightarrow 0^+    
			\end{align*}
			The same are true for $m$ and $n$.
		\end{shortsolution}
	\end{subproblem}
	\begin{subproblem}
		$ K(x)=-\dfrac{1}{x^2}$, $ M(x)=-\dfrac{1}{x^4}$, $ N(x)=-\dfrac{1}{x^6}$
		\begin{shortsolution}
			The functions $K$, $M$, and $N$ have domain $(-\infty,0)\cup (0,\infty)$, and 
			are graphed below.
			
			\begin{tikzpicture}
				\begin{axis}[
				   framed,
				   xmin=-3,xmax=3,
				   ymin=-5,ymax=5,
				   xtick={-2,-1,...,2},
				   minor ytick={-3,-1,...,3},
				   grid=both,
				   width=\solutionfigurewidth,
				   legend pos=north east,
				   ]
				   \addplot expression[domain=-3:-0.447]{-1/x^2};
				   \addplot expression[domain=-3:-0.668]{-1/x^4};
				   \addplot expression[domain=-3:-0.764]{-1/x^6};
				   \addplot expression[domain=0.447:3]{-1/x^2};
				   \addplot expression[domain=0.668:3]{-1/x^4};
				   \addplot expression[domain=0.764:3]{-1/x^6};
				   \legend{$K$,$M$,$N$}
				\end{axis}
			\end{tikzpicture}
			
			Note that
			\begin{align*}
				K(x)\rightarrow 0                                & \text{ as } x\rightarrow\infty  \\
				\mathllap{\text{and }}    K(x)\rightarrow 0      & \text{ as } x\rightarrow-\infty \\
				\intertext{and also}
				K(x)\rightarrow -\infty                          & \text{ as } x\rightarrow 0^-    \\
				\mathllap{\text{and }}   K(x)\rightarrow -\infty & \text{ as } x\rightarrow 0^+    
			\end{align*}
			The same are true for $M$ and $N$.
		\end{shortsolution}
	\end{subproblem}
	\end{problem}
\end{doyouunderstand}

\subsection*{Rational functions}
\begin{pccdefinition}[Rational functions]\label{rat:def:function}
Rational functions have the form
\[
	r(x) = \frac{p(x)}{q(x)}
\]
where both $p$ and $q$ are polynomials. 

Note that
\begin{itemize}
	\item the domain or $r$ will be all real numbers, except those that
	make the \emph{denominator}, $q(x)$, equal to $0$;
	\item the zeros of $r$ are the zeros of $p$, i.e the real numbers
	that make the \emph{numerator}, $p(x)$, equal to $0$.
\end{itemize}

\Cref{rat:ex:oddpow,rat:ex:evenpow} are particularly important because $r$ 
will behave like $\frac{1}{x}$, or $\frac{1}{x^2}$ around its vertical asymptotes, 
depending on the power that the relevant term is raised to| we will demonstrate 
this in what follows.
\end{pccdefinition}

%===================================
%   Author: Hughes
%   Date:   May 2011
%===================================
\begin{pccexample}[Rational or not]
Identify whether each of the following functions is a rational or not. If 
the function is rational, state the domain.
\begin{multicols}{3}
	\begin{enumerate}
		\item $r(x)=\dfrac{1}{x}$
		\item $f(x)=2^x+3$
		\item $g(x)=19$
		\item $h(x)=\dfrac{3+x}{4-x}$
		\item $k(x)=\dfrac{x^3+2x}{x-15}$
		\item $l(x)=9-4x$
		\item $m(x)=\dfrac{x+5}{(x-7)(x+9)}$
		\item $n(x)=x^2+6x+7$
		\item $q(x)=1-\dfrac{3}{x+1}$
	\end{enumerate}
\end{multicols}
\begin{pccsolution}
\begin{enumerate}
	\item $r$ is rational; the domain of $r$ is $(-\infty,0)\cup(0,\infty)$.
	\item $f$ is not rational.
	\item $g$ is not rational; $g$ is constant.
	\item $h$ is rational; the domain of $h$ is $(-\infty,4)\cup(4,\infty)$.
	\item $k$ is rational; the domain of $k$ is $(-\infty,15)\cup(15,\infty)$.
	\item $l$ is not rational; $l$ is linear.
	\item $m$ is rational; the domain of $m$ is $(-\infty,-9)\cup(-9,7)\cup(7,\infty)$.
	\item $n$ is not rational; $n$ is quadratic (or you might describe $n$ as a polynomial).
	\item $q$ is rational; the domain of $q$ is $(-\infty,-1)\cup (-1,\infty)$.
\end{enumerate}
\end{pccsolution}
\end{pccexample}

%===================================
%   Author: Hughes
%   Date:   May 2011
%===================================
\begin{pccexample}[Match formula to graph]
Each of the following functions is graphed in \cref{rat:fig:whichiswhich}. 
Which is which?
\[
	r(x)=\frac{1}{x-3}, \qquad q(x)=\frac{x-2}{x+5}, \qquad k(x)=\frac{1}{(x+2)(x-3)}
\]
\begin{figure}[!htb]
	\setlength{\figurewidth}{0.3\textwidth}
	\begin{subfigure}{\figurewidth}
		\begin{tikzpicture}[/pgf/declare function={f=(x-2)/(x+5);}]
			\begin{axis}[
			   framed,
			   xmin=-10,xmax=10,
			   ymin=-6,ymax=6,
			   xtick={-8,-6,...,8},
			   minor ytick={-4,-3,...,4},
			   grid=both,
			   width=\textwidth,
			   ]
			   \addplot[pccplot] expression[domain=-10:-6.37]{f};
			   \addplot[pccplot] expression[domain=-3.97:10]{f};
			   \addplot[soldot] coordinates{(2,0)};
			   \addplot[asymptote,domain=-6:6]({-5},{x});
			\end{axis}
		\end{tikzpicture}
		\caption{}
		\label{rat:fig:which1}
	\end{subfigure}
	\hfill
	\begin{subfigure}{\figurewidth}
		\begin{tikzpicture}[/pgf/declare function={f=1/(x-3);}]
			\begin{axis}[
			   framed,
			   xmin=-10,xmax=10,
			   ymin=-5,ymax=6,
			   xtick={-8,-6,...,8},
			   ytick={-4,4},
			   minor ytick={-3,...,5},
			   grid=both,
			   width=\textwidth,
			   ]
			   \addplot[pccplot] expression[domain=-10:2.8]{f};
			   \addplot[pccplot] expression[domain=3.17:10]{f};
			   \addplot[asymptote,domain=-6:6]({3},{x});
			\end{axis}
		\end{tikzpicture}
		\caption{}
		\label{rat:fig:which2}
	\end{subfigure}
	\hfill
	\begin{subfigure}{\figurewidth}
		\begin{tikzpicture}[/pgf/declare function={f=1/((x-3)*(x+2));}]
			\begin{axis}[
			   framed,
			   xmin=-10,xmax=10,
			   ymin=-5,ymax=5,
			   xtick={-8,-6,...,8},
			   ytick={-4,4},
			   minor ytick={-3,...,3},
			   grid=both,
			   width=\textwidth,
			   ]
			   \addplot[pccplot] expression[domain=-10:-2.03969]{f};
			   \addplot[pccplot] expression[domain=-1.95967:2.95967]{f};
			   \addplot[pccplot] expression[domain=3.03969:10]{f};
			   \addplot[asymptote,domain=-5:5]({-2},{x});
			   \addplot[asymptote,domain=-5:5]({3},{x});
			\end{axis}
		\end{tikzpicture}
		\caption{}
		\label{rat:fig:which3}
	\end{subfigure}
	\caption{}
	\label{rat:fig:whichiswhich}
\end{figure}

\begin{pccsolution}
Let's start with the function $r$. Note that domain of $r$ is $(-\infty,3)\cup(0,3)$, so 
we search for a function that has a vertical asymptote at $3$. There 
are two possible choices: the functions graphed in \cref{rat:fig:which2,rat:fig:which3}, 
but note that the function in \cref{rat:fig:which3} also has a vertical asymptote at $-2$ 
which is not consistent with the formula for $r(x)$. Therefore, $y=r(x)$
is graphed in \cref{rat:fig:which2}. 

The function $q$ has domain $(-\infty,-5)\cup(-5,\infty)$, so we search 
for a function that has a vertical asymptote at $-5$. The only candidate 
is the curve shown in \cref{rat:fig:which1}; note that the curve also goes through $(2,0)$, 
which is consistent with the formula for $q(x)$, since $q(2)=0$, i.e $q$
has a zero at $2$.

The function $k$ has domain $(-\infty,-2)\cup(-2,3)\cup(3,\infty)$, and 
has vertical asymptotes at $-2$ and $3$. This is consistent with 
the graph in \cref{rat:fig:which3} (and is the only curve that 
has $3$ branches).

We note that each function behaves like $\frac{1}{x}$ around its vertical asymptotes, 
because each linear factor in each denominator is raised to the power $1$; if (for example) 
the definition of $r$ was instead
\[
	r(x)=\frac{1}{(x-3)^2}
\]
then we would see that $r$ behaves like $\frac{1}{x^2}$ around its vertical asymptote, and 
the graph of $r$ would be very different. We will deal with these cases in the examples that follow.
\end{pccsolution}
\end{pccexample}

%===================================
%   Author: Hughes
%   Date:   May 2011
%===================================
\begin{pccexample}[Repeated factors in the denominator]
Consider the functions $f$, $g$, and $h$ that have formulas
\[
	f(x)=\frac{x-2}{(x-3)(x+2)}, \qquad g(x)=\frac{x-2}{(x-3)^2(x+2)}, \qquad h(x)=\frac{x-2}{(x-3)(x+2)^2}
\]
which are graphed in \cref{rat:fig:repfactd}. Note that each function has $2$ 
vertical asymptotes, and the domain of each function is 
\[
	(-\infty,-2)\cup(-2,3)\cup(3,\infty)
\]
so we are not surprised to see that each curve has $3$ branches. We also note that 
the numerator of each function is the same, which tells us that each function has 
only $1$ zero at $2$.

The functions $g$ and $h$ are different from those that we have considered previously, 
because they have a repeated factor in the denominator. Notice in particular 
the way that the functions behave around their asymptotes:
\begin{itemize}
	\item $f$ behaves like $\frac{1}{x}$ around both of its asymptotes;
	\item $g$ behaves like $\frac{1}{x}$ around $-2$, and like $\frac{1}{x^2}$ around $3$;
	\item $h$ behaves like $\frac{1}{x^2}$ around $-2$, and like $\frac{1}{x}$ around $3$.
\end{itemize}
\end{pccexample}
\begin{figure}[!htb]
	\setlength{\figurewidth}{0.3\textwidth}
	\begin{subfigure}{\figurewidth}
		\begin{tikzpicture}[/pgf/declare function={f=(x-2)/((x+2)*(x-3));}]
			\begin{axis}[
			   %                    framed,
			   xmin=-5,xmax=5,
			   ymin=-4,ymax=4,
			   xtick={-4,-2,...,4},
			   ytick={-2,2},
			   %                    grid=both,
			   width=\textwidth,
			   ]
			   \addplot[pccplot] expression[domain=-5:-2.201]{f};
			   \addplot[pccplot] expression[domain=-1.802:2.951]{f};
			   \addplot[pccplot] expression[domain=3.052:5]{f};
			   \addplot[soldot] coordinates{(2,0)};
			   %                 \addplot[asymptote,domain=-6:6]({-2},{x});
			   %                 \addplot[asymptote,domain=-6:6]({3},{x});
			\end{axis}
		\end{tikzpicture}
		\caption{$y=\dfrac{x-2}{(x+2)(x-3)}$}
		\label{rat:fig:repfactd1}
	\end{subfigure}
	\hfill
	\begin{subfigure}{\figurewidth}
		\begin{tikzpicture}[/pgf/declare function={f=(x-2)/((x+2)*(x-3)^2);}]
			\begin{axis}[
			   %                    framed,
			   xmin=-5,xmax=5,
			   ymin=-4,ymax=4,
			   xtick={-4,-2,...,4},
			   ytick={-2,2},
			   %                    grid=both,
			   width=\textwidth,
			   ]
			   \addplot[pccplot] expression[domain=-5:-2.039]{f};
			   \addplot[pccplot] expression[domain=-1.959:2.796]{f};
			   \addplot[pccplot] expression[domain=3.243:5]{f};
			   \addplot[soldot] coordinates{(2,0)};
			   %                 \addplot[asymptote,domain=-4:4]({-2},{x});
			   %                 \addplot[asymptote,domain=-4:4]({3},{x});
			\end{axis}
		\end{tikzpicture}
		\caption{$y=\dfrac{x-2}{(x+2)(x-3)^2}$}
		\label{rat:fig:repfactd2}
	\end{subfigure}
	\hfill
	\begin{subfigure}{\figurewidth}
		\begin{tikzpicture}[/pgf/declare function={f=(x-2)/((x+2)^2*(x-3));}]
			\begin{axis}[
			   %                    framed,
			   xmin=-5,xmax=5,
			   ymin=-4,ymax=4,
			   xtick={-4,-2,...,2},
			   ytick={-2,2},
			   %                    grid=both,
			   width=\textwidth,
			   ]
			   \addplot[pccplot] expression[domain=-5:-2.451]{f};
			   \addplot[pccplot] expression[domain=-1.558:2.990]{f};
			   \addplot[pccplot] expression[domain=3.010:6]{f};
			   \addplot[soldot] coordinates{(2,0)};
			   %                 \addplot[asymptote,domain=-4:4]({-2},{x});
			   %                 \addplot[asymptote,domain=-4:4]({3},{x});
			\end{axis}
		\end{tikzpicture}
		\caption{$y=\dfrac{x-2}{(x+2)^2(x-3)}$}
		\label{rat:fig:repfactd3}
	\end{subfigure}
	\caption{}
	\label{rat:fig:repfactd}
\end{figure}

\Cref{rat:def:function} says that the zeros of 
the rational function $r$ that has formula $r(x)=\frac{p(x)}{q(x)}$ are 
the zeros of $p$. Let's explore this a little more.
%===================================
%   Author: Hughes
%   Date:   May 2012
%===================================
\begin{pccexample}[Zeros] Find the zeros of each of the following functions
\[
	\alpha(x)=\frac{x+5}{3x-7}, \qquad \beta(x)=\frac{9-x}{x+1}, \qquad \gamma(x)=\frac{17x^2-10}{2x+1}
\]
\begin{pccsolution}
We find the zeros of each function in turn by setting the numerator equal to $0$. The zeros of 
$\alpha$ are found by solving 
\[
	x+5=0
\]
The zero of $\alpha$ is $-5$.

Similarly, we may solve $9-x=0$ to find the zero of $\beta$, which is clearly $9$.

The zeros of $\gamma$ satisfy the equation
\[
	17x^2-10=0
\]
which we can solve using the square root property to obtain
\[
	x=\pm\frac{10}{17}
\]
The zeros of $\gamma$ are $\pm\frac{10}{17}$.
\end{pccsolution}
\end{pccexample}

\subsection*{Long-run behavior}
Our focus so far has been on the behavior of rational functions around 
their \emph{vertical} asymptotes. In fact, rational functions also 
have interesting long-run behavior around their \emph{horizontal} or 
\emph{oblique} asymptotes. A rational function will always have either 
a horizontal or an oblique asymptote| the case is determined by the degree
of the numerator and the degree of the denominator.
\begin{pccdefinition}[Long-run behavior]\label{rat:def:longrun}
Let $r$ be the rational function that has formula
\[
	r(x) = \frac{a_n x^n + a_{n-1}x^{n-1}+\ldots + a_0}{b_m x^m + b_{m-1}x^{m-1}+\ldots+b_0}
\]
We can classify the long-run behavior of the rational function $r$ 
according to the following criteria:
\begin{itemize}
	\item if $n<m$ then  $r$ has a horizontal asymptote with equation $y=0$;
	\item if $n=m$ then $r$ has a horizontal asymptote with equation $y=\dfrac{a_n}{b_m}$;
	\item if $n>m$ then $r$ will have an oblique asymptote as $x\rightarrow\pm\infty$ (more on this in \cref{rat:sec:oblique})
\end{itemize} 
\end{pccdefinition}
We will concentrate on functions that have horizontal asymptotes until 
we reach \cref{rat:sec:oblique}.

%===================================
%   Author: Hughes
%   Date:   May 2012
%===================================
\begin{pccexample}[Long-run behavior graphically]\label{rat:ex:horizasymp}
\pccname{Kebede} has graphed the following functions in his graphing calculator
\[
	r(x)=\frac{x+1}{x-3}, \qquad s(x)=\frac{2(x+1)}{x-3}, \qquad t(x)=\frac{3(x+1)}{x-3}
\]
and obtained the curves shown in \cref{rat:fig:horizasymp}. Kebede decides 
to test his knowledgeable friend \pccname{Oscar}, and asks him 
to match the formulas to the graphs.

\begin{figure}[!htb]
	\setlength{\figurewidth}{0.3\textwidth}
	\begin{subfigure}{\figurewidth}
		\begin{tikzpicture}[/pgf/declare function={f=2*(x+1)/(x-3);}]
			\begin{axis}[
			   framed,
			   xmin=-15,xmax=15,
			   ymin=-6,ymax=6,
			   xtick={-12,-8,...,12},
			   minor ytick={-4,-3,...,4},
			   grid=both,
			   width=\textwidth,
			   ]
			   \addplot[pccplot] expression[domain=-15:2]{f};
			   \addplot[pccplot] expression[domain=5:15]{f};
			   \addplot[soldot] coordinates{(-1,0)};
			   \addplot[asymptote,domain=-6:6]({3},{x});
			   \addplot[asymptote,domain=-15:15]({x},{2});
			\end{axis}
		\end{tikzpicture}
		\caption{}
		\label{rat:fig:horizasymp1}
	\end{subfigure}
	\hfill
	\begin{subfigure}{\figurewidth}
		\begin{tikzpicture}[/pgf/declare function={f=(x+1)/(x-3);}]
			\begin{axis}[
			   framed,
			   xmin=-15,xmax=15,
			   ymin=-6,ymax=6,
			   xtick={-12,-8,...,12},
			   minor ytick={-4,-3,...,4},
			   grid=both,
			   width=\textwidth,
			   ]
			   \addplot[pccplot] expression[domain=-15:2.42857,samples=50]{f};
			   \addplot[pccplot] expression[domain=3.8:15,samples=50]{f};
			   \addplot[soldot] coordinates{(-1,0)};
			   \addplot[asymptote,domain=-6:6]({3},{x});
			   \addplot[asymptote,domain=-15:15]({x},{1});
			\end{axis}
		\end{tikzpicture}
		\caption{}
		\label{rat:fig:horizasymp2}
	\end{subfigure}
	\hfill
	\begin{subfigure}{\figurewidth}
		\begin{tikzpicture}[/pgf/declare function={f=3*(x+1)/(x-3);}]
			\begin{axis}[
			   framed,
			   xmin=-15,xmax=15,
			   ymin=-6,ymax=6,
			   xtick={-12,-8,...,12},
			   minor ytick={-4,-3,...,4},
			   grid=both,
			   width=\textwidth,
			   ]
			   \addplot[pccplot] expression[domain=-15:1.6666,samples=50]{f};
			   \addplot[pccplot] expression[domain=7:15]{f};
			   \addplot[soldot] coordinates{(-1,0)};
			   \addplot[asymptote,domain=-6:6]({3},{x});
			   \addplot[asymptote,domain=-15:15]({x},{3});
			\end{axis}
		\end{tikzpicture}
		\caption{}
		\label{rat:fig:horizasymp3}
	\end{subfigure}
	\caption{Horizontal asymptotes}
	\label{rat:fig:horizasymp}
\end{figure}

Oscar notices that each function has a vertical asymptote at $3$ and a zero at $-1$. 
The main thing that catches Oscar's eye is that each function has a different 
coefficient in the numerator, and that each curve has a different horizontal asymptote. 
In particular, Oscar notes that
\begin{itemize}
	\item the curve shown in \cref{rat:fig:horizasymp1} has a horizontal asymptote with equation $y=2$;
	\item the curve shown in \cref{rat:fig:horizasymp2} has a horizontal asymptote with equation $y=1$;
	\item the curve shown in \cref{rat:fig:horizasymp3} has a horizontal asymptote with equation $y=3$.
\end{itemize}
Oscar is able to tie it all together for Kebede by referencing \cref{rat:def:longrun}. He says 
that since the degree of the numerator and the degree of the denominator is the same 
for each of the functions $r$, $s$, and $t$, the horizontal asymptote will be determined 
by evaluating the ratio of their leading coefficients. 

Oscar therefore says that $r$ should have a horizontal asymptote $y=\frac{1}{1}=1$, $s$ should 
have a horizontal asymptote $y=\frac{2}{1}=2$, and $t$ should have a horizontal asymptote 
$y=\frac{3}{1}=3$. Kebede is able to finish the problem from here, and says that $r$ is 
shown in \cref{rat:fig:horizasymp2}, $s$ is shown in \cref{rat:fig:horizasymp1}, and 
$t$ is shown in \cref{rat:fig:horizasymp3}.
\end{pccexample}

%===================================
%   Author: Hughes
%   Date:   May 2012
%===================================
\begin{pccexample}[Long-run behavior numerically]
\pccname{Xiao} and \pccname{Dwayne} saw \cref{rat:ex:horizasymp} but are a little confused 
about horizontal asymptotes. What does it mean to say that a function $r$ has a horizontal
asymptote? 

They decide to explore the concept by 
constructing a table of values for the rational functions $R$ and  $S$ that have formulas
\[
	R(x)=\frac{-5(x+1)}{x-3}, \qquad S(x)=\frac{7(x-5)}{2(x+1)}
\]
In \cref{rat:tab:plusinfty} they model the behavior of $R$ and $S$ as $x\rightarrow\infty$, 
and in \cref{rat:tab:minusinfty} they model the behavior of $R$ and $S$ as $x\rightarrow-\infty$
by substituting very large values of $|x|$ into each function.
\begin{table}[!htb]
	\begin{minipage}{.5\textwidth}
		\centering
		\caption{$R$ and $S$ as $x\rightarrow\infty$}
		\label{rat:tab:plusinfty}
		\begin{tabular}{crr}
			\beforeheading
			$x$            & $R(x)$     & $S(x)$    \\ \afterheading
			$1\times 10^2$ & $-5.20619$ & $3.29208$ \\\normalline
			$1\times 10^3$ & $-5.02006$ & $3.47902$ \\\normalline
			$1\times 10^4$ & $-5.00200$ & $3.49790$ \\\normalline
			$1\times 10^5$ & $-5.00020$ & $3.49979$ \\\normalline
			$1\times 10^6$ & $-5.00002$ & $3.49998$ \\\lastline
		\end{tabular}
	\end{minipage}%
	\begin{minipage}{.5\textwidth}
		\centering
		\caption{$R$ and $S$ as $x\rightarrow-\infty$}
		\label{rat:tab:minusinfty}
		\begin{tabular}{crr}
			\beforeheading
			$x$             & $R(x)$     & $S(x)$    \\ \afterheading
			$-1\times 10^2$ & $-4.80583$ & $3.71212$ \\\normalline
			$-1\times 10^3$ & $-4.98006$ & $3.52102$ \\\normalline
			$-1\times 10^4$ & $-4.99800$ & $3.50210$ \\\normalline
			$-1\times 10^5$ & $-4.99980$ & $3.50021$ \\\normalline
			$-1\times 10^6$ & $-4.99998$ & $3.50002$ \\\lastline
		\end{tabular}
	\end{minipage}
\end{table}

Xiao and Dwayne study \cref{rat:tab:plusinfty,rat:tab:minusinfty} and decide that 
the functions $R$ and $S$ never actually touch their horizontal asymptotes, but they 
do get infinitely close. They also feel as if they have a better understanding of 
what it means to study the behavior of a function as $x\rightarrow\pm\infty$.
\end{pccexample}

%===================================
%   Author: Hughes
%   Date:   May 2011
%===================================
\begin{pccexample}[Repeated factors in the numerator]
Consider the functions $f$, $g$, and $h$ that have formulas
\[
	f(x)=\frac{(x-2)^2}{(x-3)(x+1)}, \qquad g(x)=\frac{x-2}{(x-3)(x+1)}, \qquad h(x)=\frac{(x-2)^3}{(x-3)(x+1)}
\]
which are graphed in \cref{rat:fig:repfactn}. We note that each function has vertical
asymptotes at $-1$ and $3$, and so the domain of each function is
\[
	(-\infty,-1)\cup(-1,3)\cup(3,\infty)
\]
We also notice that the numerators of each function are quite similar| indeed, each 
function has a zero at $2$, but how does each function behave around their zero?

Using \cref{rat:fig:repfactn} to guide us, we note that
\begin{itemize}
	\item $f$ has a horizontal intercept $(2,0)$, but the curve of 
	$f$ does not cut the horizontal axis| it bounces off it;
	\item $g$ also has a horizontal intercept $(2,0)$, and the curve 
	of $g$ \emph{does} cut the horizontal axis;
	\item $h$ has a horizontal intercept $(2,0)$, and the curve of $h$ 
	also cuts the axis, but appears flattened as it does so.
\end{itemize}

We can further enrich our study by discussing the long-run behavior of each function. 
Using the tools of \cref{rat:def:longrun}, we can deduce that 
\begin{itemize}
	\item $f$ has a horizontal asymptote with equation $y=1$;
	\item $g$ has a horizontal asymptote with equation $y=0$;
	\item $h$ does \emph{not} have a horizontal asymptote| it has an oblique asymptote (we'll 
	study this more in \cref{rat:sec:oblique}).
\end{itemize}
\end{pccexample}

\begin{figure}[!htb]
	\setlength{\figurewidth}{0.3\textwidth}
	\begin{subfigure}{\figurewidth}
		\begin{tikzpicture}[/pgf/declare function={f=(x-2)^2/((x+1)*(x-3));}]
			\begin{axis}[
			   %                    framed,
			   xmin=-5,xmax=5,
			   ymin=-10,ymax=10,
			   xtick={-4,-2,...,4},
			   ytick={-8,-4,...,8},
			   %                    grid=both,
			   width=\figurewidth,
			   ]
			   \addplot[pccplot] expression[domain=-5:-1.248,samples=50]{f};
			   \addplot[pccplot] expression[domain=-0.794:2.976,samples=50]{f};
			   \addplot[pccplot] expression[domain=3.026:5,samples=50]{f};
			   \addplot[soldot] coordinates{(2,0)};
			   %                 \addplot[asymptote,domain=-6:6]({-1},{x});
			   %                 \addplot[asymptote,domain=-6:6]({3},{x});
			\end{axis}
		\end{tikzpicture}
		\caption{$y=\dfrac{(x-2)^2}{(x+1)(x-3)}$}
		\label{rat:fig:repfactn1}
	\end{subfigure}
	\hfill
	\begin{subfigure}{\figurewidth}
		\begin{tikzpicture}[/pgf/declare function={f=(x-2)/((x+1)*(x-3));}]
			\begin{axis}[
			   %                    framed,
			   xmin=-5,xmax=5,
			   ymin=-10,ymax=10,
			   xtick={-4,-2,...,4},
			   ytick={-8,-4,...,8},
			   %                    grid=both,
			   width=\figurewidth,
			   ]
			   \addplot[pccplot] expression[domain=-5:-1.075]{f};
			   \addplot[pccplot] expression[domain=-0.925:2.975]{f};
			   \addplot[pccplot] expression[domain=3.025:5]{f};
			   \addplot[soldot] coordinates{(2,0)};
			   %                 \addplot[asymptote,domain=-6:6]({-1},{x});
			   %                 \addplot[asymptote,domain=-6:6]({3},{x});
			\end{axis}
		\end{tikzpicture}
		\caption{$y=\dfrac{x-2}{(x+1)(x-3)}$}
		\label{rat:fig:repfactn2}
	\end{subfigure}
	\hfill
	\begin{subfigure}{\figurewidth}
		\begin{tikzpicture}[/pgf/declare function={f=(x-2)^3/((x+1)*(x-3));}]
			\begin{axis}[
			   %                    framed,
			   xmin=-5,xmax=5,
			   xtick={-8,-6,...,8},
			   %                    grid=both,
			   ymin=-30,ymax=30,
			   width=\figurewidth,
			   ]
			   \addplot[pccplot] expression[domain=-5:-1.27]{f};
			   \addplot[pccplot] expression[domain=-0.806:2.99185]{f};
			   \addplot[pccplot] expression[domain=3.0085:5]{f};
			   \addplot[soldot] coordinates{(2,0)};
			   %                 \addplot[asymptote,domain=-30:30]({-1},{x});
			   %                 \addplot[asymptote,domain=-30:30]({3},{x});
			\end{axis}
		\end{tikzpicture}
		\caption{$y=\dfrac{(x-2)^3}{(x+1)(x-3)}$}
		\label{rat:fig:repfactn3}
	\end{subfigure}
	\caption{}
	\label{rat:fig:repfactn}
\end{figure}

\subsection*{Holes}
Rational functions have a vertical asymptote at $a$ if the denominator is $0$ at $a$. 
What happens if the numerator is $0$ at the same place? In this case, we say that the rational 
function has a \emph{hole} at $a$.
\begin{pccdefinition}[Holes]
The rational function
\[
	r(x)=\frac{p(x)}{q(x)}
\]
has a hole at $a$ if $p(a)=q(a)=0$. Note that holes are different from 
a vertical asymptotes. We represent that $r$ has a hole at the point 
$(a,r(a))$ on the curve $y=r(x)$ by 
using a hollow circle, $\circ$.
\end{pccdefinition}

%===================================
%   Author: Hughes
%   Date:   March 2012
%===================================
\begin{pccexample}
\pccname{Mohammed} and \pccname{Sue} have graphed the function $r$ that has formula
\[
	r(x)=\frac{x^2+x-6}{(x-2)}
\]
in their calculators, and can not decide if the correct graph 
is \cref{rat:fig:hole} or \cref{rat:fig:hole1}.

Luckily for them, Oscar is nearby, and can help them settle the debate. 
Oscar demonstrates that
\begin{align*}
	r(x) & =\frac{(x+3)(x-2)}{(x-2)} \\
	     & = x+3                     
\end{align*}
but only when $x\ne 2$, because the function is undefined at $2$. Oscar 
says that this necessarily means that the domain or $r$ is 
\[
	(-\infty,2)\cup(2,\infty)
\]
and that $r$ must have a hole at $2$. 

Mohammed and Sue are very grateful for the clarification, and conclude that 
the graph of $r$ is shown in \cref{rat:fig:hole1}.
\begin{figure}[!htb]
	\begin{minipage}{.45\textwidth}
		\begin{tikzpicture}
			\begin{axis}[
			   framed,
			   xmin=-10,xmax=10,
			   ymin=-10,ymax=10,
			   xtick={-8,-4,...,8},
			   ytick={-8,-4,...,8},
			   grid=both,
			   width=\textwidth,
			   ]
			   \addplot expression[domain=-10:7]{x+3};
			   \addplot[soldot] coordinates{(-3,0)};
			\end{axis}
		\end{tikzpicture}
		\caption{}
		\label{rat:fig:hole}
	\end{minipage}%
	\hfill
	\begin{minipage}{.45\textwidth}
		\begin{tikzpicture}
			\begin{axis}[
			   framed,
			   xmin=-10,xmax=10,
			   ymin=-10,ymax=10,
			   xtick={-8,-4,...,8},
			   ytick={-8,-4,...,8},
			   grid=both,
			   width=\textwidth,
			   ]
			   \addplot expression[domain=-10:7]{x+3};
			   \addplot[holdot] coordinates{(2,5)};
			   \addplot[soldot] coordinates{(-3,0)};
			\end{axis}
		\end{tikzpicture}
		\caption{}
		\label{rat:fig:hole1}
	\end{minipage}%
\end{figure}
\end{pccexample}

%===================================
%   Author: Hughes
%   Date:   May 2011
%===================================
\begin{pccexample}
Consider the function $f$ that has formula
\[
	f(x)=\frac{x(x+3)}{x^2-4x}
\]
The domain of $f$ is $(-\infty,0)\cup(0,4)\cup(4,\infty)$ because both $0$ and $4$ 
make the denominator equal to $0$. Notice that
\begin{align*}
	f(x) & = \frac{x(x+3)}{x(x-4)} \\
	     & = \frac{x+3}{x-4}       
\end{align*}
provided that $x\ne 0$. Since $0$ makes the numerator 
and the denominator 0 at the same time, we say that $f$ has a hole at $(0,-\nicefrac{3}{4})$. 
Note that this necessarily means that $f$ does not have a vertical intercept.

We also note $f$ has a vertical asymptote at $4$; the function is graphed in \cref{rat:fig:holeex}.
\begin{figure}[!htb]
	\centering
	\begin{tikzpicture}[/pgf/declare function={f=(x+3)/(x-4);}]
		\begin{axis}[
		   framed,
		   xmin=-10,xmax=10,
		   ymin=-10,ymax=10,
		   xtick={-8,-6,...,8},
		   ytick={-8,-6,...,8},
		   grid=both,
		   ]
		   \addplot[pccplot] expression[domain=-10:3.36364,samples=50]{f};
		   \addplot[pccplot] expression[domain=4.77:10]{f};
		   \addplot[asymptote,domain=-10:10]({4},{x});
		   \addplot[holdot]coordinates{(0,-0.75)};
		   \addplot[soldot] coordinates{(-3,0)};
		\end{axis}
	\end{tikzpicture}
	\caption{$y=\dfrac{x(x+3)}{x^2-4x}$}
	\label{rat:fig:holeex}
\end{figure}
\end{pccexample}



%===================================
%   Author: Hughes
%   Date:   March 2012
%===================================
\begin{pccexample}[Minimums and maximums]
\pccname{Seamus} and \pccname{Trang} are discussing rational functions. Seamus says that 
if a rational function has a vertical asymptote, then it can 
not possibly have local minimums and maximums, nor can it have 
global minimums and maximums.

Trang says this statement is not always true. She plots the functions 
$f$ and $g$ that have formulas
\[
	f(x)=-\frac{32(x-1)(x+1)}{(x-2)^2(x+2)^2}, \qquad g(x)=\frac{32(x-1)(x+1)}{(x-2)^2(x+2)^2}
\]
in \cref{rat:fig:minmax1,rat:fig:minmax2} and shows them to Seamus. On seeing the graphs, 
Seamus quickly corrects himself, and says  that $f$ has a local (and global) 
maximum of $2$ at $0$, and that $g$ has a local (and global) minimum of $-2$ at $0$.

\begin{figure}[!htb]
	\begin{minipage}{.45\textwidth}
		\begin{tikzpicture}[/pgf/declare function={f=-32*(x-1)*(x+1)/(( x-2)^2*(x+2)^2);}]
			\begin{axis}[
			   framed,
			   xmin=-10,xmax=10,
			   ymin=-10,ymax=10,
			   xtick={-8,-6,...,8},
			   ytick={-8,-6,...,8},
			   grid=both,
			   width=\textwidth,
			   ]
			   \addplot[pccplot] expression[domain=-10:-3.01]{f};
			   \addplot[pccplot] expression[domain=-1.45:1.45]{f};
			   \addplot[pccplot] expression[domain=3.01:10]{f};
			   \addplot[soldot] coordinates{(-1,0)(1,0)};
			\end{axis}
		\end{tikzpicture}
		\caption{$y=f(x)$}
		\label{rat:fig:minmax1}
	\end{minipage}%
	\hfill
	\begin{minipage}{.45\textwidth}
		\begin{tikzpicture}[/pgf/declare function={f=32*(x-1)*(x+1)/(( x-2)^2*(x+2)^2);}]
			\begin{axis}[
			   framed,
			   xmin=-10,xmax=10,
			   ymin=-10,ymax=10,
			   xtick={-8,-6,...,8},
			   ytick={-8,-6,...,8},
			   grid=both,
			   width=\textwidth,
			   ]
			   \addplot[pccplot] expression[domain=-10:-3.01]{f};
			   \addplot[pccplot] expression[domain=-1.45:1.45]{f};
			   \addplot[pccplot] expression[domain=3.01:10]{f};
			   \addplot[soldot] coordinates{(-1,0)(1,0)};
			\end{axis}
		\end{tikzpicture}
		\caption{$y=g(x)$}
		\label{rat:fig:minmax2}
	\end{minipage}%
\end{figure}

Seamus also notes that (in its domain) the function $f$ is always concave down, and 
that (in its domain) the function $g$ is always concave up. Furthermore, Trang
observes that each function behaves like $\frac{1}{x^2}$ around each of its vertical 
asymptotes, because each linear factor in the denominator is raised to the power $2$.

\pccname{Oscar} stops by and reminds both students about the long-run behavior; according 
to \cref{rat:def:longrun} since the degree of the denominator is greater than the
degree of the numerator (in both functions), each function has a horizontal asymptote
at $y=0$.
\end{pccexample}


\investigation*{}
%===================================
%   Author: Pettit/Hughes
%   Date:   March 2012
%===================================
\begin{problem}[The spaghetti incident]
The same Queen from \vref{exp:prob:queenschessboard} has recovered from 
the rice experiments, and has called her loyal jester for another challenge.

The jester has an $11-$inch piece of uncooked spaghetti that he puts on a table; 
he uses a book to cover $\unit[1]{inch}$ of it so that 
$\unit[10]{inches}$ hang over the edge. The jester then produces a box of $\unit{mg}$
weights that can be hung from the spaghetti.

The jester says it will take $\unit[y]{mg}$ to break the spaghetti when hung
$\unit[x]{inches}$ from the edge, according to the rule $y=\frac{100}{x}$.
\begin{margintable}
	\centering
	\captionof{table}{}
	\label{rat:tab:spaghetti}
	\begin{tabular}{cc}
		\beforeheading
		\heading{$x$} & \heading{$y$} \\
		\afterheading
		$1$           &               \\\normalline
		$2$           &               \\\normalline
		$3$           &               \\\normalline
		$4$           &               \\\normalline
		$5$           &               \\\normalline
		$6$           &               \\\normalline
		$7$           &               \\\normalline
		$8$           &               \\\normalline
		$9$           &               \\\normalline
		$10$          &               \\\lastline
	\end{tabular}
\end{margintable}
\begin{subproblem}\label{rat:prob:spaggt1}
	Help the Queen complete \cref{rat:tab:spaghetti}, and use $2$ digits after the decimal
	where appropriate.
	\begin{shortsolution}
		\begin{tabular}[t]{ld{2}}
			\beforeheading
			\heading{$x$} & \heading{$y$} \\
			\afterheading
			$1$           & 100           \\\normalline
			$2$           & 50            \\\normalline
			$3$           & 33.33         \\\normalline
			$4$           & 25            \\\normalline
			$5$           & 20            \\\normalline
			$6$           & 16.67         \\\normalline
			$7$           & 14.29         \\\normalline
			$8$           & 12.50         \\\normalline
			$9$           & 11.11         \\\normalline
			$10$          & 10            \\\lastline
		\end{tabular}
	\end{shortsolution}
\end{subproblem}
\begin{subproblem}
	What do you notice about the number of $\unit{mg}$ that it takes to break 
	the spaghetti as $x$ increases?
	\begin{shortsolution}
		It seems that the number of $\unit{mg}$ that it takes to break the spaghetti decreases
		as $x$ increases.
	\end{shortsolution}
\end{subproblem}
\begin{subproblem}\label{rat:prob:spaglt1}
	The Queen wonders what happens when $x$ gets very small| help the Queen construct 
	a table of values for $x$ and $y$ when $x=0.0001, 0.001, 0.01, 0.1, 0.5, 1$.
	\begin{shortsolution}
		\begin{tabular}[t]{d{2}l}
			\beforeheading
			\heading{$x$} & \heading{$y$} \\
			\afterheading
			0.0001        & $1000000$     \\\normalline
			0.001         & $100000$      \\\normalline
			0.01          & $10000$       \\\normalline
			0.1           & $1000$        \\\normalline
			0.5           & $200$         \\\normalline
			1             & $100$         \\\lastline
		\end{tabular}
	\end{shortsolution}
\end{subproblem}
\begin{subproblem}
	What do you notice about the number of $\unit{mg}$ that it takes to break the spaghetti
	as $x\rightarrow 0$? Would it ever make sense to let $x=0$?
	\begin{shortsolution}
		The number of $\unit{mg}$ required to break the spaghetti increases as $x\rightarrow 0$.
		We can not allow $x$ to be $0$, as we can not divide by $0$, and we can not 
		be $0$ inches from the edge of the table.
	\end{shortsolution}
\end{subproblem}
\begin{subproblem}
	Plot your results from \cref{rat:prob:spaggt1,rat:prob:spaglt1} on the same graph, 
	and join the points using a smooth curve| set the maximum value of $y$ as $200$, and 
	note that this necessarily means that you will not be able to plot all of the points.
	\begin{shortsolution}
		The graph of $y=\frac{100}{x}$ is shown below.
		
		\begin{tikzpicture}
			\begin{axis}[
			   framed,
			   xmin=-2,xmax=11,
			   ymin=-20,ymax=200,
			   xtick={2,4,...,10},
			   ytick={20,40,...,180},
			   grid=major,
			   width=\solutionfigurewidth,
			   ]
			   \addplot+[-] expression[domain=0.5:10]{100/x};
			   \addplot[soldot] coordinates{(0.5,200)(1,100)(2,50)(3,33.33)
			   (4,25)(5,20)(16.67)(7,14.29)(8,12.50)(9,11.11)(10,10)};
			\end{axis}
		\end{tikzpicture}
	\end{shortsolution}
\end{subproblem}
\begin{subproblem}
	Using your graph, observe what happens to $y$ as $x$ increases. If we could somehow
	construct a piece of uncooked spaghetti that was $\unit[101]{inches}$ long, how many 
	$\unit{mg}$ would it take to break the spaghetti?
	\begin{shortsolution}
		As $x$ increases, $y\rightarrow 0$. If we could construct a piece of spaghetti 
		$\unit[101]{inches}$ long, it would only take $\unit[1]{mg}$ to break it $\left(\frac{100}{100}=1\right)$. Of course, 
		the weight of spaghetti would probably cause it to break without the weight.
	\end{shortsolution}
\end{subproblem}
The Queen looks forward to more food-related investigations from her jester.
\end{problem}



%===================================
%   Author: Adams (Hughes)
%   Date:   March 2012
%===================================
\begin{problem}[Debt Amortization]
To amortize a debt means to pay it off in a given length of time using 
equal periodic payments. The payments include interest on the unpaid 
balance. The following formula gives the monthly payment, $M$, in dollars
that is necessary to amortize a debt of $P$ dollars in $n$ months 
at a monthly interest rate of $i$
\[
	M=\frac{P\cdot i}{1-(1+i)^{-n}}
\]
Use this formula in each of the following problems.
\begin{subproblem}
	What monthly payments are necessary on a credit card debt of \$2000 at 
	$\unit[1.5]{\%}$ monthly if you want to pay off the debt in $2$ years?
	In one year? How much money will you save by paying off the debt in the
	shorter amount of time?
	\begin{shortsolution}
		Paying off the debt in $2$ years, we use
		\begin{align*}
			M & = \frac{2000\cdot 0.015}{1-(1+0.015)^{-24}} \\
			  & \approx 99.85                               
		\end{align*}
		The monthly payments are \$99.85.
		
		Paying off the debt in $1$ year, we use
		\begin{align*}
			M & = \frac{2000\cdot 0.015}{1-(1+0.015)^{-12}} \\
			  & \approx 183.36                              
		\end{align*}
		The monthly payments are \$183.36
		
		In the $2$-year model we would pay a total of $\$99.85\cdot 12=\$2396.40$. In the
		$1$-year model we would pay a total of $\$183.36\cdot 12=\$2200.32$. We would therefore
		save $\$196.08$ if we went with the $1$-year model instead of the $2$-year model.
	\end{shortsolution}
\end{subproblem}
\begin{subproblem}
	To purchase a home, a family needs a loan of \$300,000 at $\unit[5.2]{\%}$ 
	annual interest.  Compare a $20$ year loan to a $30$ year loan and make 
	a recommendation for the family.
	(Note: when given an annual interest rate, it is a common business practice to divide by
	$12$ to get a monthly rate.)
	\begin{shortsolution}
		For the $20$-year loan we use
		\begin{align*}
			M & = \frac{300000\cdot \frac{0.052}{12}}{1-\left( 1+\frac{0.052}{12} \right)^{-12\cdot 20}} \\
			  & \approx 2013.16                                                                          
		\end{align*}
		The monthly payments are \$2013.16.
		
		For the $30$-year loan we use
		\begin{align*}
			M & = \frac{300000\cdot \frac{0.052}{12}}{1-\left( 1+\frac{0.052}{12} \right)^{-12\cdot 30}} \\
			  & \approx 1647.33                                                                          
		\end{align*}
		The monthly payments are \$1647.33.
		
		The total amount paid during the $20$-year loan is $\$2013.16\cdot 12\cdot 20=\$483,158.40$. 
		The total amount paid during the $30$-year loan is $\$1647.33\cdot 12\cdot 30=\$593,038.80$.
		
		Recommendation: if you can afford the payments, choose the $20$-year loan.
	\end{shortsolution}
\end{subproblem}
\begin{subproblem}
	\pccname{Ellen} wants to make monthly payments of \$100 to pay off a debt of \$3000 
	at \unit[12]{\%} annual interest. How long will it take her to pay off the 
	debt?
	\begin{shortsolution}
		We are given $M=100$, $P=3000$, $i=0.01$, and we need to find $n$
		in the equation
		\[
			100 = \frac{3000\cdot 0.01}{1-(1+0.01)^{-n}}
		\]
		Using logarithms, we find that $n\approx 36$. It will take 
		Ellen about $3$ years to pay off the debt.
	\end{shortsolution}
\end{subproblem}
\begin{subproblem}
	\pccname{Jake} is going to buy a new car. He puts \$2000 down and wants to finance the
	remaining \$14,000. The dealer will offer him \unit[4]{\%} annual interest for 
	$5$ years, or a \$2000
	rebate which he can use to reduce the amount of the loan and \unit[8]{\%} 
	annual interest for 5 years. Which should he choose?
	\begin{shortsolution}
		\begin{description}
			\item[Option 1:] $\unit[4]{\%}$ annual interest for $5$ years on \$14,000.
			This means that the monthly payments will be calculated using
			\begin{align*}
				M & = \frac{14000\cdot \frac{0.04}{12}}{1-\left( 1+\frac{0.04}{12} \right)^{-12\cdot 5}} \\
				  & \approx 257.83                                                                       
			\end{align*}
			The monthly payments will be $\$257.83$. The total amount paid will be
			$\$257.83\cdot 5\cdot 12=\$15,469.80$, of which $\$1469.80$ is interest.
			\item[Option 2:] $\unit[8]{\%}$ annual interest for $5$ years on \$12,000.
			This means that the monthly payments will be calculated using
			\begin{align*}
				M & = \frac{12000\cdot \frac{0.08}{12}}{1-\left( 1+\frac{0.08}{12} \right)^{-12\cdot 5}} \\
				  & \approx 243.32                                                                       
			\end{align*}
			The monthly payments will be $\$243.32$. The total amount paid 
			will be $\$243.32\cdot 5\cdot 12 =\$14,599.20$, of which $\$2599.2$ is 
			interest.
		\end{description}
		Jake should choose option 1 to minimize the amount of interest 
		he has to pay.
	\end{shortsolution}
\end{subproblem}
\end{problem}

\begin{exercises}
%===================================
%   Author: Hughes
%   Date:   March 2012
%===================================
\begin{problem}[Rational or not]
Decide if each of the following functions are rational or not. If 
they are rational, state their domain.
\begin{multicols}{3}
	\begin{subproblem}
		$r(x)=\dfrac{3}{x}$    
		\begin{shortsolution}
			$r$ is rational; the domain of $r$ is $(-\infty,0)\cup (0,\infty)$.
		\end{shortsolution}
	\end{subproblem}
	\begin{subproblem}
		$s(y)=\dfrac{y}{6}$    
		\begin{shortsolution}
			$s$ is not rational ($s$ is linear). 
		\end{shortsolution}
	\end{subproblem}
	\begin{subproblem}
		$t(z)=\dfrac{4-x}{7-8z}$    
		\begin{shortsolution}
			$t$ is rational; the domain of $t$ is $\left( -\infty,\dfrac{7}{8} \right)\cup \left( \dfrac{7}{8},\infty \right)$.
		\end{shortsolution}
	\end{subproblem}
	\begin{subproblem}
		$u(w)=\dfrac{w^2}{(w-3)(w+4)}$ 
		\begin{shortsolution}
			$u$ is rational; the domain of $w$ is $(-\infty,-4)\cup(-4,3)\cup(3,\infty)$. 
		\end{shortsolution}
	\end{subproblem}
	\begin{subproblem}
		$v(x)=\dfrac{4}{(x-2)^2}$ 
		\begin{shortsolution}
			$v$ is rational; the domain of $v$ is $(-\infty,2)\cup(2,\infty)$. 
		\end{shortsolution}
	\end{subproblem}
	\begin{subproblem}
		$w(x)=\dfrac{9-x}{x+17}$ 
		\begin{shortsolution}
			$w$ is rational; the domain of $w$ is $(-\infty,-17)\cup(-17,\infty)$. 
		\end{shortsolution}
	\end{subproblem}
	\begin{subproblem}
		$a(x)=x^2+4$ 
		\begin{shortsolution}
			$a$ is not rational ($a$ is quadratic, or a polynomial of degree $2$).
		\end{shortsolution}
	\end{subproblem}
	\begin{subproblem}
		$b(y)=3^y$
		\begin{shortsolution}
			$b$ is not rational ($b$ is exponential).
		\end{shortsolution}
	\end{subproblem}
	\begin{subproblem}
		$c(z)=\dfrac{z^2}{z^3}$ 
		\begin{shortsolution}
			$c$ is rational; the domain of $c$ is $(-\infty,0)\cup (0,\infty)$.
		\end{shortsolution}
	\end{subproblem}
	\begin{subproblem}
		$d(x)=x^2(x+3)(5x-7)$ 
		\begin{shortsolution}
			$d$ is not rational ($d$ is a polynomial).
		\end{shortsolution}
	\end{subproblem}
	\begin{subproblem}
		$e(\alpha)=\dfrac{\alpha^2}{\alpha^2-1}$ 
		\begin{shortsolution}
			$e$ is rational; the domain of $e$ is $(-\infty,-1)\cup(-1,1)\cup(1,\infty)$.
		\end{shortsolution}
	\end{subproblem}
	\begin{subproblem}
		$f(\beta)=\dfrac{3}{4}$ 
		\begin{shortsolution}
			$f$ is not rational ($f$ is constant).
		\end{shortsolution}
	\end{subproblem}
\end{multicols}
\end{problem}
%===================================
%   Author: Hughes
%   Date:   March 2012
%===================================
\begin{problem}[Function evaluation]
Let $r$ be the function that has formula 
\[
	r(x)=\frac{(x-2)(x+3)}{(x+5)(x-7)}
\]
Evaluate each of the following (if possible); if the value is undefined, 
then state so.
\begin{multicols}{4}
	\begin{subproblem}
		$r(0)$ 
		\begin{shortsolution}
			$\begin{aligned}[t]
			r(0)&=\frac{(0-2)(0+3)}{(0+5)(0-7)}\\
			&=\frac{-6}{-35}\\
			&=\frac{6}{35}
		\end{aligned}$
	\end{shortsolution}
\end{subproblem}
\begin{subproblem}
	$r(1)$ 
	\begin{shortsolution}
		$\begin{aligned}[t]
		r(1)&=\frac{(1-2)(1+3)}{(1+5)(1-7)}\\
		&=\frac{-4}{-36}\\
		&=\frac{1}{9}
	\end{aligned}$
\end{shortsolution}
\end{subproblem}
\begin{subproblem}
	$r(2)$ 
	\begin{shortsolution}
		$\begin{aligned}[t]
		r(2)&=\frac{(2-2)(2+3)}{(2+5)(2-7)}\\
		& = \frac{0}{-50}\\
		&=0
	\end{aligned}$
\end{shortsolution}
\end{subproblem}
\begin{subproblem}
	$r(4)$ 
	\begin{shortsolution}
		$\begin{aligned}[t]
		r(4)&=\frac{(4-2)(4+3)}{(4+5)(4-7)}\\
		&=\frac{14}{-27}\\
		&=-\frac{14}{27}
	\end{aligned}$
\end{shortsolution}
\end{subproblem}
\begin{subproblem}
	$r(7)$ 
	\begin{shortsolution}
		$\begin{aligned}[t]
		r(7)&=\frac{(7-2)(7+3)}{(7+5)(7-7)}\\
		& =\frac{50}{0}
	\end{aligned}$
	
	$r(7)$ is undefined.
\end{shortsolution}
\end{subproblem}
\begin{subproblem}
	$r(-3)$ 
	\begin{shortsolution}
		$\begin{aligned}[t]
		r(-3)&=\frac{(-3-2)(-3+3)}{(-3+5)(-3-7)}\\
		&=\frac{0}{-20}\\
		&=0
	\end{aligned}$
\end{shortsolution}
\end{subproblem}
\begin{subproblem}
	$r(-5)$ 
	\begin{shortsolution}
		$\begin{aligned}[t]
		r(-5)&=\frac{(-5-2)(-5+3)}{(-5+5)(-5-7)}\\
		&=\frac{14}{0}
	\end{aligned}$
	
	$r(-5)$ is undefined.
\end{shortsolution}
\end{subproblem}
\begin{subproblem}
	$r\left( \frac{1}{2} \right)$
	\begin{shortsolution}
		$\begin{aligned}[t]
		r\left( \frac{1}{2} \right)& = \frac{\left( \frac{1}{2}-2 \right)\left( \frac{1}{2}+3 \right)}{\left( \frac{1}{2}+5 \right)\left( \frac{1}{2}-7 \right)}\\
		&=\frac{-\frac{3}{2}\cdot\frac{7}{2}}{\frac{11}{2}\left( -\frac{13}{2} \right)}\\
		&=\frac{-\frac{21}{4}}{-\frac{143}{4}}\\
		&=\frac{37}{143}
	\end{aligned}$
\end{shortsolution}
\end{subproblem}
\end{multicols}
\end{problem}
%===================================
%   Author: Hughes
%   Date:   March 2012
%===================================
\begin{problem}[Holes or asymptotes?]
State the domain of each of the following rational functions. Identify
any holes or asymptotes.
\begin{multicols}{3}
	\begin{subproblem}
		$f(x)=\dfrac{12}{x-2}$ 
		\begin{shortsolution}
			$f$ has a vertical asymptote at $2$; the domain of $f$ is $(-\infty,2)\cup (2,\infty)$. 
		\end{shortsolution}
	\end{subproblem}
	\begin{subproblem}
		$g(x)=\dfrac{x^2+x}{(x+1)(x-2)}$ 
		\begin{shortsolution}
			$g$ has a vertical asymptote at $2$, and a hole at $-1$; the domain of $g$ is $(-\infty,-1)\cup(-1,2)\cup(2,\infty)$. 
		\end{shortsolution}
	\end{subproblem}
	\begin{subproblem}
		$h(x)=\dfrac{x^2+5x+4}{x^2+x-12}$ 
		\begin{shortsolution}
			$h$ has a vertical asymptote at $3$, and a whole at $-4$; the domain of $h$ is $(-\infty,-4)\cup(-4,3)\cup(3,\infty)$. 
		\end{shortsolution}
	\end{subproblem}
	\begin{subproblem}
		$k(z)=\dfrac{z+2}{2z-3}$ 
		\begin{shortsolution}
			$k$ has a vertical asymptote at $\dfrac{3}{2}$; the domain of $k$ is $\left( -\infty,\dfrac{3}{2} \right)\cup\left( \dfrac{3}{2},\infty \right)$. 
		\end{shortsolution}
	\end{subproblem}
	\begin{subproblem}
		$l(w)=\dfrac{w}{w^2+1}$ 
		\begin{shortsolution}
			$l$ does not have any vertical asymptotes nor holes; the domain of $w$ is $(-\infty,\infty)$. 
		\end{shortsolution}
	\end{subproblem}
	\begin{subproblem}
		$m(t)=\dfrac{14}{13-t^2}$ 
		\begin{shortsolution}
			$m$ has vertical asymptotes at $\pm\sqrt{13}$; the domain of $m$ is $(-\infty,\sqrt{13})\cup(-\sqrt{13},\sqrt{13})\cup(\sqrt{13},\infty)$.
		\end{shortsolution}
	\end{subproblem}
\end{multicols}
\end{problem}

%===================================
%   Author: Hughes
%   Date:   May 2011
%===================================
\begin{problem}[Find a formula from a graph]
Consider the rational functions graphed in \cref{rat:fig:findformula}. Find 
the vertical asymptotes for each function, together with any zeros, and 
give a possible formula for each. 
\begin{shortsolution}
	\begin{itemize}
		\item \Vref{rat:fig:formula1}: possible formula is $r(x)=\dfrac{1}{x+5}$ 
		\item \Vref{rat:fig:formula2}: possible formula is $r(x)=\dfrac{(x+3)}{(x-5)}$
		\item \Vref{rat:fig:formula3}: possible formula is $r(x)=\dfrac{1}{(x-4)(x+3)}$.
	\end{itemize}
\end{shortsolution}
\end{problem}

\begin{figure}[!htb]
	\begin{widepage}
	\setlength{\figurewidth}{0.3\textwidth}
	\begin{subfigure}{\figurewidth}
		\begin{tikzpicture}[/pgf/declare function={f=1/(x+4);}]
			\begin{axis}[
			   framed,
			   xmin=-10,xmax=10,
			   ymin=-6,ymax=6,
			   xtick={-8,-6,...,8},
			   minor ytick={-4,-3,...,4},
			   grid=both,
			   width=\textwidth,
			   ]
			   \addplot[pccplot] expression[domain=-10:-4.16667,samples=50]{f};
			   \addplot[pccplot] expression[domain=-3.83333:10,samples=50]{f};
			   \addplot[asymptote,domain=-6:6]({-4},{x});
			\end{axis}
		\end{tikzpicture}
		\caption{}
		\label{rat:fig:formula1}
	\end{subfigure}
	\hfill
	\begin{subfigure}{\figurewidth}
		\begin{tikzpicture}[/pgf/declare function={f=(x+3)/(x-5);}]
			\begin{axis}[
			   framed,
			   xmin=-10,xmax=10,
			   ymin=-6,ymax=6,
			   xtick={-8,-6,...,8},
			   minor ytick={-4,-3,...,4},
			   grid=both,
			   width=\textwidth,
			   ]
			   \addplot[pccplot] expression[domain=-10:3.85714]{f};
			   \addplot[pccplot] expression[domain=6.6:10]{f};
			   \addplot[soldot] coordinates{(-3,0)};
			   \addplot[asymptote,domain=-6:6]({5},{x});
			   \addplot[asymptote,domain=-10:10]({x},{1});
			\end{axis}
		\end{tikzpicture}
		\caption{}
		\label{rat:fig:formula2}
	\end{subfigure}
	\hfill
	\begin{subfigure}{\figurewidth}
		\begin{tikzpicture}[/pgf/declare function={f=1/((x-4)*(x+3));}]
			\begin{axis}[
			   framed,
			   xmin=-10,xmax=10,
			   ymin=-3,ymax=3,
			   xtick={-8,-6,...,8},
			   minor ytick={-4,-3,...,4},
			   grid=both,
			   width=\textwidth,
			   ]
			   \addplot[pccplot] expression[domain=-10:-3.0473]{f};
			   \addplot[pccplot] expression[domain=-2.95205:3.95205]{f};
			   \addplot[pccplot] expression[domain=4.0473:10]{f};
			   \addplot[asymptote,domain=-3:3]({-3},{x});
			   \addplot[asymptote,domain=-3:3]({4},{x});
			   \addplot[asymptote,domain=-10:10]({x},{0});
			\end{axis}
		\end{tikzpicture}
		\caption{}
		\label{rat:fig:formula3}
	\end{subfigure}
	\caption{}
	\label{rat:fig:findformula}
	\end{widepage}
\end{figure}

%===================================
%   Author: Hughes
%   Date:   May 2011
%===================================
\begin{problem}[Find a formula from a description]
In each of the following problems, give a formula of a rational 
function that has the listed properties.
\begin{subproblem}
	Vertical asymptote at $2$.
	\begin{shortsolution}
		Possible option: $r(x)=\dfrac{1}{x-2}$. Note that we could multiply the 
		numerator or denominator by any real number and still have the desired properties.
	\end{shortsolution}
\end{subproblem}
\begin{subproblem}
	Vertical asymptote at $5$.
	\begin{shortsolution}
		Possible option: $r(x)=\dfrac{1}{x-5}$. Note that we could multiply the 
		numerator or denominator by any real number and still have the desired properties.
	\end{shortsolution}
\end{subproblem}
\begin{subproblem}
	Vertical asymptote at $-2$, and zero at $6$.
	\begin{shortsolution}
		Possible option: $r(x)=\dfrac{x-6}{x+2}$. Note that we could multiply the 
		numerator or denominator by any real number and still have the desired properties.
	\end{shortsolution}
\end{subproblem}
\begin{subproblem}
	Zeros at $2$ and $-5$ and vertical asymptotes at $1$ and $-7$.
	\begin{shortsolution}
		Possible option: $r(x)=\dfrac{(x-2)(x+5)}{(x-1)(x+7)}$. Note that we could multiply the 
		numerator or denominator by any real number and still have the desired properties.
	\end{shortsolution}
\end{subproblem}
\end{problem}

%===================================
%   Author: Hughes
%   Date:   May 2011
%===================================
\begin{problem}[Given formula, find horizontal asymptotes]
Each of the following functions has a horizontal asymptote. Write the equation
of the horizontal asymptote for each function.
\begin{multicols}{3}
	\begin{subproblem}
		$f(x) = \dfrac{1}{x}$
		\begin{shortsolution}
			$y=0$
		\end{shortsolution}
	\end{subproblem}
	\begin{subproblem}
		$g(x) = \dfrac{2x+3}{x}$
		\begin{shortsolution}
			$y=2$
		\end{shortsolution}
	\end{subproblem}
	\begin{subproblem}
		$h(x) = \dfrac{x^2+2x}{x^2+3}$
		\begin{shortsolution}
			$y=1$
		\end{shortsolution}
	\end{subproblem}
	\begin{subproblem}
		$k(x) = \dfrac{x^2+7}{x}$
		\begin{shortsolution}
			$y=1$
		\end{shortsolution}
	\end{subproblem}
	\begin{subproblem}
		$l(x)=\dfrac{3x-2}{5x+8}$ 
		\begin{shortsolution}
			$y=\dfrac{3}{5}$
		\end{shortsolution}
	\end{subproblem}
	\begin{subproblem}
		$m(x)=\dfrac{3x-2}{5x^2+8}$ 
		\begin{shortsolution}
			$y=0$
		\end{shortsolution}
	\end{subproblem}
	\begin{subproblem}
		$n(x)=\dfrac{(6x+1)(x-7)}{(11x-8)(x-5)}$ 
		\begin{shortsolution}
			$y=\dfrac{6}{11}$
		\end{shortsolution}
	\end{subproblem}
	\begin{subproblem}
		$p(x)=\dfrac{19x^3}{5-x^4}$ 
		\begin{shortsolution}
			$y=0$
		\end{shortsolution}
	\end{subproblem}
	\begin{subproblem}
		$q(x)=\dfrac{14x^2+x}{1-7x^2}$ 
		\begin{shortsolution}
			$y=-2$
		\end{shortsolution}
	\end{subproblem}
\end{multicols}
\end{problem}

%===================================
%   Author: Hughes
%   Date:   May 2012
%===================================
\begin{problem}[Given horizontal asymptotes, find formula]
In each of the following problems, give a formula for a function that 
has the given horizontal asymptote. Note that there may be more than one option.
\begin{multicols}{4}
	\begin{subproblem}
		$y=7$ 
		\begin{shortsolution}
			Possible option: $f(x)=\dfrac{7(x-2)}{x+1}$. Note that there
			are other options, provided that the degree of the numerator is the same as the degree
			of the denominator, and that the ratio of the leading 
			coefficients is $7$.
		\end{shortsolution}
	\end{subproblem}
	\begin{subproblem}
		$y=-1$ 
		\begin{shortsolution}
			Possible option: $f(x)=\dfrac{5-x^2}{x^2+10}$. Note that there
			are other options, provided that the degree of the numerator is the same as the degree
			of the denominator, and that the ratio of the leading 
			coefficients is $10$.
		\end{shortsolution}
	\end{subproblem}
	\begin{subproblem}
		$y=53$ 
		\begin{shortsolution}
			Possible option: $f(x)=\dfrac{53x^3}{x^3+4x^2-7}$. Note that there
			are other options, provided that the degree of the numerator is the same as the degree
			of the denominator, and that the ratio of the leading 
			coefficients is $53$.
		\end{shortsolution}
	\end{subproblem}
	\begin{subproblem}
		$y=-17$ 
		\begin{shortsolution}
			Possible option: $f(x)=\dfrac{34(x+2)}{7-2x}$. Note that there
			are other options, provided that the degree of the numerator is the same as the degree
			of the denominator, and that the ratio of the leading 
			coefficients is $-17$.
		\end{shortsolution}
	\end{subproblem}
	\begin{subproblem}
		$y=\dfrac{3}{2}$ 
		\begin{shortsolution}
			Possible option: $f(x)=\dfrac{3x+4}{2(x+1)}$. Note that there
			are other options, provided that the degree of the numerator is the same as the degree
			of the denominator, and that the ratio of the leading 
			coefficients is $\dfrac{3}{2}$.
		\end{shortsolution}
	\end{subproblem}
	\begin{subproblem}
		$y=0$ 
		\begin{shortsolution}
			Possible option: $f(x)=\dfrac{4}{x}$. Note that there
			are other options, provided that the degree of the numerator is less than the degree
			of the denominator.
		\end{shortsolution}
	\end{subproblem}
	\begin{subproblem}
		$y=-1$ 
		\begin{shortsolution}
			Possible option: $f(x)=\dfrac{10x}{5-10x}$. Note that there
			are other options, provided that the degree of the numerator is the same as the degree
			of the denominator, and that the ratio of the leading 
			coefficients is $-1$.
		\end{shortsolution}
	\end{subproblem}
	\begin{subproblem}
		$y=2$ 
		\begin{shortsolution}
			Possible option: $f(x)=\dfrac{8x-3}{4x+1}$. Note that there
			are other options, provided that the degree of the numerator is the same as the degree
			of the denominator, and that the ratio of the leading 
			coefficients is $2$.
		\end{shortsolution}
	\end{subproblem}
\end{multicols}
\end{problem}

%===================================
%   Author: Hughes
%   Date:   May 2011
%===================================
\begin{problem}[Find a formula from a description]
In each of the following problems, give a formula for a function that 
has the prescribed properties. Note that there may be more than one option.
\begin{subproblem}
	$f(x)\rightarrow 3$ as $x\rightarrow\pm\infty$.
	\begin{shortsolution}
		Possible option: $f(x)=\dfrac{3(x-2)}{x+7}$. Note that 
		the zero and asymptote of $f$ could be changed, and $f$ would still have the desired properties.
	\end{shortsolution}
\end{subproblem}
\begin{subproblem}
	$r(x)\rightarrow -4$ as $x\rightarrow\pm\infty$.
	\begin{shortsolution}
		Possible option: $r(x)=\dfrac{-4(x-2)}{x+7}$. Note that 
		the zero and asymptote of $r$ could be changed, and $r$ would still have the desired properties.
	\end{shortsolution}
\end{subproblem}
\begin{subproblem}
	$k(x)\rightarrow 2$ as $x\rightarrow\pm\infty$, and $k$ has vertical asymptotes at $-3$ and $5$.
	\begin{shortsolution}
		Possible option: $k(x)=\dfrac{2x^2}{(x+3)(x-5)}$. Note that the denominator 
		must have the given factors; the numerator could be any degree $2$ polynomial, provided the 
		leading coefficient is $2$.
	\end{shortsolution}
\end{subproblem}
\end{problem}

%===================================
%   Author: Hughes
%   Date:   Feb 2011
%===================================
\begin{problem}
Let $r$ be the rational function that has
\[
	r(x) = \frac{(x+2)(x-1)}{(x+3)(x-4)}
\]  
Each of the following questions are in relation to this function.
\begin{subproblem}
	What is the vertical intercept of this function? State your answer as an 
	ordered pair. \index{rational functions!vertical intercept}
	\begin{shortsolution}
		$\left(0,\frac{1}{6}\right)$
	\end{shortsolution}
\end{subproblem}
\begin{subproblem}\label{rat:prob:rational}
	What values of $x$ make the denominator equal to $0$?
	\begin{shortsolution}
		$-3,4$
	\end{shortsolution}
\end{subproblem}
\begin{subproblem}
	Use your answer to \cref{rat:prob:rational} to write the domain of the function in 
	both interval, and set builder notation. %\index{rational functions!domain}\index{domain!rational functions}
	\begin{shortsolution}
		Interval notation: $(-\infty,-3)\cup (-3,4)\cup (4,\infty)$. 
		Set builder: $\{x|x\ne -3, \mathrm{and}\, x\ne 4\}$
	\end{shortsolution}
\end{subproblem}
\begin{subproblem}
	What are the vertical asymptotes of the function? State your answers in 
	the form $x=$
	\begin{shortsolution}
		$x=-3$ and $x=4$
	\end{shortsolution}
\end{subproblem}
\begin{subproblem}\label{rat:prob:zeroes}
	What values of $x$ make the numerator equal to $0$?
	\begin{shortsolution}
		$-2,1$ 
	\end{shortsolution}
\end{subproblem}
\begin{subproblem}
	Use your answer to \cref{rat:prob:zeroes} to write the horizontal intercepts of 
	$r$ as ordered pairs.
	\begin{shortsolution}
		$(-2,0)$ and $(1,0)$
	\end{shortsolution}
\end{subproblem}
\end{problem}


%===================================
%   Author: Hughes
%   Date:   May 2011
%===================================
\begin{problem}[Holes]
\pccname{Josh} and \pccname{Pedro} are discussing the function
\[
	r(x)=\frac{x^2-1}{(x+3)(x-1)}
\]
\begin{subproblem}
	What is the domain of $r$?
	\begin{shortsolution}
		The domain of $r$ is $(-\infty,-3)\cup(-3,1)\cup(1,\infty)$.
	\end{shortsolution}
\end{subproblem}
\begin{subproblem}
	Josh notices that the numerator can be factored- can you see how?
	\begin{shortsolution}
		$(x^2-1)=(x-1)(x+1)$
	\end{shortsolution}
\end{subproblem}
\begin{subproblem}
	Pedro asks, `Doesn't that just mean that 
	\[
		r(x)=\frac{x+1}{x+3}
	\]
	for all values of $x$?' Josh says, `Nearly\ldots but not for all values of $x$'. 
	What does Josh mean?
	\begin{shortsolution}
		$r(x)=\dfrac{x+1}{x+3}$ provided that $x\ne -1$. 
	\end{shortsolution}
\end{subproblem}
\begin{subproblem}
	Where does $r$ have vertical asymptotes, and where does it have holes?
	\begin{shortsolution}
		The function $r$ has a vertical asymptote at $-3$, and a hole at $1$.
	\end{shortsolution}
\end{subproblem}
\begin{subproblem}
	Sketch a graph of $r$.
	\begin{shortsolution}
		A graph of $r$ is shown below.
		
		\begin{tikzpicture}
			\begin{axis}[
			   framed,
			   xmin=-10,xmax=10,
			   ymin=-10,ymax=10,
			   xtick={-8,-6,...,8},
			   ytick={-8,-6,...,8},
			   grid=both,
			   width=\solutionfigurewidth,
			   ]
			   \addplot[pccplot] expression[domain=-10:-3.25]{(x+1)/(x+3)};
			   \addplot[pccplot] expression[domain=-2.75:10]{(x+1)/(x+3)};
			   \addplot[asymptote,domain=-10:10]({-3},{x});
			   \addplot[holdot]coordinates{(1,0.5)};
			\end{axis}
		\end{tikzpicture}
	\end{shortsolution}
\end{subproblem}
\end{problem}

%===================================
%   Author: Hughes
%   Date:   July 2012
%===================================
\begin{problem}[Function algebra]
Let $r$ and $s$ be the rational functions that have formulas
\[
	r(x)=\frac{2-x}{x+3}, \qquad s(x)=\frac{x^2}{x-4}
\]
Evaluate each of the following (if possible).
\begin{multicols}{4}
	\begin{subproblem}
		$(r+s)(5)$ 
		\begin{shortsolution}
			$\frac{197}{8}$ 
		\end{shortsolution}
	\end{subproblem}
	\begin{subproblem}
		$(r-s)(3)$
		\begin{shortsolution}
			$\frac{53}{6}$ 
		\end{shortsolution}
	\end{subproblem}
	\begin{subproblem}
		$(r\cdot s)(4)$ 
		\begin{shortsolution}
			Undefined.
		\end{shortsolution}
	\end{subproblem}
	\begin{subproblem}
		$\left( \frac{r}{s} \right)(1)$ 
		\begin{shortsolution}
			$-\frac{3}{4}$ 
		\end{shortsolution}
	\end{subproblem}
\end{multicols}
\end{problem}


%===================================
%   Author: Hughes
%   Date:   July 2012
%===================================
\begin{problem}[Transformations: given the transformation, find the formula]
Let $r$ be the rational function that has formula.
\[
	r(x)=\frac{x+5}{2x-3}
\]
In each of the following problems apply the given transformation to the function $r$ and 
write a formula for the transformed version of $r$.
\begin{multicols}{2}
	\begin{subproblem}
		Shift $r$ to the right by $3$ units. 
		\begin{shortsolution}
			$r(x-3)=\frac{x+2}{2x-9}$
		\end{shortsolution}
	\end{subproblem}
	\begin{subproblem}
		Shift $r$ to the left by $4$ units. 
		\begin{shortsolution}
			$r(x+4)=\frac{x+9}{2x+5}$
		\end{shortsolution}
	\end{subproblem}
	\begin{subproblem}
		Shift $r$ up by $\pi$ units. 
		\begin{shortsolution}
			$r(x)+\pi=\frac{x+5}{2x-3}+\pi$
		\end{shortsolution}
	\end{subproblem}
	\begin{subproblem}
		Shift $r$ down by $17$ units. 
		\begin{shortsolution}
			$r(x)-17=\frac{x+5}{2x-3}-17$
		\end{shortsolution}
	\end{subproblem}
	\begin{subproblem}
		Reflect $r$ over the horizontal axis.
		\begin{shortsolution}
			$-r(x)=-\frac{x+5}{2x-3}$
		\end{shortsolution}
	\end{subproblem}
	\begin{subproblem}
		Reflect $r$ over the vertical axis.
		\begin{shortsolution}
			$r(-x)=\frac{x-5}{2x+3}$
		\end{shortsolution}
	\end{subproblem}
\end{multicols}
\end{problem}


%===================================
%   Author: Hughes
%   Date:   May 2011
%===================================
\begin{problem}[Find a formula from a table]\label{rat:prob:findformula}
\Crefrange{rat:tab:findformular}{rat:tab:findformulau} show values of rational functions $r$, $q$, $s$, 
and $t$. Assume that any values marked with an X are undefined.

\begin{table}[!htb]
	\begin{widepage}
	\centering
	\caption{Tables for \cref{rat:prob:findformula}}
	\label{rat:tab:findformula}
	\begin{subtable}{.2\textwidth}
		\centering
		\caption{$y=r(x)$}
		\label{rat:tab:findformular}
		\begin{tabular}{rr}
			\beforeheading
			$x$  & $y$                \\ \afterheading
			$-4$ & $\nicefrac{7}{2}$  \\\normalline
			$-3$ & $-18$              \\\normalline
			$-2$ & X                  \\\normalline
			$-1$ & $-4$               \\\normalline
			$0$  & $\nicefrac{-3}{2}$ \\\normalline
			$1$  & $\nicefrac{-2}{3}$ \\\normalline
			$2$  & $\nicefrac{-1}{4}$ \\\normalline
			$3$  & $0$                \\\normalline
			$4$  & $\nicefrac{1}{6}$  \\\lastline
		\end{tabular}
	\end{subtable}
	\hfill
	\begin{subtable}{.2\textwidth}
		\centering
		\caption{$y=s(x)$}
		\label{rat:tab:findformulas}
		\begin{tabular}{rr}
			\beforeheading
			$x$  & $y$                 \\ \afterheading
			$-4$ & $\nicefrac{-2}{21}$ \\\normalline
			$-3$ & $\nicefrac{-1}{12}$ \\\normalline
			$-2$ & $0$                 \\\normalline
			$-1$ & X                   \\\normalline
			$0$  & $\nicefrac{-2}{3}$  \\\normalline
			$1$  & $\nicefrac{-3}{4}$  \\\normalline
			$2$  & $\nicefrac{-4}{3}$  \\\normalline
			$3$  & X                   \\\normalline
			$4$  & $\nicefrac{6}{5}$   \\\lastline
		\end{tabular}
	\end{subtable}
	\hfill
	\begin{subtable}{.2\textwidth}
		\centering
		\caption{$y=t(x)$}
		\label{rat:tab:findformulat}
		\begin{tabular}{rr}
			\beforeheading
			$x$  & $y$               \\ \afterheading
			$-4$ & $\nicefrac{3}{5}$ \\\normalline
			$-3$ & $0$               \\\normalline
			$-2$ & X                 \\\normalline
			$-1$ & $3$               \\\normalline
			$0$  & $3$               \\\normalline
			$1$  & X                 \\\normalline
			$2$  & $0$               \\\normalline
			$3$  & $\nicefrac{3}{5}$ \\\normalline
			$4$  & $\nicefrac{7}{9}$ \\\lastline
		\end{tabular}
	\end{subtable}
	\hfill
	\begin{subtable}{.2\textwidth}
		\centering
		\caption{$y=u(x)$}
		\label{rat:tab:findformulau}
		\begin{tabular}{rr}
			\beforeheading
			$x$  & $y$                \\ \afterheading
			$-4$ & $\nicefrac{16}{7}$ \\\normalline
			$-3$ & X                  \\\normalline
			$-2$ & $-\nicefrac{4}{5}$ \\\normalline
			$-1$ & $-\nicefrac{1}{8}$ \\\normalline
			$0$  & $0$                \\\normalline
			$1$  & $-\nicefrac{1}{8}$ \\\normalline
			$2$  & $-\nicefrac{4}{5}$ \\\normalline
			$3$  & X                  \\\normalline
			$4$  & $\nicefrac{16}{7}$ \\\lastline
		\end{tabular}
	\end{subtable}
	\end{widepage}
\end{table}
\begin{subproblem}
	Given that the formula for $r(x)$ has the form $r(x)=\dfrac{x-A}{x-B}$, use \cref{rat:tab:findformular}
	to find values of $A$ and $B$.
	\begin{shortsolution}
		$A=3$ and $B=-2$, so $r(x)=\dfrac{x-3}{x+2}$.
	\end{shortsolution}
\end{subproblem}
\begin{subproblem}
	Check your formula by computing $r(x)$ at the values specified in the table.
	\begin{shortsolution}
		$\begin{aligned}[t]
		r(-4)&= \frac{-4-3}{-4+2}\\
		&= \frac{7}{2}\\
	\end{aligned}$
	
	$r(-3)=\ldots$ etc
\end{shortsolution}
\end{subproblem}
\begin{subproblem}
	The function $s$ in \cref{rat:tab:findformulas} has two vertical asymptotes and one zero. 
	Can you find a formula for $s(x)$?
	\begin{shortsolution}
		$s(x)=\dfrac{x+2}{(x-3)(x+1)}$
	\end{shortsolution}
\end{subproblem}
\begin{subproblem}
	Check your formula by computing $s(x)$ at the values specified in the table.
	\begin{shortsolution}
		$\begin{aligned}[t]
		s(-4)&=\frac{-4+2}{(-4-3)(-4+1)}\\
		&=-\frac{2}{21}
	\end{aligned}$
	
	$s(-3)=\ldots$ etc
\end{shortsolution}
\end{subproblem}
\begin{subproblem}
	Given that the formula for $t(x)$ has  the form $t(x)=\dfrac{(x-A)(x-B)}{(x-C)(x-D)}$, use \cref{rat:tab:findformulat} to find the 
	values of $A$, $B$, $C$, and $D$; hence write a formula for $t(x)$.
	\begin{shortsolution}
		$t(x)=\dfrac{(x+3)(x-2)}{(x+2)(x+1)}$ 
	\end{shortsolution}
\end{subproblem}
\begin{subproblem}
	Given that the formula for $u(x)$ has  the form $u(x)=\dfrac{(x-A)^2}{(x-B)(x-C)}$, use \cref{rat:tab:findformulau} to find the 
	values of $A$, $B$, and $C$; hence write a formula for $u(x)$.
	\begin{shortsolution}
		$u(x)=\dfrac{x^2}{(x+3)(x-3)}$ 
	\end{shortsolution}
\end{subproblem}
\end{problem}
\end{exercises}

\section{Graphing rational functions (horizontal asymptotes)}
\reformatstepslist{R} % the steps list should be R1, R2, \ldots
We studied rational functions in the previous section, but were 
not asked to graph them; in this section we will demonstrate the 
steps to be followed in order to sketch graphs of the functions. 

Remember from \vref{rat:def:function} that rational functions have 
the form
\[
	r(x)=\frac{p(x)}{q(x)}
\]
In this section we will restrict attention to the case when 
\[
	\text{degree of }p\leq \text{degree of }q
\]
Note that this necessarily means that each function that we consider
in this section \emph{will have a horizontal asymptote} (see \vref{rat:def:longrun}).
The cases in which the degree of $p$ is greater than the degree of $q$ 
is covered in the next section.

Before we begin, it is important to remember the following:
\begin{itemize}
	\item Our sketches will give a good representation of the overall 
	shape of the graph, but until we have the tools of calculus (from MTH 251)
	we can not find local minimums, local maximums, and inflection points algebraically. This
	means that we will make our best guess as to where these points are.
	\item We will not concern ourselves too much with the vertical scale (because of 
	our previous point)| we will, however, mark the vertical intercept (assuming there is one), 
	and any horizontal asymptotes.
\end{itemize}
\begin{pccspecialcomment}[Steps to follow when sketching rational functions]\label{rat:def:stepsforsketch}
	\begin{steps}
		\item \label{rat:step:first} Find all vertical asymptotes and holes, and mark them on the 
		graph using dashed vertical lines and open circles $\circ$ respectively.
		\item Find any intercepts, and mark them using solid circles $\bullet$;
		determine if the curve cuts the axis, or bounces off it at each zero.
		\item Determine the behavior of the function around each asymptote| does
		it behave like $\frac{1}{x}$ or $\frac{1}{x^2}$?
		\item \label{rat:step:penultimate} Determine the long-run behavior of the function, and mark the horizontal 
		asymptote using a dashed horizontal line.
		\item \label{rat:step:last}  Deduce the overall shape of the curve, and sketch it. If there isn't
		enough information from the previous steps, then construct a table of values
		including sample points from each branch.
	\end{steps}
	Remember that until we have the tools of calculus, we won't be able to 
	find the exact coordinates of local minimums, local maximums, and points
	of inflection.
\end{pccspecialcomment}

The examples that follow show how \crefrange{rat:step:first}{rat:step:last} can be
applied to a variety of different rational functions.

%===================================
%   Author: Hughes
%   Date:   May 2012
%===================================
\begin{pccexample}\label{rat:ex:1overxminus2p2}
Use \crefrange{rat:step:first}{rat:step:last} to sketch a graph of the function $r$ 
that has formula
\[
	r(x)=\frac{1}{x-2}
\]
\begin{pccsolution}
\begin{steps}
	\item $r$ has a vertical asymptote at $2$; $r$ does not have any holes. The curve of 
	$r$ will have $2$ branches.
	\item $r$ does not have any zeros since the numerator is never equal to $0$. The
	vertical intercept of $r$ is $\left( 0,-\frac{1}{2} \right)$.
	\item $r$ behaves like $\frac{1}{x}$ around its vertical asymptote since $(x-2)$
	is raised to the power $1$.
	\item Since the degree of the numerator is less than the degree of the denominator, 
	according to \vref{rat:def:longrun} the horizontal asymptote of $r$ has equation $y=0$.
	\item We put the details we have obtained so far on \cref{rat:fig:1overxminus2p1}. Notice 
	that there is only one way to complete the graph, which we have done in \cref{rat:fig:1overxminus2p2}.
\end{steps}
\end{pccsolution}
\end{pccexample}

\begin{figure}[!htbp]
	\begin{subfigure}{.45\textwidth}
		\begin{tikzpicture}
			\begin{axis}[
			   xmin=-5,xmax=5,
			   ymin=-5,ymax=5,
			   width=\textwidth,
			   ]
			   \addplot[asymptote,domain=-5:5]({2},{x});
			   \addplot[asymptote,domain=-5:5]({x},{0});
			   \addplot[soldot] coordinates{(0,-0.5)}node[axisnode,anchor=north east]{$\left( 0,-\frac{1}{2} \right)$};
			\end{axis}
		\end{tikzpicture}
		\caption{}
		\label{rat:fig:1overxminus2p1}
	\end{subfigure}%
	\hfill
	\begin{subfigure}{.45\textwidth}
		\begin{tikzpicture}[/pgf/declare function={f=1/(x-2);}]
			\begin{axis}[
			   xmin=-5,xmax=5,
			   ymin=-5,ymax=5,
			   width=\textwidth,
			   ]
			   \addplot[pccplot] expression[domain=-5:1.8,samples=50]{f};
			   \addplot[pccplot] expression[domain=2.2:5]{f};
			   \addplot[asymptote,domain=-5:5]({2},{x});
			   \addplot[asymptote,domain=-5:5]({x},{0});
			   \addplot[soldot] coordinates{(0,-0.5)}node[axisnode,anchor=north east]{$\left( 0,-\frac{1}{2} \right)$};
			\end{axis}
		\end{tikzpicture}
		\caption{}
		\label{rat:fig:1overxminus2p2}
	\end{subfigure}%
	\caption{$y=\dfrac{1}{x-2}$}
\end{figure}

The function $r$ in \cref{rat:ex:1overxminus2p2} has a horizontal asymptote which has equation $y=0$.
This asymptote lies on the horizontal axis, and you might (understandably) find it hard 
to distinguish between the two lines (\cref{rat:fig:1overxminus2p2}). When faced
with such a situation, it is perfectly acceptable to draw the horizontal axis 
as a dashed line| just make sure to label it correctly. We will demonstrate this 
in the next example.

%===================================
%   Author: Hughes
%   Date:   May 2012
%===================================
\begin{pccexample}\label{rat:ex:1overxp1}
Use \crefrange{rat:step:first}{rat:step:last} to sketch a graph of the function $v$ 
that has formula
\[
	v(x)=\frac{10}{x}
\]
\begin{pccsolution}
\begin{steps}
	\item $v$ has a vertical asymptote at $0$. $v$ does not have
	any holes. The curve of $v$ will have $2$ branches.
	\item $v$ does not have any zeros (since $10\ne 0$). Furthermore, $v$ 
	does not have a vertical intercept since $v(0)$ is undefined.
	\item $v$ behaves like $\frac{1}{x}$ around its vertical asymptote.
	\item $v$ has a horizontal asymptote with equation $y=0$.
	\item We put the details we have obtained so far in \cref{rat:fig:1overxp1}. 
	We do not have enough information to sketch $v$ yet (because $v$ does 
	not have any intercepts), so let's pick a sample 
	point in either of the $2$ branches| it doesn't matter where our sample point 
	is, because we know what the overall shape will be. Let's compute $v(2)$
	\begin{align*}
		v(2) & =\dfrac{10}{2} \\
		     & = 5            
	\end{align*}
	We therefore mark the point $(2,5)$ on \cref{rat:fig:1overxp2}, and then complete the sketch using 
	the details we found in the previous steps.
\end{steps}

\begin{figure}[!htbp]
	\begin{subfigure}{.45\textwidth}
		\begin{tikzpicture}
			\begin{axis}[
			   xmin=-10,xmax=10,
			   ymin=-10,ymax=10,
			   xtick={-5,5},
			   ytick={-5,5},
			   axis line style={color=white},
			   width=\textwidth,
			   ]
			   \addplot[asymptote,<->,domain=-10:10]({0},{x});
			   \addplot[asymptote,<->,domain=-10:10]({x},{0});
			\end{axis}
		\end{tikzpicture}
		\caption{}
		\label{rat:fig:1overxp1}
	\end{subfigure}%
	\hfill
	\begin{subfigure}{.45\textwidth}
		\begin{tikzpicture}[/pgf/declare function={f=10/x;}]
			\begin{axis}[
			   xmin=-10,xmax=10,
			   ymin=-10,ymax=10,
			   xtick={-5,5},
			   ytick={-5,5},
			   axis line style={color=white},
			   width=\textwidth,
			   ]
			   \addplot[pccplot] expression[domain=-10:-1]{f};
			   \addplot[pccplot] expression[domain=1:10]{f};
			   \addplot[soldot] coordinates{(2,5)}node[axisnode,anchor=south west]{$(2,5)$};
			   \addplot[asymptote,<->,domain=-10:10]({0},{x});
			   \addplot[asymptote,<->,domain=-10:10]({x},{0});
			\end{axis}
		\end{tikzpicture}
		\caption{}
		\label{rat:fig:1overxp2}
	\end{subfigure}%
	\caption{$y=\dfrac{10}{x}$}
\end{figure}
\end{pccsolution}
\end{pccexample}

%===================================
%   Author: Hughes
%   Date:   May 2012
%===================================
\begin{pccexample}\label{rat:ex:asympandholep1}
Use \crefrange{rat:step:first}{rat:step:last} to sketch a graph of the function $u$ 
that has formula
\[
	u(x)=\frac{-4(x^2-9)}{x^2-8x+15}
\]
\begin{pccsolution}
\begin{steps}
	\item We begin by factoring both the numerator and denominator of $u$ to help
	us find any vertical asymptotes or holes
	\begin{align*}
		u(x) & =\frac{-4(x^2-9)}{x^2-8x+15}     \\
		     & =\frac{-4(x+3)(x-3)}{(x-5)(x-3)} \\
		     & =\frac{-4(x+3)}{x-5}             
	\end{align*}
	provided that $x\ne 3$. Therefore $u$ has a vertical asymptote at $5$ and 
	a hole at $3$. The curve of $u$ has $2$ branches.
	\item $u$ has a simple zero at $-3$. The vertical intercept of $u$ is $\left( 0,\frac{12}{5} \right)$.
	\item $u$ behaves like $\frac{1}{x}$ around its vertical asymptote at $4$.
	\item Using \vref{rat:def:longrun} the equation of the horizontal asymptote of $u$ is $y=-4$.
	\item We put the details we have obtained so far on \cref{rat:fig:1overxminus2p1}. Notice 
	that there is only one way to complete the graph, which we have done in \cref{rat:fig:1overxminus2p2}.
\end{steps}

\begin{figure}[!htbp]
	\begin{subfigure}{.45\textwidth}
		\begin{tikzpicture}
			\begin{axis}[
			   xmin=-10,xmax=10,
			   ymin=-20,ymax=20,
			   xtick={-8,-6,...,8},
			   ytick={-10,10},
			   width=\textwidth,
			   ]
			   \addplot[asymptote,domain=-20:20]({4},{x});
			   \addplot[asymptote,domain=-10:10]({x},{-4});
			   \addplot[soldot] coordinates{(-3,0)(0,2.4)}node[axisnode,anchor=south east]{$\left( 0,\frac{12}{5} \right)$};
			   \addplot[holdot] coordinates{(3,12)};
			\end{axis}
		\end{tikzpicture}
		\caption{}
		\label{rat:fig:asympandholep1}
	\end{subfigure}%
	\hfill
	\begin{subfigure}{.45\textwidth}
		\begin{tikzpicture}[/pgf/declare function={f=-4*(x+3)/(x-5);}]
			\begin{axis}[
			   xmin=-10,xmax=10,
			   ymin=-20,ymax=20,
			   xtick={-8,-6,...,8},
			   ytick={-10,10},
			   width=\textwidth,
			   ]
			   \addplot[pccplot] expression[domain=-10:3.6666,samples=50]{f};
			   \addplot[pccplot] expression[domain=7:10]{f};
			   \addplot[asymptote,domain=-20:20]({5},{x});
			   \addplot[asymptote,domain=-10:10]({x},{-4});
			   \addplot[soldot] coordinates{(-3,0)(0,2.4)}node[axisnode,anchor=south east]{$\left( 0,\frac{12}{5} \right)$};
			   \addplot[holdot] coordinates{(3,12)};
			\end{axis}
		\end{tikzpicture}
		\caption{}
		\label{rat:fig:asympandholep2}
	\end{subfigure}%
	\caption{$y=\dfrac{-4(x+3)}{x-5}$}
\end{figure}
\end{pccsolution}
\end{pccexample}

\Cref{rat:ex:1overxminus2p2,rat:ex:1overxp1,rat:ex:asympandholep1} have focused on functions
that only have one vertical asymptote; the remaining examples in this section
concern functions that have more than one vertical asymptote. We will demonstrate
that \crefrange{rat:step:first}{rat:step:last} still apply.

%===================================
%   Author: Hughes
%   Date:   May 2012
%===================================
\begin{pccexample}\label{rat:ex:sketchtwoasymp}
Use \crefrange{rat:step:first}{rat:step:last} to sketch a graph of the function $w$ 
that has formula
\[
	w(x)=\frac{2(x+3)(x-5)}{(x+5)(x-4)}
\]
\begin{pccsolution}
\begin{steps}
	\item $w$ has vertical asymptotes at $-5$ and $4$. $w$ does not have 
	any holes. The curve of $w$ will have $3$ branches.
	\item $w$ has simple zeros at $-3$ and $5$. The vertical intercept of $w$ 
	is $\left( 0,\frac{3}{2} \right)$.
	\item $w$ behaves like $\frac{1}{x}$ around both of its vertical 
	asymptotes.
	\item The degree of the numerator of $w$ is $2$ and the degree of the 
	denominator of $w$ is also $2$. Using the ratio of the leading coefficients
	of the numerator and denominator, we say that $w$ has a horizontal 
	asymptote with equation $y=\frac{2}{1}=2$.
	\item We put the details we have obtained so far on \cref{rat:fig:sketchtwoasymptp1}. 
	
	The function $w$ is a little more complicated than the functions that 
	we have considered in the previous examples because the curve has $3$
	branches. When graphing such functions, it is generally a good idea to start with the branch
	for which you have the most information| in this case, that is the \emph{middle} branch
	on the interval $(-5,4)$.
	
	Once we have drawn the middle branch, there is only one way to complete the graph 
	(because of our observations about the behavior of $w$ around its vertical asymptotes), 
	which we have done in \cref{rat:fig:sketchtwoasymptp2}.
\end{steps}
\end{pccsolution}
\end{pccexample}

\begin{figure}[!htbp]
	\begin{subfigure}{.45\textwidth}
		\begin{tikzpicture}
			\begin{axis}[
			   xmin=-10,xmax=10,
			   ymin=-10,ymax=10,
			   xtick={-8,-6,...,8},
			   ytick={-5,5},
			   width=\textwidth,
			   ]
			   \addplot[asymptote,domain=-10:10]({-5},{x});
			   \addplot[asymptote,domain=-10:10]({4},{x});
			   \addplot[asymptote,domain=-10:10]({x},{2});
			   \addplot[soldot] coordinates{(-3,0)(5,0)};
			   \addplot[soldot] coordinates{(0,1.5)}node[axisnode,anchor=north west]{$\left( 0,\frac{3}{2} \right)$};
			\end{axis}
		\end{tikzpicture}
		\caption{}
		\label{rat:fig:sketchtwoasymptp1}
	\end{subfigure}%
	\hfill
	\begin{subfigure}{.45\textwidth}
		\begin{tikzpicture}[/pgf/declare function={f=2*(x+3)*(x-5)/( (x+5)*(x-4));}]
			\begin{axis}[
			   xmin=-10,xmax=10,
			   ymin=-10,ymax=10,
			   xtick={-8,-6,...,8},
			   ytick={-5,5},
			   width=\textwidth,
			   ]
			   \addplot[asymptote,domain=-10:10]({-5},{x});
			   \addplot[asymptote,domain=-10:10]({4},{x});
			   \addplot[asymptote,domain=-10:10]({x},{2});
			   \addplot[soldot] coordinates{(-3,0)(5,0)};
			   \addplot[soldot] coordinates{(0,1.5)}node[axisnode,anchor=north west]{$\left( 0,\frac{3}{2} \right)$};
			   \addplot[pccplot] expression[domain=-10:-5.56708]{f};
			   \addplot[pccplot] expression[domain=-4.63511:3.81708]{f};
			   \addplot[pccplot] expression[domain=4.13511:10]{f};
			\end{axis}
		\end{tikzpicture}
		\caption{}
		\label{rat:fig:sketchtwoasymptp2}
	\end{subfigure}%
	\caption{$y=\dfrac{2(x+3)(x-5)}{(x+5)(x-4)}$}
\end{figure}

The rational functions that we have considered so far have had simple
factors in the denominator; each function has behaved like $\frac{1}{x}$ 
around each of its vertical asymptotes. \Cref{rat:ex:2asympnozeros,rat:ex:2squaredasymp}
consider functions that have a repeated factor in the denominator.

%===================================
%   Author: Hughes
%   Date:   May 2012
%===================================
\begin{pccexample}\label{rat:ex:2asympnozeros}
Use \crefrange{rat:step:first}{rat:step:last} to sketch a graph of the function $f$ 
that has formula
\[
	f(x)=\frac{100}{(x+5)(x-4)^2}
\]
\begin{pccsolution}
\begin{steps}
	\item $f$ has vertical asymptotes at $-5$ and $4$. $f$ does not have 
	any holes. The curve of $f$ will have $3$ branches.
	\item $f$ does not have any zeros (since $100\ne 0$). The vertical intercept of $f$ 
	is $\left( 0,\frac{5}{4} \right)$.
	\item $f$ behaves like $\frac{1}{x}$ around $-5$ and behaves like $\frac{1}{x^2}$
	around $4$.
	\item The degree of the numerator of $f$ is $0$ and the degree of the 
	denominator of $f$ is $2$. $f$ has a horizontal asymptote with 
	equation $y=0$.
	\item We put the details we have obtained so far on \cref{rat:fig:2asympnozerosp1}. 
	
	The function $f$ is similar to the function $w$ that we considered in \cref{rat:ex:sketchtwoasymp}|
	it has two vertical asymptotes and $3$ branches, but in contrast to $w$ it does not have any zeros.
	
	We sketch $f$ in \cref{rat:fig:2asympnozerosp2}, using the middle branch as our guide
	because we have the most information about the function on the interval $(-5,4)$.
	
	Once we have drawn the middle branch, there is only one way to complete the graph 
	because of our observations about the behavior of $f$ around its vertical asymptotes (it behaves like $\frac{1}{x}$), 
	which we have done in \cref{rat:fig:2asympnozerosp2}.
	
	Note that we are not yet able to find the local minimum of $f$ algebraically on the interval $(-5,4)$, 
	so we make a reasonable guess as to where it is| we can be confident that it is above the horizontal axis
	since $f$ has no zeros. You may think that this is unsatisfactory, but once we have the tools of calculus, we will 
	be able to find local minimums more precisely.
\end{steps}
\end{pccsolution}
\end{pccexample}

\begin{figure}[!htbp]
	\begin{subfigure}{.45\textwidth}
		\begin{tikzpicture}
			\begin{axis}[
			   xmin=-10,xmax=10,
			   ymin=-10,ymax=10,
			   xtick={-8,-6,...,8},
			   ytick={-5,5},
			   width=\textwidth,
			   ]
			   \addplot[asymptote,domain=-10:10]({-5},{x});
			   \addplot[asymptote,domain=-10:10]({4},{x});
			   \addplot[asymptote,domain=-10:10]({x},{0});
			   \addplot[soldot] coordinates{(0,1.25)}node[axisnode,anchor=south east]{$\left( 0,\frac{5}{4} \right)$};
			\end{axis}
		\end{tikzpicture}
		\caption{}
		\label{rat:fig:2asympnozerosp1}
	\end{subfigure}%
	\hfill
	\begin{subfigure}{.45\textwidth}
		\begin{tikzpicture}[/pgf/declare function={f=100/( (x+5)*(x-4)^2);}]
			\begin{axis}[
			   xmin=-10,xmax=10,
			   ymin=-10,ymax=10,
			   xtick={-8,-6,...,8},
			   ytick={-5,5},
			   width=\textwidth,
			   ]
			   \addplot[asymptote,domain=-10:10]({-5},{x});
			   \addplot[asymptote,domain=-10:10]({4},{x});
			   \addplot[asymptote,domain=-10:10]({x},{0});
			   \addplot[soldot] coordinates{(0,1.25)}node[axisnode,anchor=south east]{$\left( 0,\frac{5}{4} \right)$};
			   \addplot[pccplot] expression[domain=-10:-5.12022]{f};
			   \addplot[pccplot] expression[domain=-4.87298:2.87298,samples=50]{f};
			   \addplot[pccplot] expression[domain=5:10]{f};
			\end{axis}
		\end{tikzpicture}
		\caption{}
		\label{rat:fig:2asympnozerosp2}
	\end{subfigure}%
	\caption{$y=\dfrac{100}{(x+5)(x-4)^2}$}
\end{figure}

%===================================
%   Author: Hughes
%   Date:   May 2012
%===================================
\begin{pccexample}\label{rat:ex:2squaredasymp}
Use \crefrange{rat:step:first}{rat:step:last} to sketch a graph of the function $g$ 
that has formula
\[
	g(x)=\frac{50(2-x)}{(x+3)^2(x-5)^2}
\]
\begin{pccsolution}
\begin{steps}
	\item $g$ has vertical asymptotes at $-3$ and $5$. $g$ does 
	not have any holes. The curve of $g$ will have $3$ branches.
	\item $g$ has a simple zero at $2$. The vertical intercept of $g$ is
	$\left( 0,\frac{4}{9} \right)$.
	\item $g$ behaves like $\frac{1}{x^2}$ around both of its 
	vertical asymptotes.
	\item The degree of the numerator of $g$ is $1$ and the degree of the denominator
	of $g$ is $4$. Using \vref{rat:def:longrun}, we calculate that 
	the horizontal asymptote of $g$ has equation $y=0$.
	\item The details that we have found so far have been drawn in 
	\cref{rat:fig:2squaredasymp1}. The function $g$ is similar to the functions 
	we considered in \cref{rat:ex:sketchtwoasymp,rat:ex:2asympnozeros} because 
	it has $2$ vertical asymptotes and $3$ branches. 
	
	We sketch $g$ using the middle branch as our guide because we have the most information
	about $g$ on the interval $(-3,5)$. Note that there is no other way to draw this branch
	without introducing other zeros which $g$ does not have.
	
	Once we have drawn the middle branch, there is only one way to complete the graph 
	because of our observations about the behavior of $g$ around its vertical asymptotes| it 
	behaves like $\frac{1}{x^2}$.
	
\end{steps}
\end{pccsolution}
\end{pccexample}

\begin{figure}[!htbp]
	\begin{subfigure}{.45\textwidth}
		\begin{tikzpicture}
			\begin{axis}[
			   xmin=-10,xmax=10,
			   ymin=-10,ymax=10,
			   xtick={-8,-6,...,8},
			   ytick={-5,5},
			   width=\textwidth,
			   ]
			   \addplot[asymptote,domain=-10:10]({-3},{x});
			   \addplot[asymptote,domain=-10:10]({5},{x});
			   \addplot[asymptote,domain=-10:10]({x},{0});
			   \addplot[soldot] coordinates{(2,0)(0,4/9)}node[axisnode,anchor=south west]{$\left( 0,\frac{4}{9} \right)$};
			\end{axis}
		\end{tikzpicture}
		\caption{}
		\label{rat:fig:2squaredasymp1}
	\end{subfigure}%
	\hfill
	\begin{subfigure}{.45\textwidth}
		\begin{tikzpicture}[/pgf/declare function={f=50*(2-x)/( (x+3)^2*(x-5)^2);}]
			\begin{axis}[
			   xmin=-10,xmax=10,
			   ymin=-10,ymax=10,
			   xtick={-8,-6,...,8},
			   ytick={-5,5},
			   width=\textwidth,
			   ]
			   \addplot[asymptote,domain=-10:10]({-3},{x});
			   \addplot[asymptote,domain=-10:10]({5},{x});
			   \addplot[asymptote,domain=-10:10]({x},{0});
			   \addplot[soldot] coordinates{(2,0)(0,4/9)}node[axisnode,anchor=south west]{$\left( 0,\frac{4}{9} \right)$};
			   \addplot[pccplot] expression[domain=-10:-3.61504]{f};
			   \addplot[pccplot] expression[domain=-2.3657:4.52773]{f};
			   \addplot[pccplot] expression[domain=5.49205:10]{f};
			\end{axis}
		\end{tikzpicture}
		\caption{}
		\label{rat:fig:2squaredasymp2}
	\end{subfigure}%
	\caption{$y=\dfrac{50(2-x)}{(x+3)^2(x-5)^2}$}
\end{figure}

Each of the rational functions that we have considered so far has had either 
a \emph{simple} zero, or no zeros at all. Remember from our work on polynomial 
functions, and particularly \vref{poly:def:multzero}, that a \emph{repeated} zero
corresponds to the curve of the function behaving differently at the zero
when compared to how the curve behaves at a simple zero. \Cref{rat:ex:doublezero} details a 
function that has a non-simple zero.

%===================================
%   Author: Hughes
%   Date:   June 2012
%===================================
\begin{pccexample}\label{rat:ex:doublezero}
Use \crefrange{rat:step:first}{rat:step:last} to sketch a graph of the function $g$ 
that has formula
\[
	h(x)=\frac{(x-3)^2}{(x+4)(x-6)}
\]
\begin{pccsolution}
\begin{steps}
	\item $h$ has vertical asymptotes at $-4$ and $6$. $h$ does 
	not have any holes. The curve of $h$ will have $3$ branches.
	\item $h$ has a zero at $3$ that has \emph{multiplicity $2$}. 
	The vertical intercept of $h$ is
	$\left( 0,-\frac{3}{8} \right)$.
	\item $h$ behaves like $\frac{1}{x}$ around both of its 
	vertical asymptotes.
	\item The degree of the numerator of $h$ is $2$ and the degree of the denominator
	of $h$ is $2$. Using \vref{rat:def:longrun}, we calculate that 
	the horizontal asymptote of $h$ has equation $y=1$.
	\item The details that we have found so far have been drawn in 
	\cref{rat:fig:doublezerop1}. The function $h$ is different 
	from the functions that we have considered in previous examples because 
	of the multiplicity of the zero at $3$.
	
	We sketch $h$ using the middle branch as our guide because we have the most information
	about $h$ on the interval $(-4,6)$. Note that there is no other way to draw this branch
	without introducing other zeros which $h$ does not have| also note how 
	the curve bounces off the horizontal axis at $3$.
	
	Once we have drawn the middle branch, there is only one way to complete the graph 
	because of our observations about the behavior of $h$ around its vertical asymptotes| it 
	behaves like $\frac{1}{x}$.
	
\end{steps}
\end{pccsolution}
\end{pccexample}

\begin{figure}[!htbp]
	\begin{subfigure}{.45\textwidth}
		\begin{tikzpicture}
			\begin{axis}[
			   xmin=-10,xmax=10,
			   ymin=-5,ymax=5,
			   xtick={-8,-6,...,8},
			   ytick={-3,3},
			   width=\textwidth,
			   ]
			   \addplot[asymptote,domain=-10:10]({-4},{x});
			   \addplot[asymptote,domain=-10:10]({6},{x});
			   \addplot[asymptote,domain=-10:10]({x},{1});
			   \addplot[soldot] coordinates{(3,0)(0,-3/8)}node[axisnode,anchor=north west]{$\left( 0,-\frac{3}{8} \right)$};
			\end{axis}
		\end{tikzpicture}
		\caption{}
		\label{rat:fig:doublezerop1}
	\end{subfigure}%
	\hfill
	\begin{subfigure}{.45\textwidth}
		\begin{tikzpicture}[/pgf/declare function={f=(x-3)^2/((x+4)*(x-6));}]
			\begin{axis}[
			   xmin=-10,xmax=10,
			   ymin=-5,ymax=5,
			   xtick={-8,-6,...,8},
			   ytick={-3,3},
			   width=\textwidth,
			   ]
			   \addplot[asymptote,domain=-10:10]({-4},{x});
			   \addplot[asymptote,domain=-10:10]({6},{x});
			   \addplot[asymptote,domain=-10:10]({x},{1});
			   \addplot[soldot] coordinates{(3,0)(0,-3/8)}node[axisnode,anchor=north west]{$\left( 0,-\frac{3}{8} \right)$};
			   \addplot[pccplot] expression[domain=-10:-5.20088]{f};
			   \addplot[pccplot] expression[domain=-3.16975:5.83642,samples=50]{f};
			   \addplot[pccplot] expression[domain=6.20088:10]{f};
			\end{axis}
		\end{tikzpicture}
		\caption{}
		\label{rat:fig:doublezerop2}
	\end{subfigure}%
	\caption{$y=\dfrac{(x-3)^2}{(x+4)(x-6)}$}
\end{figure}
\begin{exercises}
%===================================
%   Author: Hughes
%   Date:   June 2012
%===================================
\begin{problem}[\Cref{rat:step:last}]\label{rat:prob:deduce}
\pccname{Katie} is working on graphing rational functions. She 
has been concentrating on functions that have the form
\begin{equation}\label{rat:eq:deducecurve}
	f(x)=\frac{a(x-b)}{x-c}
\end{equation}
Katie notes that functions with this type of formula have a zero
at $b$, and a vertical asymptote at $c$. Furthermore, these functions
behave like $\frac{1}{x}$ around their vertical asymptote, and the 
curve of each function will have $2$ branches.

Katie has been working with $3$ functions that have the form given 
in \cref{rat:eq:deducecurve}, and has followed \crefrange{rat:step:first}{rat:step:penultimate};
her results are shown in \cref{rat:fig:deducecurve}. There is just one
more thing to do to complete the graphs| follow \cref{rat:step:last}.
Help Katie finish each graph by deducing the curve of each function.
\begin{shortsolution}
	\Vref{rat:fig:deducecurve1}
	
	\begin{tikzpicture}[/pgf/declare function={f=3*(x+4)/(x+5);}]
		\begin{axis}[
		   xmin=-10,xmax=10,
		   ymin=-10,ymax=10,
		   xtick={-8,-6,...,8},
		   width=\solutionfigurewidth,
		   ]
		   \addplot[soldot] coordinates{(-4,0)(0,12/5)};
		   \addplot[asymptote,domain=-10:10]({-5},{x});
		   \addplot[asymptote,domain=-10:10]({x},{3});
		   \addplot[pccplot] expression[domain=-10:-5.42857]{f};
		   \addplot[pccplot] expression[domain=-4.76923:10,samples=50]{f};
		\end{axis}
	\end{tikzpicture}
	
	\Vref{rat:fig:deducecurve2}
	
	\begin{tikzpicture}[/pgf/declare function={f=-3*(x-2)/(x-4);}]
		\begin{axis}[
		   xmin=-10,xmax=10,
		   ymin=-10,ymax=10,
		   xtick={-8,-6,...,8},
		   width=\solutionfigurewidth,
		   ]
		   \addplot[soldot] coordinates{(2,0)(0,-3/2)};
		   \addplot[asymptote,domain=-10:10]({4},{x});
		   \addplot[asymptote,domain=-10:10]({x},{-3});
		   \addplot[pccplot] expression[domain=-10:3.53846,samples=50]{f};
		   \addplot[pccplot] expression[domain=4.85714:10]{f};
		\end{axis}
	\end{tikzpicture}
	
	\Vref{rat:fig:deducecurve4}
	
	\begin{tikzpicture}[/pgf/declare function={f=2*(x-6)/(x-4);}]
		\begin{axis}[
		   xmin=-10,xmax=10,
		   ymin=-10,ymax=10,
		   xtick={-8,-6,...,8},
		   width=\solutionfigurewidth,
		   ]
		   \addplot[soldot] coordinates{(6,0)(0,3)};
		   \addplot[asymptote,domain=-10:10]({x},{2});
		   \addplot[asymptote,domain=-10:10]({4},{x});
		   \addplot[pccplot] expression[domain=-10:3.5,samples=50]{f};
		   \addplot[pccplot] expression[domain=4.3333:10]{f};
		\end{axis}
	\end{tikzpicture}
\end{shortsolution}
\end{problem}

\begin{figure}[!htb]
	\begin{widepage}
	\setlength{\figurewidth}{0.3\textwidth}
	\begin{subfigure}{\figurewidth}
		\begin{tikzpicture}
			\begin{axis}[
			   xmin=-10,xmax=10,
			   ymin=-10,ymax=10,
			   xtick={-8,-6,...,8},
			   width=\textwidth,
			   ]
			   \addplot[soldot] coordinates{(-4,0)(0,12/5)};
			   \addplot[asymptote,domain=-10:10]({-5},{x});
			   \addplot[asymptote,domain=-10:10]({x},{3});
			\end{axis}
		\end{tikzpicture}
		\caption{}
		\label{rat:fig:deducecurve1}
	\end{subfigure}%
	\hfill
	\begin{subfigure}{\figurewidth}
		\begin{tikzpicture}
			\begin{axis}[
			   xmin=-10,xmax=10,
			   ymin=-10,ymax=10,
			   xtick={-8,-6,...,8},
			   width=\textwidth,
			   ]
			   \addplot[soldot] coordinates{(2,0)(0,-3/2)};
			   \addplot[asymptote,domain=-10:10]({4},{x});
			   \addplot[asymptote,domain=-10:10]({x},{-3});
			\end{axis}
		\end{tikzpicture}
		\caption{}
		\label{rat:fig:deducecurve2}
	\end{subfigure}%
	\hfill
	\begin{subfigure}{\figurewidth}
		\begin{tikzpicture}
			\begin{axis}[
			   xmin=-10,xmax=10,
			   ymin=-10,ymax=10,
			   xtick={-8,-6,...,8},
			   width=\textwidth,
			   ]
			   \addplot[soldot] coordinates{(6,0)(0,3)};
			   \addplot[asymptote,domain=-10:10]({x},{2});
			   \addplot[asymptote,domain=-10:10]({4},{x});
			\end{axis}
		\end{tikzpicture}
		\caption{}
		\label{rat:fig:deducecurve4}
	\end{subfigure}
	\caption{Graphs for \cref{rat:prob:deduce}}
	\label{rat:fig:deducecurve}
	\end{widepage}
\end{figure}

%===================================
%   Author: Hughes
%   Date:   June 2012
%===================================
\begin{problem}[\Cref{rat:step:last} for  more complicated rational functions]\label{rat:prob:deducehard}
\pccname{David} is also working on graphing rational functions, and 
has been concentrating on functions that have the form
\[
	r(x)=\frac{a(x-b)(x-c)}{(x-d)(x-e)}
\]
David notices that functions with this type of formula have simple zeros 
at $b$ and $c$, and vertical asymptotes at $d$ and $e$. Furthermore, 
these functions behave like $\frac{1}{x}$ around both vertical asymptotes,
and the curve of the function will have $3$ branches. 

David has followed \crefrange{rat:step:first}{rat:step:penultimate} for 
$3$ separate functions, and drawn the results in \cref{rat:fig:deducehard}.
Help David finish each graph by deducing the curve of each function.
\begin{shortsolution}
	\Vref{rat:fig:deducehard1}
	
	\begin{tikzpicture}[/pgf/declare function={f=(x-6)*(x+3)/( (x-4)*(x+1));}]
		\begin{axis}[
		   xmin=-10,xmax=10,
		   ymin=-10,ymax=10,
		   xtick={-8,-6,...,8},
		   width=\solutionfigurewidth,
		   ]
		   \addplot[soldot] coordinates{(-3,0)(6,0)(0,9/2)};
		   \addplot[asymptote,domain=-10:10]({-1},{x});
		   \addplot[asymptote,domain=-10:10]({4},{x});
		   \addplot[asymptote,domain=-10:10]({x},{2});
		   \addplot[pccplot] expression[domain=-10:-1.24276]{f};
		   \addplot[pccplot] expression[domain=-0.6666:3.66667]{f};
		   \addplot[pccplot] expression[domain=4.24276:10]{f};
		\end{axis}
	\end{tikzpicture}
	
	\Vref{rat:fig:deducehard2}
	
	\begin{tikzpicture}[/pgf/declare function={f=3*(x-2)*(x+3)/( (x-6)*(x+5));}]
		\begin{axis}[
		   xmin=-10,xmax=10,
		   ymin=-10,ymax=10,
		   xtick={-8,-6,...,8},
		   width=\solutionfigurewidth,
		   ]
		   \addplot[soldot] coordinates{(-3,0)(2,0)(0,3/5)};
		   \addplot[asymptote,domain=-10:10]({-5},{x});
		   \addplot[asymptote,domain=-10:10]({6},{x});
		   \addplot[asymptote,domain=-10:10]({x},{3});
		   \addplot[pccplot] expression[domain=-10:-5.4861]{f};
		   \addplot[pccplot] expression[domain=-4.68395:5.22241]{f};
		   \addplot[pccplot] expression[domain=7.34324:10]{f};
		\end{axis}
	\end{tikzpicture}
	
	\Vref{rat:fig:deducehard3}
	
	\begin{tikzpicture}[/pgf/declare function={f=2*(x-7)*(x+3)/( (x+6)*(x-5));}]
		\begin{axis}[
		   xmin=-10,xmax=10,
		   ymin=-10,ymax=10,
		   xtick={-8,-6,...,8},
		   width=\solutionfigurewidth,
		   ]
		   \addplot[soldot] coordinates{(-3,0)(7,0)(0,1.4)};
		   \addplot[asymptote,domain=-10:10]({-6},{x});
		   \addplot[asymptote,domain=-10:10]({5},{x});
		   \addplot[asymptote,domain=-10:10]({x},{2});
		   \addplot[pccplot] expression[domain=-10:-6.91427]{f};
		   \addplot[pccplot] expression[domain=-5.42252:4.66427]{f};
		   \addplot[pccplot] expression[domain=5.25586:10]{f};
		\end{axis}
	\end{tikzpicture}
	
\end{shortsolution}
\end{problem}

\begin{figure}[!htb]
	\begin{widepage}
	\setlength{\figurewidth}{0.3\textwidth}
	\begin{subfigure}{\figurewidth}
		\begin{tikzpicture}
			\begin{axis}[
			   xmin=-10,xmax=10,
			   ymin=-10,ymax=10,
			   xtick={-8,-6,...,8},
			   width=\textwidth,
			   ]
			   \addplot[soldot] coordinates{(-3,0)(6,0)(0,9/2)};
			   \addplot[asymptote,domain=-10:10]({-1},{x});
			   \addplot[asymptote,domain=-10:10]({4},{x});
			   \addplot[asymptote,domain=-10:10]({x},{2});
			\end{axis}
		\end{tikzpicture}
		\caption{}
		\label{rat:fig:deducehard1}
	\end{subfigure}%
	\hfill
	\begin{subfigure}{\figurewidth}
		\begin{tikzpicture}
			\begin{axis}[
			   xmin=-10,xmax=10,
			   ymin=-10,ymax=10,
			   xtick={-8,-6,...,8},
			   width=\textwidth,
			   ]
			   \addplot[soldot] coordinates{(-3,0)(2,0)(0,3/5)};
			   \addplot[asymptote,domain=-10:10]({-5},{x});
			   \addplot[asymptote,domain=-10:10]({6},{x});
			   \addplot[asymptote,domain=-10:10]({x},{3});
			\end{axis}
		\end{tikzpicture}
		\caption{}
		\label{rat:fig:deducehard2}
	\end{subfigure}%
	\hfill
	\begin{subfigure}{\figurewidth}
		\begin{tikzpicture}
			\begin{axis}[
			   xmin=-10,xmax=10,
			   ymin=-10,ymax=10,
			   xtick={-8,-6,...,8},
			   width=\textwidth,
			   ]
			   \addplot[soldot] coordinates{(-3,0)(7,0)(0,1.4)};
			   \addplot[asymptote,domain=-10:10]({-6},{x});
			   \addplot[asymptote,domain=-10:10]({5},{x});
			   \addplot[asymptote,domain=-10:10]({x},{2});
			\end{axis}
		\end{tikzpicture}
		\caption{}
		\label{rat:fig:deducehard3}
	\end{subfigure}%
	\hfill
	\caption{Graphs for \cref{rat:prob:deducehard}}
	\label{rat:fig:deducehard}
	\end{widepage}
\end{figure}
%===================================
%   Author: Adams (Hughes)
%   Date:   March 2012
%===================================
\begin{problem}[\Crefrange{rat:step:first}{rat:step:last}]
Use \crefrange{rat:step:first}{rat:step:last} to sketch a graph of 
each of the following functions
\fixthis{need 2 more subproblems here}
\begin{multicols}{4}
	\begin{subproblem}
		$y=\dfrac{4}{x+2}$ 
		\begin{shortsolution}
			Vertical intercept: $(0,2)$; vertical asymptote: $x=-2$, horizontal asymptote: $y=0$.
			
			\begin{tikzpicture}
				\begin{axis}[
				   framed,
				   xmin=-5,xmax=5,
				   ymin=-5,ymax=5,
				   grid=both,
				   width=\solutionfigurewidth,
				   ]
				   \addplot[pccplot] expression[domain=-5:-2.8]{4/(x+2)};
				   \addplot[pccplot] expression[domain=-1.2:5]{4/(x+2)};
				   \addplot[soldot]coordinates{(0,2)};
				   \addplot[asymptote,domain=-5:5]({-2},{x});
				   \addplot[asymptote,domain=-5:5]({x},{0});
				\end{axis}
			\end{tikzpicture}
		\end{shortsolution}
	\end{subproblem}
	\begin{subproblem}
		$y=\dfrac{2x-1}{x^2-9}$ 
		\begin{shortsolution}
			Vertical intercept:$\left( 0,\frac{1}{9} \right)$; 
			horizontal intercept: $\left( \frac{1}{2},0 \right)$;
			vertical asymptotes: $x=-3$, $x=3$, horizontal asymptote: $y=0$.
			
			\begin{tikzpicture}
				\begin{axis}[
				   framed,
				   xmin=-5,xmax=5,
				   ymin=-5,ymax=5,
				   grid=both,
				   width=\solutionfigurewidth,
				   ]
				   \addplot[pccplot] expression[domain=-5:-3.23974]{(2*x-1)/(x^2-9)};
				   \addplot[pccplot,samples=50] expression[domain=-2.77321:2.83974]{(2*x-1)/(x^2-9)};
				   \addplot[pccplot] expression[domain=3.17321:5]{(2*x-1)/(x^2-9)};
				   \addplot[soldot]coordinates{(0,1/9)(1/2,0)};
				   \addplot[asymptote,domain=-5:5]({-3},{x});
				   \addplot[asymptote,domain=-5:5]({3},{x});
				   \addplot[asymptote,domain=-5:5]({x},{0});
				\end{axis}
			\end{tikzpicture}
		\end{shortsolution}
	\end{subproblem}
	\begin{subproblem}
		$y=\dfrac{x+3}{x-5}$ 
		\begin{shortsolution}
			Vertical intercept $\left( 0,-\frac{3}{5} \right)$; horizontal
			intercept: $(-3,0)$; vertical asymptote: $x=5$; horizontal asymptote: $y=1$.
			
			\begin{tikzpicture}
				\begin{axis}[
				   framed,
				   xmin=-10,xmax=10,
				   ymin=-5,ymax=5,
				   xtick={-8,-6,...,8},
				   minor ytick={-3,-1,...,3},
				   grid=both,
				   width=\solutionfigurewidth,
				   ]
				   \addplot[pccplot] expression[domain=-10:3.666]{(x+3)/(x-5)};
				   \addplot[pccplot] expression[domain=7:10]{(x+3)/(x-5)};
				   \addplot[asymptote,domain=-5:5]({5},{x});
				   \addplot[asymptote,domain=-10:10]({x},{1});
				   \addplot[soldot]coordinates{(0,-3/5)(-3,0)};
				\end{axis}
			\end{tikzpicture}
		\end{shortsolution}
	\end{subproblem}
	\begin{subproblem}
		$y=\dfrac{2x+3}{3x-1}$ 
		\begin{shortsolution}
			Vertical intercept: $(0,-3)$; horizontal intercept: $\left( -\frac{3}{2},0 \right)$;
			vertical asymptote: $x=\frac{1}{3}$, horizontal asymptote: $y=\frac{2}{3}$.
			
			\begin{tikzpicture}[/pgf/declare function={f=(2*x+3)/(3*x-1);}]
				\begin{axis}[
				   framed,
				   xmin=-5,xmax=5,
				   ymin=-5,ymax=5,
				   grid=both,
				   width=\solutionfigurewidth,
				   ]
				   \addplot[pccplot] expression[domain=-5:0.1176]{f};
				   \addplot[pccplot] expression[domain=0.6153:5]{f};
				   \addplot[asymptote,domain=-5:5]({1/3},{x});
				   \addplot[asymptote,domain=-5:5]({x},{2/3});
				   \addplot[soldot]coordinates{(0,-3)(-3/2,0)};
				\end{axis}
			\end{tikzpicture}
		\end{shortsolution}
	\end{subproblem}
	\begin{subproblem}
		$y=\dfrac{4-x^2}{x^2-9}$ 
		\begin{shortsolution}
			Vertical intercept: $\left( 0,-\frac{4}{9} \right)$;
			horizontal intercepts: $(2,0)$, $(-2,0)$; 
			vertical asymptotes: $x=-3$, $x=3$; horizontal asymptote: $y=-1$.
			
			\begin{tikzpicture}[/pgf/declare function={f=(4-x^2)/(x^2-9);}]
				\begin{axis}[
				   framed,
				   xmin=-5,xmax=5,
				   ymin=-5,ymax=5,
				   grid=both,
				   width=\solutionfigurewidth,
				   ]
				   \addplot[pccplot] expression[domain=-5:-3.20156]{f};
				   \addplot[pccplot,samples=50] expression[domain=-2.85774:2.85774]{f};
				   \addplot[pccplot] expression[domain=3.20156:5]{f};
				   \addplot[asymptote,domain=-5:5]({-3},{x});
				   \addplot[asymptote,domain=-5:5]({3},{x});
				   \addplot[asymptote,domain=-5:5]({x},{-1});
				   \addplot[soldot] coordinates{(-2,0)(2,0)(0,-4/9)};
				\end{axis}
			\end{tikzpicture}
		\end{shortsolution}
	\end{subproblem}
	\begin{subproblem}
		$y=\dfrac{(4x+5)(3x-4)}{(2x+5)(x-5)}$ 
		\begin{shortsolution}
			Vertical intercept: $\left( 0,\frac{4}{5} \right)$; 
			horizontal intercepts: $\left( -\frac{5}{4},0 \right)$, $\left( \frac{4}{3},0 \right)$; 
			vertical asymptotes: $x=-\frac{5}{2}$, $x=5$; horizontal asymptote: $y=6$.
			
			\begin{tikzpicture}[/pgf/declare function={f=(4*x+5)*(3*x-4)/((2*x+5)*(x-5));}]
				\begin{axis}[
				   framed,
				   xmin=-10,xmax=10,
				   ymin=-20,ymax=20,
				   xtick={-8,-6,...,8},
				   ytick={-10,0,...,10},
				   minor ytick={-15,-5,...,15},
				   grid=both,
				   width=\solutionfigurewidth,
				   ]
				   \addplot[pccplot] expression[domain=-10:-2.73416]{f};
				   \addplot[pccplot] expression[domain=-2.33689:4.2792]{f};
				   \addplot[pccplot] expression[domain=6.26988:10]{f};
				   \addplot[asymptote,domain=-20:20]({-5/2},{x});
				   \addplot[asymptote,domain=-20:20]({5},{x});
				   \addplot[asymptote,domain=-10:10]({x},{6});
				   \addplot[soldot]coordinates{(0,4/5)(-5/4,0)(4/3,0)};
				\end{axis}
			\end{tikzpicture}
		\end{shortsolution}
	\end{subproblem}
\end{multicols}
\end{problem}
%===================================
%   Author: Hughes
%   Date:   March 2012
%===================================
\begin{problem}[Inverse functions]
Each of the following rational functions are invertible 
\[
	F(x)=\frac{2x+1}{x-3}, \qquad G(x)= \frac{1-4x}{x+3}
\]
\begin{subproblem}
	State the domain of each function. 
	\begin{shortsolution}
		\begin{itemize}
			\item The domain of $F$ is $(-\infty,3)\cup(3,\infty)$.
			\item The domain of $G$ is $(-\infty,-3)\cup(-3,\infty)$.
		\end{itemize}
	\end{shortsolution}
\end{subproblem}
\begin{subproblem}
	Find the inverse of each function, and state its domain.
	\begin{shortsolution}
		\begin{itemize}
			\item $F^{-1}(x)=\frac{3x+1}{x-2}$; the domain of $F^{-1}$ is $(-\infty,2)\cup(2,\infty)$.
			\item $G^{-1}(x)=\frac{3x+1}{x+4}$; the domain of $G^{-1}$ is $(-\infty,-4)\cup(-4,\infty)$.
		\end{itemize}
	\end{shortsolution}
\end{subproblem}
\begin{subproblem}
	Hence state the range of the original functions.
	\begin{shortsolution}
		\begin{itemize}
			\item The range of $F$ is the domain of $F^{-1}$, which is $(-\infty,2)\cup(2,\infty)$.
			\item The range of $G$ is the domain of $G^{-1}$, which is $(-\infty,-4)\cup(-4,\infty)$.
		\end{itemize}
	\end{shortsolution}
\end{subproblem}
\begin{subproblem}
	State the range of each inverse function. 
	\begin{shortsolution}
		\begin{itemize}
			\item The range of $F^{-1}$ is the domain of $F$, which is $(-\infty,3)\cup(3,\infty)$.
			\item The range of $G^{-1}$ is the domain of $G$, which is $(-\infty,-3)\cup(-3,\infty)$.
		\end{itemize}<++>
	\end{shortsolution}
\end{subproblem}
\end{problem}
%===================================
%   Author: Hughes
%   Date:   March 2012
%===================================
\begin{problem}[Composition]
Let $r$ and $s$ be the rational functions that have formulas
\[
	r(x)=\frac{3}{x^2},\qquad s(x)=\frac{4-x}{x+5}
\]
Evaluate each of the following.
\begin{multicols}{3}
	\begin{subproblem}
		$(r\circ s)(0)$
		\begin{shortsolution}
			$\frac{75}{16}$ 
		\end{shortsolution}
	\end{subproblem}
	\begin{subproblem}
		$(s\circ r)(0)$ 
		\begin{shortsolution}
			$(s\circ r)(0)$ is undefined.
		\end{shortsolution}
	\end{subproblem}
	\begin{subproblem}
		$(r\circ s)(2)$ 
		\begin{shortsolution}
			$\frac{147}{4}$ 
		\end{shortsolution}
	\end{subproblem}
	\begin{subproblem}
		$(s\circ r)(3)$ 
		\begin{shortsolution}
			$192$ 
		\end{shortsolution}
	\end{subproblem}
	\begin{subproblem}
		$(s\circ r)(4)$ 
		\begin{shortsolution}
			$(s\circ r)(4)$ is undefined.
		\end{shortsolution}
	\end{subproblem}
	\begin{subproblem}
		$(s\circ r)(x)$ 
		\begin{shortsolution}
			$\dfrac{4x^2-3}{1+5x^2}$
		\end{shortsolution}
	\end{subproblem}
\end{multicols}
\end{problem}
%===================================
%   Author: Hughes
%   Date:   March 2012
%===================================
\begin{problem}[Piecewise rational functions]
The function $R$ has formula
\[
	R(x)=
	\begin{dcases}
		\frac{2}{x+3},    & x<-5     \\
		\frac{x-4}{x-10}, & x\geq -5 
	\end{dcases}
\]
Evaluate each of the following.
\begin{multicols}{4}
	\begin{subproblem}
		$R(-6)$    
		\begin{shortsolution}
			$-\frac{2}{3}$ 
		\end{shortsolution}
	\end{subproblem}
	\begin{subproblem}
		$R(-5)$ 
		\begin{shortsolution}
			$\frac{3}{5}$ 
		\end{shortsolution}
	\end{subproblem}
	\begin{subproblem}
		$R(-3)$ 
		\begin{shortsolution}
			$\frac{7}{13}$ 
		\end{shortsolution}
	\end{subproblem}
	\begin{subproblem}
		$R(5)$ 
		\begin{shortsolution}
			$-\frac{1}{5}$ 
		\end{shortsolution}
	\end{subproblem}
\end{multicols}
\begin{subproblem}
	What is the domain of $R$? 
	\begin{shortsolution}
		$(-\infty,10)\cup(10,\infty)$ 
	\end{shortsolution}
\end{subproblem}
\end{problem}
\end{exercises}

\section{Graphing rational functions (oblique asymptotes)}\label{rat:sec:oblique}
\begin{subproblem}
	$y=\dfrac{x^2+1}{x-4}$ 
	\begin{shortsolution}
		\begin{enumerate}
			\item $\left( 0,-\frac{1}{4} \right)$
			\item Vertical asymptote: $x=4$.
			\item A graph of the function is shown below
			
			\begin{tikzpicture}[/pgf/declare function={f=(x^2+1)/(x-4);}]
				\begin{axis}[
				   framed,
				   xmin=-20,xmax=20,
				   ymin=-30,ymax=30,
				   xtick={-10,10},
				   minor xtick={-15,-5,...,15},
				   minor ytick={-10,10},
				   grid=both,
				   width=\solutionfigurewidth,
				   ]
				   \addplot[pccplot,samples=50] expression[domain=-20:3.54724]{f};
				   \addplot[pccplot,samples=50] expression[domain=4.80196:20]{f};
				   \addplot[asymptote,domain=-30:30]({4},{x});
				\end{axis}
			\end{tikzpicture}
		\end{enumerate}
	\end{shortsolution}
\end{subproblem}
\begin{subproblem}
	$y=\dfrac{x^3(x+3)}{x-5}$ 
	\begin{shortsolution}
		\begin{enumerate}
			\item $(0,0)$, $(-3,0)$
			\item Vertical asymptote: $x=5$, horizontal asymptote: none.
			\item A graph of the function is shown below
			
			\begin{tikzpicture}[/pgf/declare function={f=x^3*(x+3)/(x-5);}]
				\begin{axis}[
				   framed,
				   xmin=-10,xmax=10,
				   ymin=-500,ymax=2500,
				   xtick={-8,-6,...,8},
				   ytick={500,1000,1500,2000},
				   grid=both,
				   width=\solutionfigurewidth,
				   ]
				   \addplot[pccplot,samples=50] expression[domain=-10:4]{f};
				   \addplot[pccplot] expression[domain=5.6068:9.777]{f};
				   \addplot[asymptote,domain=-500:2500]({5},{x});
				\end{axis}
			\end{tikzpicture}
		\end{enumerate}
	\end{shortsolution}
\end{subproblem}
