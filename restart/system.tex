
\subsection{State, control, and governing equations\label{sec:State,-control,-and}}

The rest of the article focuses on controlling systems of the form
in Algorithm~\ref{algo:god-alg} in which some parts of the state
can be controlled directly (for example, in the form of boundary control).
We wish to solve the system in Algorithm~\ref{algo:god-alg} $T$
time-steps forward, i.e. we wish to determine the discrete state values
$\discrete{\link}{\tind}$ for all links $\link\in\links$ and all
time-steps $\tind\in\intrange 0{\ntime-1}$. Furthermore, at each
time-step $\tind$, we assume a set of ``control'' variables $\tuple{\condiscrete 1{\tind}}{\condiscrete{\ncontrols}{\tind}}\in\mathbb{R}^{\ncontrols}$
that influence the solution of the Riemann problems at junctions,
where $\ncontrols$ is the number of controlled values at each time-step,
and each control may be updated at each time-step. We assume that
a control may only influence a subset of junctions, which is a reasonable
assumption if the controls have some spatial locality. Thus, for a
junction $\jn\in\jns$, we assume without loss of generality that
a subset of the control parameters $\tuple{\condiscrete{\cind_{\jn}^{1}}{\tind}}{\condiscrete{\cind_{\jn}^{\ncontrols_{\jn}}}{\tind}}\in\mathbb{R}^{\ncontrols_{\jn}}$
influence the solution of the Riemann solver. Similar to the notation
developed for state variables, for control variables, we define $\junccon{\jn}{\tind}\defeq\tuple{\condiscrete{\cind_{\jn}^{1}}{\tind}}{\condiscrete{\cind_{\jn}^{\ncontrols_{\jn}}}{\tind}}$
as the concatenation of the control variables around the junction
$\jn$. To account for the addition of controls, we modify the Riemann
problem at a junction $\jn\in\jns$ at time-step $\tind$ to be a
function of the current state of connecting links $\juncstate{\jn}{\tind}$,
and the current control parameters $\junccon{\jn}{\tind}$. Then using
the same notation as before, we express the Riemann solver as:

\begin{eqnarray*}
\RS_{\jn}: & \mathbb{R}^{\ninc_{\jn}+\nout_{\jn}}\times\mathbb{R}^{\ncontrols_{\jn}} & \rightarrow\mathbb{R}^{\ninc_{\jn}+\nout_{\jn}}\\
 & \left(\juncstate{\jn}{\tind},\junccon{\jn}{\tind}\right) & \mapsto\RS_{\jn}\left(\juncstate{\jn}{\tind},\junccon{\jn}{\tind}\right)=\junctrace{\jn}{\tind}
\end{eqnarray*}


We represent the entire state of the solved system with the vector
$\state\in\mathbb{R}^{\nlinks\ntime}$, where for $\link\in\links$
and $k\in\intrange 0{\ntime-1}$, we have $\state_{\nlinks k+\link}=\discrete{\link}{\tind}$.
Similarly, we represent the entire control vector by $\control\in\mathbb{R}^{\ncontrols\ntime}$,
where $\control_{\ncontrols\tind+\cind}=\condiscrete{\cind}{\tind}$.

For each state variable $\discrete{\link}{\tind}$, write the corresponding
update equation $\syseq_{\link}^{\tind}$:

\begin{eqnarray*}
\syseq_{\link}^{\tind}: & \mathbb{R}^{\nlinks\ntime}\times\mathbb{R}^{\ncontrols\ntime} & \rightarrow\mathbb{R}\\
 & \left(\state,\control\right) & \mapsto\syseq_{\link}^{\tind}\left(\state,\control\right)=0
\end{eqnarray*}


This takes the following form:

\begin{eqnarray}
h_{\link}^{0}\left(\state,\control\right)=\discrete{\link}0-\initdiscrete_{\link} & =0\label{eq:init-ge}\\
\syseq_{\link}^{\tind}\left(\state,\control\right)=\discrete{\link}{\tind}-\discrete{\link}{\tind-1}+\frac{\Delta t}{L_{\link}}f\left(\RS_{\jdown{\link}}\left(\juncstate{\jdown{\link}}{\tind-1},\junccon{\jdown{\link}}{\tind-1}\right)\right)_{\link}\nonumber \\
-\frac{\Delta t}{L_{\link}}f\left(\RS_{\jup{\link}}\left(\juncstate{\jup{\link}}{\tind-1},\junccon{\jup{\link}}{\tind-1}\right)\right)_{\link} & =0 & \forall k\in\intrange 2{\ntime-1},\label{eq:main-ge}
\end{eqnarray}
or in terms of the Godunov junction flux:

\begin{eqnarray}
\syseq_{\link}^{\tind}\left(\state,\control\right)= & \discrete{\link}{\tind}-\discrete{\link}{\tind-1} & +\dfrac{\Delta t}{\Delta x}\left(\god_{\jdown{\link}}\left(\juncstate{\jdown{\link}}{\tind},\junccon{\jdown{\link}}{\tind-1}\right)\right)_{\link}\nonumber \\
 &  & -\dfrac{\Delta t}{\Delta x}\left(\god_{\jup{\link}}\left(\juncstate{\jup{\link}}{\tind},\junccon{\jup{\link}}{\tind-1}\right)\right)_{\link}\label{eq:syseq-god}
\end{eqnarray}
for all links $\link\in\links$, where $\initdiscrete_{\link}$ is
the initial condition for link $\link$. Thus, we can construct a
system of $\nlinks T$ governing equations $H\left(\state,\control\right)=0$,
where the $h_{\link,k}$ is the equation in $H$ at index $\nlinks k+\link$,
identical to the ordering of the corresponding discrete state variable. 
