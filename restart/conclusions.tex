%!TEX root =restart.tex
\section{Conclusions\label{sec:Conclusions}}

This article has detailed a simple framework for finite-horizon optimal control methods on a network of scalar conservation laws derived from first discretizing the network via the Godunov method, then applying the discrete adjoint to this system. To tailor the framework to a specific application, one need only provide the partial derivatives of the Riemann solver at a network junction as well as the partial derivates of the objective. Furthermore, we show that for this class of problems, the sparsity pattern allows the problem to be implemented with only linear memory and linear computational complexity with respect to the number of state and control parameters. We demonstrate the scalability of the approach by implementing a coordinated ramp metering algorithm using the adjoint method and applying the algorithm to a realistic freeway network in California. The algorithm runs in a fraction of real-time and produces significant improvements over existing algorithms. Future work includes investigating distributed, coordinated control schemes over physical networks via the adjoint method to allow traffic control strategies to scale to regional-scale networks.