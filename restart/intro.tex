%!TEX root = restart.tex
\section{Introduction} % (fold)
\label{sec:introduction}

% section introduction (end)
Networks of scalar conservation law \textit{partial differential equations}~(PDEs) are an efficient framework for modeling physical phenomena, such as gas pipeline flow~\cite{Rothfarb1970}, supply chain~\cite{Brunnermeier1999}, water distribution~\cite{Nguyen}, or freeway traffic evolution~\cite{garavello2006traffic,work2010traffic}. Optimization and control of these networks is an active field of research~\cite{Gugat2005,Bayen2006,Kotsialos2004}. TODO: add more details about control of continuous and discrete PDE's. e.g. Miroslav, Demetriou, Fahroo ~\cite{Fahroo2000,Krstic2008}.

The adjoint method, as used in the optimization and estimation of PDE-constrained systems, can be applied in various ways depending on the framework of interest (PDE, discr. The continuous adjoint method~\cite{Jacquet2005,Gugat2005,Moin1994,Reuther1996} operates directly on the PDE and a so called adjoint PDE system, which when solved gives an explicit expression of the gradient of the underlying optimization problem at a particular point in the control space. Conversely, the discrete adjoint method~\cite{Giles2000,Gugat2005,Kotsialos2004} first discretizes a continuous-time PDE and then requires the solution of a set of linear equations to solve for the gradient. Finally, a third approach exists, which uses automatic differentiation techniques to automatically generate an adjoint solver from the numerical representation of the forward system~\cite{Muller2005,Giering1998}.

While the continuous adjoint formulation results in a compact formulation, better intuition into the system's sensitivities with respect to the objective, and well-posedness of the control's solution, it is often difficult or tedious to derive its solution by hand for each physical system and objective. Additionally, the continuous adjoint must eventually be discretized in order to produce numerical solutions for the optimization problem. Finally, the differentiation of the forward PDE is sometimes problematic due to the lack of regularity of the solution~\cite{work2010traffic,garavello2006traffic}. The discrete adjoint does not provide an explicit formulation of the gradient of the discretized optimization problem, \textbf{We need to include the pros of the method, but your last round of edits deleted the pro without suggesting a replacement. Previously it said, but its derivation is often simplier and formulaic}. Automatic differentiation techniques can simplify the repetetive steps of the discrete adjoint derivation, but sometimes at the cost of sub-optimal code implementations with respect to memory and CPU consumption~\cite{Giles}. A more-detailed analysis of the trade-offs associated with each method is given in~\cite{Giles}.

There exist many applications of the adjoint method for control, optimization and estimation of physical systems in engineering. Shape optimization of aircraft~\cite{Reuther1996,Giles1997,Moin1994} has applied the method effectively to reduce the computational cost in gradient methods associated with the large number of optimization parameters. The technique has also been applied in parameter identification of biological systems~\cite{Raffard2008}. State estimation problems can be phrased as optimal control problems by setting the unknown state variables as control parameters and penalizing errors in resulting state predictions from known values. This approach has been applied to such problems as open water state estimation~\cite{Castaings2006,Strub2009} and freeway traffic state estimation~\cite{Jacqueta}.

Since conservation laws may be nonlinear by nature and lead to nonconvex or nonlinear formulations, fewer efficient optimization techniques exist for the discretized version of these problems than for convex problems for example. One approach is to approximate the system with a ``linearized'' version in order to use very efficient linear programming techniques. In transportation, by relaxing the Godunov discretization scheme, the linearization approach was used in~\cite{gomes2006optimal} for optimal ramp metering, and in~\cite{ziliaskopoulos2000linear} for optimal route assignment. The ramp metering technique in~\cite{Muralidharana} uses an additional control parameter (variable speed limits) to mimic the linearized freeway dynamics. While the upside of these methods is reduced computational complexity and the guarantee of finding a globally optimal solution, the downside is that the model of the linearized physical system may greatly differ from the actual system to which the control policies would be applied.

Alternatively, nonlinear optimization techniques can be applied to the discretized system without any modification to the underlying dynamics. This approach leads to more expensive optimization algorithms, such as gradient descent, and no guarantee of finding a global optimum. The difficulty in this approach comes in the computation of the gradient, which, if using finite differences, requires a full forward-simulation for each perturbation of a control parameter. This approach is taken in~\cite{Ramon2013,Frejo2011} for computing decentralized ramp metering strategies, and is shown to be infeasible for realistic network sizes.

Ramp metering is a popular freeway control strategy, providing a means of dynamically controlling freeway throughput without directly impeding mainline flow or implementing complex tolling systems. While metering strategies have been developed using microscopic models~\cite{Ben-Akiva2003}, most strategies are based off macroscopic state parameters, such as vehicle density and the density's relation to speed~\cite{richards1956shock,lighthill1955kinematic,daganzo1995cell}. Reactive metering strategies~\cite{Papageorgiou1991,Papamichail,Kachroo2003} use feedback from freeway loop detectors to target a desired mainline density, while predictive metering strategies~\cite{Frejo2011,Kotsialos2004,gomes2006optimal,Chen1997} use a physical model with predicted boundary flow data to generate policies over a finite time horizon. Predictive methods are often embedded within a model predictive control loop to handle uncertainties in the boundary data and cumulative model errors~\cite{Muralidharana}.

This article develops a framework for efficient control of discretized conservation law PDE networks using the adjoint method~\cite{Giles2000,Pironneau1974} via Godunov discretization~\cite{godunov1959}, while detailing its application to coordinated ramp metering on freeway networks. We show how the complexity of the gradient computation in nonlinear optimal control problems can be greatly decreased by using the discrete adjoint method and exploiting the decoupling nature of the problem's network structure, leading to efficient gradient descent methods. After giving a general framework for computing the gradient over the class of scalar conservation law networks, we show that the system's partial derivatives have a sparsity structure resulting in gradient computation times linear in the number of state and control variables. The results are demonstrated by running a coordinated ramp metering strategy on a 19 mile freeway stretch in California in a fraction of real-time, while giving traffic performance superior to that of popular alternative metering algorithms.

The article is organized as follows. Section~\ref{sec:Preliminaries} gives an overview of scalar conservation law networks and their discretization via the Godunov method, while introducing the nonlinear, finite-horizon optimal control problem. Section~\ref{sec:Adjoint-method} details the adjoint method derivation for this class of problems and shows how it can be used to compute the gradient in linear time in the number of discrete state and control variables.  Section~\ref{sec:Applications-to-Optimal} shows how the adjoint method can be applied to the problem of optimal coordinated ramp metering, with numerical results on a real freeway network in California shown in Section~\ref{sec:Numerical-results-for}. Finally, some concluding remarks are given in Section~\ref{sec:Conclusions}.
