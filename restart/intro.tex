%!TEX root = restart.tex
\section{Introduction} % (fold)
\label{sec:introduction}

In  this paper, we propose a descrete adjoint approach to compute optimal ramp-metering stategies on road networks modeled by conservation laws.
Networks of one-dimensional conservation laws,
described by systems of nonlinear first-order hyperbolic \textit{partial
differential equations}~(PDEs), are an efficient framework for modeling
physical phenomena,  such as gas pipeline flow~\cite{Rothfarb1970}, supply
chain~\cite{Brunnermeier1999}, water channels~\cite{Rabbani2010,litrico2009boundary}, or freeway
traffic evolution~\cite{garavello2006traffic,work2010traffic,frazzoli2002real}. Optimization
and  control of these networks is an active field of
research~\cite{Gugat2005,Bayen2006,Kotsialos2004}. More generally, numerous
techniques exist for the control of conservation laws, such as, for example,
backstepping~\cite{Coron2013,Glass2007}, Lyapunov-based methods~\cite{Coron2013}, and
optimal control methods~\cite{Jacquet2006,Blanchard2013,Keller2013}.
In particular, a common  approach is to compute the gradient of the cost functional via the \textit{adjoint method} [{\it some refs??}].
Nevertheless, its implementation in the framework of nonlinear conservation laws presents several difficulties linked to the discontinuous  character of the solutions. In particular, the presence of shocks in the solutions requires a careful sensitivity analysis based on the use of shift-differentials, see [Bressan-Marson;  S. Ulbrich, A sensitivity and adjoint calculus for discontinuous solutions of hyperbolic
conservation laws with source terms, SIAM J. Control Optim., 41 (2002), pp. 740-797; S. Ulbrich, Adjoint-based derivative computations for the optimal control of discontinuous
solutions of hyperbolic conservation laws, Systems Control Lett., 48 (2003), pp. 313-328.]

Extensive study has also been conducted on the choice of the method for effectively computing the gradient. In particular, the continuous adjoint
method~\cite{Jacquet2005,Gugat2005,Moin1994,Reuther1996} operates directly on
the PDE and a so-called adjoint PDE system, which when solved can be used to
obtain an explicit  expression of the gradient of the underlying optimization
problem. Conversely,  the discrete adjoint
method~\cite{Giles2000,Gugat2005,Kotsialos2004} first discretizes a
continuous-time PDE and then requires the solution of a set of linear
equations  to solve for the gradient. Finally, a third approach exists, which
uses  automatic differentiation techniques to automatically generate an
adjoint  solver from the numerical representation of the forward
system~\cite{Muller2005,Giering1998}.

It is well known that the numerical treatment of the adjoint method imposes a careful choice of the scheme to avoid the introduction of numerical errors  at discontinuities [Giles, Discrete adjoint approximations with shocks].
Rigorous convergence results for optimization problems have been provided  for Lax-Friedichs type schemes [Giles-Ulbrich, SIAM 2012] and relaxation methods [Banda-Herty, COAP 2010].
The case of road networks in free flow conditions is addressed in [MR2164806].
In our more general setting, the presence of juction conditions modeling the onramp and offramp dynamics imposes the use of a modified Godunov scheme that precisely takes into account the flows at the network nodes. Moreover, general existence and stability results for the corresponding system of equations modeling traffic evolution on the network are still missing at the moment.
Therefore, establishing rigorous convergence results for the gradient computation in this framework is out of the scope of this paper. Here we made the choice of the discrete adjoint approach, which derives the gradient directly from the 
discretized system, thus avoiding working directly with weak boundary 
conditions in the continuous 
system~\cite{garavello2006traffic,work2010traffic,strub2006weak}.
 





%While the continuous adjoint formulation results in a compact formulation, 
%better intuition into the system's sensitivities with respect to the objective, 
%and well-posedness of the control's solution (when it can be proved), it is 
%often difficult to derive for systems of hyperbolic nonlinear PDEs controlled 
%by boundary conditions, when these boundary conditions have to be written in the 
%weak sense.
%Additionally, the continuous adjoint must eventually be discretized in order to 
%produce numerical solutions for the optimization problem. Finally, the 
%differentiation of the forward PDE is sometimes problematic due to the lack of 
%regularity of the solution~\cite{garavello2006traffic,work2010traffic} which 
%makes the formal definition of the adjoint problem more difficult.
%The discrete adjoint approach derives the gradient directly from the 
%discretized system, thus avoiding working directly with weak boundary 
%conditions in the continuous 
%system~\cite{garavello2006traffic,work2010traffic,strub2006weak}.
%Automatic differentiation techniques can simplify the repetitive 
%steps of the discrete adjoint derivation, but sometimes at the cost of 
%sub-optimal code implementations with respect to memory and CPU 
%consumption~\cite{Giles}. A more-detailed analysis of the trade-offs associated 
%with each method is given in~\cite{Giles}.

There exist many applications of the adjoint method for control, optimization 
and estimation of physical systems in engineering. Shape optimization of 
aircraft~\cite{Reuther1996,Giles1997,Moin1994} has applied the method 
effectively to reduce the computational cost in gradient methods associated 
with the large number of optimization parameters. The technique has also been 
applied in parameter identification of biological systems~\cite{Raffard2008}. 
State estimation problems can be phrased as optimal control problems by setting 
the unknown state variables as control parameters and penalizing errors in 
resulting state predictions from known values. This approach has been applied 
to such problems as open water state estimation~\cite{Castaings2006,Strub2009} 
and freeway traffic state estimation~\cite{Jacqueta}.

Since conservation laws may be nonlinear by nature and lead to non-convex or 
nonlinear formulations of the corresponding optimization problem, fewer 
efficient optimization techniques exist for the 
discretized version of these problems than for convex problems for example. One 
approach is to approximate the system with a ``relaxed'' version in order to 
use efficient linear programming techniques. In transportation, by 
relaxing the Godunov discretization scheme, the linearization approach was used 
in~\cite{gomes2006optimal} for optimal ramp metering, and 
in~\cite{ziliaskopoulos2000linear} for optimal route assignment which is exact 
when the relaxation gap can be shown to be zero. The ramp 
metering technique in~\cite{Muralidharana} uses an additional control parameter 
(variable speed limits) to mimic the linearized freeway dynamics. While the 
upside of these methods is reduced computational complexity and the guarantee 
of finding a globally optimal solution, the downside is that the model of the 
linearized physical system may greatly differ from the actual system to which 
the control policies would be applied.

Alternatively, nonlinear optimization techniques can be applied to the 
discretized system without any modification to the underlying dynamics. This 
approach leads to more expensive optimization algorithms, such as gradient 
descent, and no guarantee of finding a global optimum. One difficulty in this 
approach comes in the computation of the gradient, which, if using finite 
differences, requires a full forward-simulation for each perturbation of a 
control parameter. This approach is taken in~\cite{Ramon2013,Frejo2011} to 
compute several types of decentralized ramp metering strategies. The increased 
complexity of the finite differences approach for each additional control 
parameter makes the method unsuitable for real-time application on 
moderately-sized freeway networks.

Ramp metering is a common freeway control strategy, providing a means of 
dynamically controlling freeway throughput without directly impeding mainline 
flow or implementing complex tolling systems. While metering strategies have 
been developed using microscopic models~\cite{Ben-Akiva2003}, most strategies 
are based off macroscopic state parameters, such as vehicle density and the 
density's relation to 
speed~\cite{richards1956shock,lighthill1955kinematic,daganzo1995cell}. Reactive 
metering strategies~\cite{Papageorgiou1991,Papamichail,Kachroo2003} use 
feedback from freeway loop detectors to target a desired mainline density, 
while predictive metering 
strategies~\cite{Frejo2011,Kotsialos2004,gomes2006optimal,Chen1997} use a 
physical model with predicted boundary flow data to generate policies over a 
finite time horizon. Predictive methods are often embedded within a model 
predictive control loop to handle uncertainties in the boundary data and 
cumulative model errors~\cite{Muralidharana}.

This article develops a framework for efficient control of discretized 
conservation law PDE networks using the adjoint 
method~\cite{Giles2000,Pironneau1974} via Godunov 
discretization~\cite{godunov1959}, while detailing its application to 
coordinated ramp metering on freeway networks. Note that the method can be 
extended without significant difficulty to other numerical schemes commonly 
used to discretize hyperbolic PDEs. We show how the complexity of 
the gradient computation in nonlinear optimal control problems can be greatly 
decreased by using the discrete adjoint method and exploiting the decoupling 
nature of the problem's network structure, leading to efficient gradient 
computation methods. After giving a general framework for computing the gradient 
over the class of scalar conservation law networks, we show that the system's 
partial derivatives have a sparsity structure resulting in gradient computation 
times linear in the number of state and control variables for networks of small 
vertex degree. The results are 
demonstrated by running a coordinated ramp metering strategy on a 19 mile 
freeway stretch in California faster than real-time, while giving traffic 
performance superior to that of state of the art practitioners tools.

The rest of the article is organized as follows. 
Section~\ref{sec:Preliminaries} gives an 
overview of scalar conservation law networks and their discretization via the 
Godunov method, while introducing the nonlinear, finite-horizon optimal control 
problem. Section~\ref{sec:Adjoint-method} details the adjoint method derivation 
for this class of problems and shows how it can be used to compute the gradient 
in linear time in the number of discrete state and control variables.  
Section~\ref{sec:Applications-to-Optimal} shows how the adjoint method can be 
applied to the problem of optimal coordinated ramp metering, with numerical 
results on a real freeway network in California shown in 
Section~\ref{sec:Numerical-results-for}. Finally, some concluding remarks are 
given in Section~\ref{sec:Conclusions}.
