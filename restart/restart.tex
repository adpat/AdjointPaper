%!TEX root = restart.tex
\documentclass[english]{article}
\usepackage[T1]{fontenc}
\usepackage[latin9]{inputenc}
\usepackage{listings}
\usepackage{longtable}
\usepackage{float}
\usepackage{wrapfig}
\usepackage{amsthm}
\usepackage{amsmath}
\usepackage{amssymb}
\usepackage{graphicx}
\usepackage{esint}

\makeatletter

%%%%%%%%%%%%%%%%%%%%%%%%%%%%%% LyX specific LaTeX commands.
%% Because html converters don't know tabularnewline
\providecommand{\tabularnewline}{\\}
\floatstyle{ruled}
\newfloat{algorithm}{tbp}{loa}
\providecommand{\algorithmname}{Algorithm}
\floatname{algorithm}{\protect\algorithmname}

%%%%%%%%%%%%%%%%%%%%%%%%%%%%%% Textclass specific LaTeX commands.
\theoremstyle{plain}
\newtheorem{thm}{\protect\theoremname}
  \theoremstyle{definition}
  \newtheorem{defn}[thm]{\protect\definitionname}
  \theoremstyle{remark}
  \newtheorem{rem}[thm]{\protect\remarkname}
  \theoremstyle{remark}
  \newtheorem{note}[thm]{\protect\notename}
  \theoremstyle{plain}
  \newtheorem{prop}[thm]{\protect\propositionname}

%%%%%%%%%%%%%%%%%%%%%%%%%%%%%% User specified LaTeX commands.
\usepackage{fullpage}
\let\originalleft\left
\let\originalright\right
\renewcommand{\left}{\mathopen{}\mathclose\bgroup\originalleft}
\renewcommand{\right}{\aftergroup\egroup\originalright}

\@ifundefined{showcaptionsetup}{}{%
 \PassOptionsToPackage{caption=false}{subfig}}
\usepackage{subfig}
\makeatother

\usepackage{babel}
  \providecommand{\definitionname}{Definition}
  \providecommand{\notename}{Note}
  \providecommand{\propositionname}{Proposition}
  \providecommand{\remarkname}{Remark}
\providecommand{\theoremname}{Theorem}

\DeclareMathOperator{\sgn}{sgn}

\begin{document}

\title{Adjoint-based optimization on a network of discretized scalar conservation
law PDE's with applications to coordinated ramp metering}


\author{Jack}
\maketitle
\begin{abstract}
The adjoint method provides a computationally efficient means of calculating
the gradient for applications in constrained optimization. In this
article, we consider a network of scalar conservation law PDE's with
general topology, whose behavior is modified by a set of control parameters
in order to minimize some objective. After discretizing the PDE system
via the Godunov scheme, we detail the computation of the gradient
of the discretized system with respect to the control parameters and
show that the complexity of its computation scales linearly with the
number of discrete state variables for networks with small vertex
degree. The method is applied to solve the problem of coordinated
ramp metering on freeway networks and is shown to improve the performance
and running time over existing methods.
\end{abstract}
\input{mainCommands}
\global\long\def\actualrampflowsymbol{D}
\global\long\def\actualrampflow#1#2{\actualrampflowsymbol_{#1}\left(#2\right)}
\global\long\def\totalrampflowsymbol{\bar{D}}
\global\long\def\totalrampflow#1#2{\totalrampflowsymbol_{#1}\left(#2\right)}
\global\long\def\fluxsymbol{\gamma}
\global\long\def\flux#1#2{\fluxsymbol_{#1}\left(#2\right)}
\global\long\def\rampfluxsymbol{\gamma^{\text{r}}}
\global\long\def\rampflux#1#2{\rampfluxsymbol_{#1}\left(#2\right)}
\global\long\def\maxrampfluxsymbol{\gamma^{\text{r,\ensuremath{\max}}}}
\global\long\def\maxrampflux#1{\gamma_{#1}^{\text{r,\ensuremath{\max}}}}
\global\long\def\totalrampflowsymbol{\bar{D}}
\global\long\def\fluxsymbol{\gamma}
\global\long\def\virtualind#1{#1'}
\global\long\def\rsind#1{\hat{#1}}
\global\long\def\linkmlone{1}
\global\long\def\linkmltwo{2}
\global\long\def\linkon{\text{r}}
\global\long\def\linkoff{\text{o}}
\global\long\def\virtualmap#1{V\left(#1\right)}
\global\long\def\invvirtualmap#1{V^{-1}\left(#1\right)}
\global\long\def\limitingside{LS}
\global\long\def\supplylimited{SL}
\global\long\def\demandlimited{DL}
\global\long\def\negate#1{\bar{#1}}
\global\long\def\maxflux{\bar{\fluxsymbol}}
\global\long\def\feasibleflux{\Omega}
\global\long\def\rsrhomap#1{\rsind{\densitysymbol}\left(#1\right)}
\global\long\def\rsinrhomap#1{\rsind{\densitysymbol}_{\text{in}}\left(#1\right)}
\global\long\def\rsoutrhomap#1{\rsind{\densitysymbol}_{\text{out}}\left(#1\right)}
\global\long\def\fnflux#1{f\left(#1\right)}
\global\long\def\jsvirt#1{\virtualind{JS}\left(#1\right)}
\global\long\def\RS#1{RS\left(#1\right)}

\newcommand \trim[4]{
\renewcommand \PreviewBbAdjust{#1 #2 #3 #4}
}

\newcommand \scale{2}
\newcommand \bound[1]{\Big(1 - \frac{\maxR{#1}}{r} \Big)}
\newcommand \tick[2]{
(#2, #1) -- (#2, -#1)
}

\newcommand \smalltick[1]{\tick{0.01}{#1}}
\newcommand \largetick[1]{[line width=1pt]\tick{0.02}{#1}}
 
 %input

 %cells
\global\long\def\flowsymbol{f}
 \global\long\def\flowout#1#2{\flowsymbol_{#1}^{\text{out}}\left(#2\right)}
 \global\long\def\flowin#1#2{\flowsymbol_{#1}^{\text{in}}\left(#2\right)}
 \global\long\def\flowoutnt#1#2{\flowsymbol_{#1}^{\text{out}}}
 \global\long\def\flowinnt#1#2{\flowsymbol_{#1}^{\text{in}}}
 \global\long\def\flowoutmc#1#2#3{\flowsymbol_{#1,#3}^{\text{out}}\left(#2\right)}
 \global\long\def\flowinmc#1#2#3{\flowsymbol_{#1,#3}^{\text{in}}\left(#2\right)}
 \global\long\def\flowoutmcnt#1#2#3{\flowsymbol_{#1,#3}^{\text{out}}}
 \global\long\def\flowinmcnt#1#2#3{\flowsymbol_{#1,#3}^{\text{in}}}

 \global\long\def\inputfluxsymbol{D}
 \global\long\def\inputflux#1#2{\inputfluxsymbol_{#1}\left(#2\right)}
 \global\long\def\inputfluxmc#1#2#3{\inputfluxmc_{#1,#3}\left(#2\right)}

 \global\long\def\density#1#2{\densitysymbol_{#1}\left(#2\right)}
 \global\long\def\densitymc#1#2#3{\densitysymbol_{#1,#3}\left(#2\right)}
  \global\long\def\densitymcnt#1#2#3{\densitysymbol_{#1,#2}}

 \global\long\def\splitratiosymbol{\beta}
 \global\long\def\splitratio#1#2#3#4{\splitratiosymbol_{#1#2,#4}\left(#3\right)}
 \global\long\def\aggsplitratio#1#2#3{\splitratiosymbol_{#1#2}\left(#3\right)}
 \global\long\def\splitrationt#1#2#3#4{\splitratiosymbol_{#1#2,#3}}
 \global\long\def\aggsplitrationt#1#2#3{\splitratiosymbol_{#1#2}}

 \global\long\def\prioritysymbol{P}
 \global\long\def\priority#1{\prioritysymbol_{#1}}
 \global\long\def\prioritym#1#2{\prioritysymbol_{#1#2}}

\global\long\def\celldemandsymbol{\delta}
\global\long\def\celldemand#1#2{\celldemandsymbol_{#1}\left(#2\right)}
\global\long\def\celldemandnt#1#2{\celldemandsymbol_{#1}}

\global\long\def\cellsupplysymbol{\sigma}
 \global\long\def\cellsupply#1#2{\cellsupplysymbol_{#1}\left(#2\right)}
  \global\long\def\cellsupplynt#1#2{\cellsupplysymbol_{#1}}


 \global\long\def\boundarydemandsymbol{d}
 \global\long\def\boundarydemand#1#2{\boundarydemandsymbol_{#1}\left(#2\right)}
 \global\long\def\boundarydemandmc#1#2#3{\boundarydemandsymbol_{#1,#3}\left(#2\right)}
 \global\long\def\boundaryqueuesymbol{l}
 \global\long\def\boundaryqueue#1#2{\boundaryqueuesymbol_{#1}\left(#2\right)}
 \global\long\def\boundaryqueuemc#1#2#3{\boundaryqueuesymbol_{#1,#3}\left(#2\right)}
 \global\long\def\boundaryqueuemb#1#2#3{\boundaryqueuesymbol^#3_{#1}\left(#2\right)}
 \global\long\def\boundaryqueuemcmb#1#2#3#4{\boundaryqueuesymbol^#4_{#1,#3}\left(#2\right)}
 \global\long\def\boundaryqueueinitmc#1#2{\boundaryqueuesymbol_{#1,#2}\left(0\right)}
 \global\long\def\boundaryflowsymbol{r}
 \global\long\def\boundaryflow#1#2{\boundaryflowsymbol_{#1}\left(#2\right)}
 \global\long\def\boundaryflowmc#1#2#3{\boundaryflowsymbol_{#1,#3}\left(#2\right)}
 \global\long\def\boundaryflowmb#1#2#3{\boundaryflowsymbol^#3_{#1}\left(#2\right)}
 \global\long\def\boundaryflowmcmb#1#2#3#4{\boundaryflowsymbol^#4_{#1,#3}\left(#2\right)}
 
 %input
 \global\long\def\inputfluxsymbol{D}
 \global\long\def\inputflux#1#2{D_{#1}\left(#2\right)}
 \global\long\def\inputfluxmc#1#2#3{D_{{#1},{#3}}\left(#2\right)}

 \global\long\def\densitysymbol{\rho}
 \global\long\def\density#1#2{\densitysymbol_{#1}\left(#2\right)}
 \global\long\def\densitynt#1#2{\densitysymbol_{#1}}
 \global\long\def\densityinit#1{\densitysymbol_{#1}^{0}}
 \global\long\def\densityinitmc#1#2{\densitysymbol_{#1,#2}^{0}}

 \global\long\def\compliantratiosymbol{\alpha}
 \global\long\def\compliantratiomc#1{\alpha_#1}

 \global\long\def\commoditysymbol{c}
 \global\long\def\junctionsymbol{z}
 \global\long\def\constraintsymbol{x}

 \global\long\def\sourcesetsymbol{{\cal S}}
 \global\long\def\sinksetsymbol{{\cal T}}
 \global\long\def\junctionsetsymbol{{\cal J}}
 \global\long\def\linksetsymbol{{\cal A}}
 \global\long\def\commoditysetsymbol{C}


 \global\long\def\injunctionset#1{\junctionsetsymbol^{in}}
 \global\long\def\outjunctionset#1{\junctionsetsymbol^{out}}
 
 \global\long\def\H#1#2#3{H^{#1}_{#2, #3}}
 \global\long\def\Hmc#1#2#3#4{H^{#1}_{#2, #3, #4}}
 
 \global\long\def\I#1#2#3{I^{#1}_{#2, #3}}
 \global\long\def\Imc#1#2#3#4{I^{#1}_{#2, #3, #4}}




% !TEX root = rampMeteringViaTheAdjoint.tex

Constants
\begin{longtable}{ll}
$\deltat,\deltax$ & Time, space discretization size\tabularnewline
$\ffspeed{\icell}$ & Free flow speed on cell $\icell$\tabularnewline
$\congspeed{\icell}$ & Congestion wave speed on cell $\icell$\tabularnewline
$\jamdensity{\icell}$ & Jam density on cell $\icell$\tabularnewline
$\fmax{\icell}$ & Max flow leaving mainline of cell $\icell$\tabularnewline
$\priority{\icell}$ & Priority factor for cell $\icell$\tabularnewline
\end{longtable}

Inputs
\begin{longtable}{ll}
$\densityinit{\icell}$ & Initial density on cell $\icell$\tabularnewline
$\rampqueueinit{\icell}$ & Initial queue length for onramp entering cell $\icell$\tabularnewline
$\inputflux{\icell}{\itime}$ & Input flow on cell $\icell$, time step $\itime$\tabularnewline
$\offrampratio{\icell}{\itime}$ & Offramp split ratio on cell $\icell$, time step $\itime$\tabularnewline
\end{longtable}

Variables
\begin{longtable}{ll}
$\flowin{\icell}{\itime}$ & Flow into cell $\icell $ from mainline, time step $\itime$\tabularnewline
$\flowout{\icell}{\itime}$ & Flow out of cell $\icell$ onto mainline, time step $\itime$\tabularnewline
$\density{\icell}{\itime}$ & Density on cell $\icell$, time step $\itime$\tabularnewline
$\rampflow{\icell}{\itime}$ & Flow from onramp entering cell $\icell$, time step $\itime$\tabularnewline
$\rampqueue{\icell}{\itime}$ & Queue length for onramp entering cell $\icell$, time step $\itime$\tabularnewline
$\rampcontrol{\icell}{\itime}$ & Max flow from onramp entering cell $\icell$, time step $\itime$\tabularnewline
$\cellsupply{\icell}{\itime}$ & Supply on cell $\icell$, time step $\itime$\tabularnewline
$\celldemand{\icell}{\itime}$ & Demand on cell $\icell$, time step $\itime$\tabularnewline
$\rampdemand{\icell}{\itime}$ & Onramp demand entering cell $\icell$, time step $\itime$\tabularnewline
\end{longtable}%!TEX root = restart.tex
\section{Introduction} % (fold)
\label{sec:introduction}

% section introduction (end)
Networks of scalar conservation law \textit{partial differential 
equations}~(PDEs) are an efficient framework for modeling physical phenomena, 
such as gas pipeline flow~\cite{Rothfarb1970}, supply 
chain~\cite{Brunnermeier1999}, water distribution~\cite{Nguyen}, or freeway 
traffic evolution~\cite{garavello2006traffic,work2010traffic}. Optimization and 
control of these networks is an active field of 
research~\cite{Gugat2005,Bayen2006,Kotsialos2004}. More generally, many 
techniques for control and stabilization of distributed parameter systems have 
been studied, e.g. model reference adaptive control for system 
identification~\cite{Bohm1998}, backstepping method for stabilization of 
hyperbolic PDEs~\cite{Vazquez2011}, Lyapunov-based feedback laws for systems of 
conservation laws~\cite{Coron2007a}, methods for finite horizon optimal 
control~\cite{Fahroo2002}, and more specifically adjoint-based approaches to 
optimal control of continuous~\cite{Reuther1996} and discretized 
PDEs~\cite{Giles2000}.

The adjoint method, as used in the optimization and estimation of 
PDE-constrained systems, can be applied in various ways depending on the 
framework of interest (PDE, discretization of the PDE, or code implementing the 
discretization of the PDE. The continuous adjoint 
method~\cite{Jacquet2005,Gugat2005,Moin1994,Reuther1996} operates directly on 
the PDE and a so-called adjoint PDE system, which when solved can be used to 
obtain an explicit 
expression of the gradient of the underlying optimization problem. Conversely, 
the discrete adjoint 
method~\cite{Giles2000,Gugat2005,Kotsialos2004} first discretizes a 
continuous-time PDE and then requires the solution of a set of linear equations 
to solve for the gradient. Finally, a third approach exists, which uses 
automatic differentiation techniques to automatically generate an adjoint 
solver from the numerical representation of the forward 
system~\cite{Muller2005,Giering1998}.

While the continuous adjoint formulation results in a compact formulation, 
better intuition into the system's sensitivities with respect to the objective, 
and well-posedness of the control's solution (when it can be proved), it is 
often difficult to derive for systems of hyperbolic nonlinear PDEs controlled 
by boundary condition, when these boundary conditions have to be written in the 
weak sense.
Additionally, the continuous adjoint must eventually be discretized in order to 
produce numerical solutions for the optimization problem. Finally, the 
differentiation of the forward PDE is sometimes problematic due to the lack of 
regularity of the solution~\cite{work2010traffic,garavello2006traffic} which 
makes the formal definition of the adjoint problem more difficult.
The discrete adjoint approach derives the gradient directly from the 
discretized system, thus avoiding working directly with weak boundary 
conditions in the continuous system. Automatic differentiation techniques can 
simplify the repetetive 
steps of the discrete adjoint derivation, but sometimes at the cost of 
sub-optimal code implementations with respect to memory and CPU 
consumption~\cite{Giles}. A more-detailed analysis of the trade-offs associated 
with each method is given in~\cite{Giles}.

There exist many applications of the adjoint method for control, optimization 
and estimation of physical systems in engineering. Shape optimization of 
aircraft~\cite{Reuther1996,Giles1997,Moin1994} has applied the method 
effectively to reduce the computational cost in gradient methods associated 
with the large number of optimization parameters. The technique has also been 
applied in parameter identification of biological systems~\cite{Raffard2008}. 
State estimation problems can be phrased as optimal control problems by setting 
the unknown state variables as control parameters and penalizing errors in 
resulting state predictions from known values. This approach has been applied 
to such problems as open water state estimation~\cite{Castaings2006,Strub2009} 
and freeway traffic state estimation~\cite{Jacqueta}.

Since conservation laws may be nonlinear by nature and lead to nonconvex or 
nonlinear formulations of the corresponding optimization problem, fewer 
efficient optimization techniques exist for the 
discretized version of these problems than for convex problems for example. One 
approach is to approximate the system with a ``linearized'' version in order to 
use efficient linear programming techniques. In transportation, by 
relaxing the Godunov discretization scheme, the linearization approach was used 
in~\cite{gomes2006optimal} for optimal ramp metering, and 
in~\cite{ziliaskopoulos2000linear} for optimal route assignment which is exact 
when the relaxation gap can be shown to be zero. The ramp 
metering technique in~\cite{Muralidharana} uses an additional control parameter 
(variable speed limits) to mimic the linearized freeway dynamics. While the 
upside of these methods is reduced computational complexity and the guarantee 
of finding a globally optimal solution, the downside is that the model of the 
linearized physical system may greatly differ from the actual system to which 
the control policies would be applied.

Alternatively, nonlinear optimization techniques can be applied to the 
discretized system without any modification to the underlying dynamics. This 
approach leads to more expensive optimization algorithms, such as gradient 
descent, and no guarantee of finding a global optimum. The difficulty in this 
approach comes in the computation of the gradient, which, if using finite 
differences, requires a full forward-simulation for each perturbation of a 
control parameter. This approach is taken in~\cite{Ramon2013,Frejo2011} to 
compute several types of decentralized ramp metering strategies. The increased 
complexity of the finite differences approach for each additional control 
parameter makes the method unsuitable for real-time application on 
moderately-sized freeway networks.

Ramp metering is a common freeway control strategy, providing a means of 
dynamically controlling freeway throughput without directly impeding mainline 
flow or implementing complex tolling systems. While metering strategies have 
been developed using microscopic models~\cite{Ben-Akiva2003}, most strategies 
are based off macroscopic state parameters, such as vehicle density and the 
density's relation to 
speed~\cite{richards1956shock,lighthill1955kinematic,daganzo1995cell}. Reactive 
metering strategies~\cite{Papageorgiou1991,Papamichail,Kachroo2003} use 
feedback from freeway loop detectors to target a desired mainline density, 
while predictive metering 
strategies~\cite{Frejo2011,Kotsialos2004,gomes2006optimal,Chen1997} use a 
physical model with predicted boundary flow data to generate policies over a 
finite time horizon. Predictive methods are often embedded within a model 
predictive control loop to handle uncertainties in the boundary data and 
cumulative model errors~\cite{Muralidharana}.

The present article develops a framework for efficient control of discretized 
conservation law PDE networks using the adjoint 
method~\cite{Giles2000,Pironneau1974} via Godunov 
discretization~\cite{godunov1959}, while detailing its application to 
coordinated ramp metering on freeway networks. Note that the method can be 
extended without significant difficulty to other numerical schemes commonly 
used to discretize hyperbolic PDEs. We show how the complexity of 
the gradient computation in nonlinear optimal control problems can be greatly 
decreased by using the discrete adjoint method and exploiting the decoupling 
nature of the problem's network structure, leading to efficient gradient 
descent methods. After giving a general framework for computing the gradient 
over the class of scalar conservation law networks, we show that the system's 
partial derivatives have a sparsity structure resulting in gradient computation 
times linear in the number of state and control variables for networks of small 
vertex degree. The results are 
demonstrated by running a coordinated ramp metering strategy on a 19 mile 
freeway stretch in California faster than real-time, while giving traffic 
performance superior to that of state of the art practicioners tools.

The article is organized as follows. Section~\ref{sec:Preliminaries} gives an 
overview of scalar conservation law networks and their discretization via the 
Godunov method, while introducing the nonlinear, finite-horizon optimal control 
problem. Section~\ref{sec:Adjoint-method} details the adjoint method derivation 
for this class of problems and shows how it can be used to compute the gradient 
in linear time in the number of discrete state and control variables.  
Section~\ref{sec:Applications-to-Optimal} shows how the adjoint method can be 
applied to the problem of optimal coordinated ramp metering, with numerical 
results on a real freeway network in California shown in 
Section~\ref{sec:Numerical-results-for}. Finally, some concluding remarks are 
given in Section~\ref{sec:Conclusions}.
\input{prelims.tex}\input{system.tex}%!TEX root = restart.tex
\section{Adjoint Based Flow Optimization\label{sec:Adjoint-method}}


\subsection{Optimal Control Problem Formulation\label{par:Optimization-Problem}}

In addition to our governing equations $\sys\left(\state,\control\right)=0$, where we assume each $h_i^k \in \mathcal{C}^1$,
we also introduce a cost function $\cost \in \mathcal{C}^1$.

\begin{eqnarray*}
\cost: & \mathbb{R}^{\nlinks T}\times\mathbb{R}^{\ncontrols T} & \rightarrow\mathbb{R}\\
& \left(\state,\control\right) & \mapsto\cost\left(\state,\control\right)
\end{eqnarray*}
which returns a scalar that serves as a metric of performance of the
state and control values of the system. We wish to minimize the quantity
$\cost$ over the set of control parameters $\control$, while constraining
the state of the system to satisfy the governing equations $\sys\left(\state,\control\right)=0$,
which is, again, the concatenated version of~\eqref{eq:main-ge} or~\eqref{eq:syseq-god}.
We summarize this with the following optimization problem:

\begin{eqnarray}
\underset{\control}{\min} & \cost\left(\state,\control\right)\nonumber \\
\text{subject to:} & \sys\left(\state,\control\right)=0\label{eq:op-problem}
\end{eqnarray}
Both the cost function and governing equations may be non-convex in
this problem.


\subsection{Calculating the Gradient\label{par:Calculating-the-gradient}}

We wish to use gradient information in order to find control values
$\control^{*}$ that give locally optimal costs $\cost^{*}=\cost\left(\state\left(\control^{*}\right),\control^{*}\right)$.
Since there may exist many local minima for this optimization problem~\eqref{eq:op-problem}
(which is non-convex in general), gradient\emph{ }methods do not guarantee
global optimality of $\control^{*}$\emph{. }Still, nonlinear optimization
methods such as interior point optimization utilize gradient information
to improve performance~\cite{Andreas2005}.

In a descent algorithm, the optimization procedure will have to descend
a cost function, by coupling the gradient, which, at a nominal point
$\left(\nominal{\state},\nominal{\control}\right)$ is given by:

\begin{equation}
d_{\control}\cost\left(\nominal{\state},\nominal{\control}\right)=\evaluate{\pfrac{\cost\text{\ensuremath{\left(\state,\control\right)}}}{\state}}{\nominal{\state},\nominal{\control}}\Dfrac{\state}{\control}+\evaluate{\pfrac{\cost\text{\ensuremath{\left(\state,\control\right)}}}{\control}}{\nominal{\state},\nominal{\control}}\label{eq:j-v}.
\end{equation}

\begin{note}
For Equation~\eqref{eq:j-v} to be valid, all required partial and full derivatives must be well-defined, including $\Dfrac{\state}{\control}$. In some applications, this assumption does not necessarily hold,
either because $f$ itself is not smooth or because $\god$ is not
smooth (and thus $H \notin \mathcal{C}^1$), as is the case for the LWR equation with concave fundamental diagrams. There are several settings in which the
conditions for differentiability are satisfied, see in particular~\cite{Gugat2005,Flasskamp2012}.
\end{note}

The main difficulty is to compute the term $\Dfrac{\state}{\control}$.
We take advantage of the fact that the derivative of $H\left(\state,\control\right)$
with respect to $\control$ is equal to zero along trajectories of
the system:

\begin{equation}
d_{\control}\sys\left(\nominal{\state},\nominal{\control}\right)=\evaluate{\pfrac{\sys\text{\ensuremath{\left(\state,\control\right)}}}{\state}}{\nominal{\state},\nominal{\control}}\Dfrac{\state}{\control}+\evaluate{\pfrac{\sys\text{\ensuremath{\left(\state,\control\right)}}}{\control}}{\nominal{\state},\nominal{\control}}=0\label{eq:h-v}.
\end{equation}


The partial derivative terms, $\Hx\in\mathbb{R}^{\nlinks\ntime\times\nlinks\ntime}$,
$\Hu\in\mathbb{R}^{\nlinks\ntime\times\ncontrols\ntime}$, $\Jx\in\mathbb{R}^{\nlinks\ntime}$,
and $\Ju\in\mathbb{R}^{\ncontrols\ntime}$, can all be evaluated (more
details provided in Section~\ref{sub:Evaluating--and}) and then
treated as constant matrices. Thus, in order to evaluate $d_{\control}\cost\left(\nominal{\state},\nominal{\control}\right)\in\mathbb{R}^{\ncontrols\ntime}$,
we must solve a coupled system of matrix equations.

\paragraph{Forward system.\label{par:Forward-system}}

If we solve for $\Dfrac{\state}{\control}\in\mathbb{R}^{\nlinks\ntime\times\ncontrols\ntime}$
in~\eqref{eq:h-v}, which we call the \emph{forward system}:

\[
\Hx\Dfrac{\state}{\control}=-\Hu,
\]
then we can substitute the solved value for $\Dfrac{\state}{\control}$
into~\eqref{eq:j-v} to obtain the full expression for the gradient.
Section~\ref{sub:Evaluating--and} below gives details on the invertibility
of $\Hx$, guaranteeing a solution for $\Dfrac{\state}{\control}$.


\paragraph{Adjoint system.\label{par:Adjoint-system}}

Instead of evaluating $\Dfrac{\state}{\control}$ directly, the adjoint
method solves the following system, called the adjoint system,
for a new unknown variable $\lambda\in\mathbb{R}^{\nlinks\ntime}$
(called the adjoint variable):

\begin{equation}
\Hx^{T}\lambda=-\Jx^{T}\label{eq:adjoint}
\end{equation}

Under certain additional conditions on the flux function and discretization scheme, the adjoint system in Equation~\eqref{eq:adjoint} may be shown to converge to the continuous adjoint system as the discretization steps go towards zero, as described in the following works~\cite{Ulbrich2003AdjointBased,Banda2012Adjoint,Gugat2005}. No such convergence results exist in our setting of using a Godunov discretization with general $n\times m$ junctions.

The expression for the gradient becomes:

\begin{equation}
d_{\control}\cost\left(\nominal{\state},\nominal{\control}\right)=\lambda^{T}\Hu+\Ju\label{eq:adjoint-grad}
\end{equation}

We note that Equations~\eqref{eq:adjoint} and~\eqref{eq:adjoint-grad} can be alternatively derived using the first-order \emph{Karush-Kuhn-Tucker} (KKT) conditions, coupled with the constraint qualification in Equation~\eqref{eq:op-problem}. Given the assumed non-convexity of the underlying system, first-order KKT conditions are necessary, but not sufficient conditions for optimality of $\mathbf{\control}$ and $\lambda$. For practical applications to non-convex systems and for the purposes of this article, we do not necessarily seek global \emph{or local} optimality, but rather the direction of steepest descent given in Equation~\eqref{eq:adjoint-grad} in order to \emph{improve} the performance of the system.

We define $\degree{\state}$ to be the maximum junction degree on
the network:

\begin{equation}
\degree{\state}=\max_{\jn\in\jns}\left(\ninc_{\jn}+\nout_{\jn}\right),\label{eq:dx}
\end{equation}
and also define $\degree{\control}$ to be the maximum number of constraints
that a single control variable appears in, which is equivalent to:

\begin{equation}
\degree{\control}=\max_{\condiscrete{}{}\in\control}\sum_{\jn\in\jns:\condiscrete{}{}\in\junccon{\jn}{\tind}}\left(\ninc_{\jn}+\nout_{\jn}\right)\label{eq:dv}.
\end{equation}


Note that $\left\{ \convar\in\junccon{\jn}{\tind}:\jn\in\jns\right\} $
is a $\tind$-dependent set. By convention, junctions are either actuated
or not, so there is no dependency on $\tind$, i.e. if $\exists\tind$
s.t. $\convar\in\junccon{\jn}{\tind}$, then $\forall\tind$, $\convar\in\junccon{\jn}{\tind}$.

Using these definitions, we show later in Section~\ref{sub:Complexity-of-solving}
how the complexity of computing the gradient is reduced from $O(\degree{\state}\nlinks\ncontrols\ntime^{2})$
to $O(\ntime\left(\degree{\state}\nlinks+\degree{\control}\ncontrols\right))$
by considering the adjoint method over the forward method.

\begin{figure}
\begin{centering}
\subfloat[\label{fig:genneta}]{\begin{centering}
\includegraphics[width=0.33\columnwidth]{figs-gen/gen-net}
\par\end{centering}

}\subfloat[\label{fig:gennetb}]{\begin{centering}
\includegraphics[width=0.33\columnwidth]{figs-gen/gen-net-dx}
\par\end{centering}

}\subfloat[\label{fig:gennetc}]{\begin{centering}
\includegraphics[width=0.33\columnwidth]{figs-gen/gen-net-dv}
\par\end{centering}

}
\par\end{centering}

\caption{Depiction of $D_{\state}$ and $D_{v}$ for an arbitrary graph. Fig.~\ref{fig:genneta}
shows the underlying graphical structure for an arbitrary PDE network.
Some control parameter $\convar_{1}$ has influence over junctions
$A$, $B$, and $F$, while another control parameter $\convar_{2}$
has influence over only junction $C$. Fig.~\ref{fig:gennetb}
depicts the center junction having the largest number of connecting
edges, thus giving $D_{\state}=5$. Fig.~\ref{fig:gennetc} shows
that control parameter $\convar_{1}$ influences three junctions with
sum of junctions degrees equal to six, which is maximal over the other
control parameter $\convar_{2}$. leading to the result $D_{\control}=6$.
Note that in Fig.~\ref{fig:gennetc}, the link going from junction
$A$ to junction $B$ is counted twice: once as an outgoing link $\vec{AB}$
and once as in incoming link $\vec{BA}$.\label{fig:Depicting--and}}
\end{figure}

A graphical depiction of $D_{\state}$ and $D_{\control}$ are given
in Fig.~\ref{fig:Depicting--and}. Freeway networks are usually considered to have topologies that are
nearly planar, leading to junctions degrees which typically do not
exceed 3 or 4, regardless of the total number of links. Also, from
the locality argument for control variables in Section~(\ref{sec:State,-control,-and}),
a single control variable's influence over state variables will not
grow with the size of the network. Since the $\degree{\state}$ and
$\degree{\control}$ typically do not grow with $\nlinks\ntime$ or
$\ncontrols\ntime$ for freeway networks, the complexity of evaluating
the gradient for such networks can be considered linear for the adjoint
method.


\subsection{Evaluating the Partial Derivatives\label{sub:Evaluating--and}}

While no assumptions are made about the sparsity of the cost function
$\cost$, the networked-structure of the PDE system and the Godunov
discretization scheme allows us to say more about the structure and
sparsity of $\Hx$ and $\Hu$.


\paragraph{Partial derivative expressions.}

Given that the governing equations require the evaluation of a Riemann
solver at each step, we detail some of the necessary computational
steps in evaluating the $\Hx$ and $\Hu$ matrices. 

If we consider a particular governing equation $\syseq_{\link}^{\tind}\left(\state,\control\right)=0$,
then we may determine the partial term with respect to $\discrete jl\in\state$
by applying the chain rule:

\begin{align}
\pfrac{\syseq_{\link}^{\tind}}{\discrete jl}=\pfrac{\discrete{\link}{\tind}}{\discrete jl}-\pfrac{\discrete{\link}{\tind-1}}{\discrete jl} & +\frac{\Delta t}{L_{i}}f'\left(\RS_{\jdown{\link}}\left(\juncstate{\jdown{\link}}{\tind-1},\junccon{\jdown{\link}}{\tind-1}\right)_{\link}\right)\pfrac{}{\discrete jl}\left(\RS_{\jdown{\link}}\left(\juncstate{\jdown{\link}}{\tind-1},\junccon{\jdown{\link}}{\tind-1}\right)_{\link}\right)\label{eq:dhdufull} \\
& -\frac{\Delta t}{L_{i}}f'\left(\RS_{\jup{\link}}\left(\juncstate{\jup{\link}}{\tind-1},\junccon{\jup{\link}}{\tind-1}\right)_{\link}\right)\pfrac{}{\discrete jl}\left(\RS_{\jup{\link}}\left(\juncstate{\jup{\link}}{\tind-1},\junccon{\jup{\link}}{\tind-1}\right)_{\link}\right)\nonumber                       
\end{align}				
or if we consider the composed Riemann flux solver $\god_{\jn}$ in~\eqref{eq:god-jn}:

\begin{equation}
\pfrac{\syseq_{\link}^{\tind}}{\discrete jl}=\pfrac{\discrete{\link}{\tind}}{\discrete jl}-\pfrac{\discrete{\link}{\tind-1}}{\discrete jl}+\frac{\Delta t}{L_{i}}\left(\pfrac{}{\discrete jl}\left(\god_{\jdown{\link}}\left(\juncstate{\jdown{\link}}{\tind-1},\junccon{\jdown{\link}}{\tind-1}\right)\right)_{\link}-\pfrac{}{\discrete jl}\left(\god_{\jup{\link}}\left(\juncstate{\jup{\link}}{\tind-1},\junccon{\jup{\link}}{\tind-1}\right)\right)_{\link}\right)\label{eq:dhdugod}
\end{equation}


A diagram of the structure of the $\Hx$ matrix is given in Fig.~(\ref{fig:partial-ordering}).
\begin{figure}
\subfloat[\label{fig:partial-ordering}Ordering of the partial derivative terms.
Constraints and state variables are clustered first by time, and then
by cell index.]{\includegraphics[width=0.45\columnwidth]{figs-gen/dstate}

}\texttt{\hfill{}}\subfloat[\label{fig:sparsity-diagram}Sparsity structure of the $\Hx$ matrix.
Besides the diagonal blocks, which are identity matrices, blocks where
$l\neq\tind-1$ are zero.]{\includegraphics[width=0.45\columnwidth]{figs-gen/sparsity-two}

\texttt{}

}

\caption{Structure of the $\Hx$ matrix.}


\end{figure}
Similarly for $\Hu$, we take a control parameter $\condiscrete jl\in\control$,
and derive the expression:

\begin{align}
\pfrac{\syseq_{\link}^{\tind}}{\condiscrete jl}= & +\frac{\Delta t}{L_{i}}f'\left(\RS_{\jdown{\link}}\left(\juncstate{\jdown{\link}}{\tind-1},\junccon{\jdown{\link}}{\tind-1}\right)_{\link}\right)\pfrac{}{\condiscrete jl}\left(\RS_{\jdown{\link}}\left(\juncstate{\jdown{\link}}{\tind-1},\junccon{\jdown{\link}}{\tind-1}\right)_{\link}\right)\label{eq:dhdvfull} \\
& -\frac{\Delta t}{L_{i}}f'\left(\RS_{\jup{\link}}\left(\juncstate{\jup{\link}}{\tind-1},\junccon{\jup{\link}}{\tind-1}\right)_{\link}\right)\pfrac{}{\condiscrete jl}\left(\RS_{\jup{\link}}\left(\juncstate{\jup{\link}}{\tind-1},\junccon{\jup{\link}}{\tind-1}\right)_{\link}\right)\nonumber                       
\end{align}
or for the composed Godunov junction flux solver $\god_{\jn}$:

\begin{equation}
\pfrac{\syseq_{\link}^{\tind}}{\condiscrete jl}=\frac{\Delta t}{L_{i}}\left(\pfrac{}{\condiscrete jl}\left(\god_{\jdown{\link}}\left(\juncstate{\jdown{\link}}{\tind-1},\junccon{\jdown{\link}}{\tind-1}\right)\right)_{\link}-\pfrac{}{\condiscrete jl}\left(\god_{\jup{\link}}\left(\juncstate{\jup{\link}}{\tind-1},\junccon{\jup{\link}}{\tind-1}\right)\right)_{\link}\right)\label{eqdhdvgod}.
\end{equation}


Analyzing~\eqref{eq:dhdufull}, the only partial terms that are not
trivial to compute are $\pfrac{}{\discrete jl}\left(\RS_{\jdown{\link}}\left(\juncstate{\jdown{\link}}{\tind-1},\junccon{\jdown{\link}}{\tind-1}\right)_{\link}\right)$
and $\pfrac{}{\discrete jl}\left(\RS_{\jup{\link}}\left(\juncstate{\jup{\link}}{\tind-1},\junccon{\jup{\link}}{\tind-1}\right)_{\link}\right)$.
Similarly for~\eqref{eq:dhdvfull}, the only nontrivial terms are
$\pfrac{}{\condiscrete jl}\left(\RS_{\jdown{\link}}\left(\juncstate{\jdown{\link}}{\tind-1},\junccon{\jdown{\link}}{\tind-1}\right)_{\link}\right)$
and $\pfrac{}{\condiscrete jl}\left(\RS_{\jup{\link}}\left(\juncstate{\jup{\link}}{\tind-1},\junccon{\jup{\link}}{\tind-1}\right)_{\link}\right)$.
Once one obtains the solutions to these partial terms, then one can
construct the full $\Hx$ and $\Hu$ matrices and use~\eqref{eq:adjoint}
and~\eqref{eq:adjoint-grad} to obtain the gradient value.

As these expressions are written for a general scalar conservation
law, the only steps in computing the gradient that are specific to
a particular conservation law and Riemann solver are computing the
derivative of the flux function $f$ and the partial derivative terms
just discussed. These expressions are explicitly calculated for the
problem of optimal ramp metering in Section~(\ref{sec:Applications-to-Optimal}).


\subsection{Complexity of Solving Gradient via Forward Method vs. Adjoint Method\label{sub:Complexity-of-solving}}

This section demostrates the following proposition:

\begin{prop}
\textup{The total complexity for the adjoint method on a scalar hyperbolic
network of PDEs is }$O(\ntime\left(\degree{\state}\nlinks+\degree{\control}\ncontrols\right))$.\end{prop}

We can show the lower-triangular structure and invertibility of $\Hx$
by examining~\eqref{eq:init-ge} and~\eqref{eq:main-ge}. For $\tind\in\intrange 1{\ntime-1}$,
we have that $\syseq_{\link}^{\tind}$ is only a function of $\discrete{\link}{\tind}$
and of the state variables from the previous time-step $\tind-1$.
Thus, based on our ordering scheme in Section~\ref{sec:State,-control,-and}
of ordering variables by increasing time-step and ordering constraints
by corresponding variable, we know that the diagonal terms of $\Hx$ are
always $1$ and all upper-triangular terms must be zero (since those
terms correspond to constraints with a dependence of \emph{future}
values). These two conditions demonstrate both that $\Hx$ is lower-triangular
and is invertible due to the ones along the diagonal.

Additionally, if we consider taking partial derivatives with respect
to the variable $\discrete jl$, then we can deduce from Equation~\eqref{eq:main-ge}
that all partial terms will be zero except for the diagonal term,
and those terms involving constraints at time $j+1$ with links connecting
to the downstream and upstream junctions $\jdown j$ and $\jup j$
respectively. To summarize, $\Hx$ matrices for systems described
in Section~\ref{sec:State,-control,-and} will be square, invertible,
lower-triangular and each column will have a maximum cardinality equal
to $\degree{\state}$ in~\eqref{eq:dx}. The sparsity structure of
$\Hx$ is depicted in Fig.~\ref{fig:sparsity-diagram}.

Using the same line of argument for the maximum cardinality of $\Hx$,
we can bound the maximum cardinality of each column of $\Hu$. Taking
a single control variable $\condiscrete jl$, the variable can only
appear in the constraints at time-step $j+1$ that correspond to a link
that connects to a junction $\jn$ such that $\condiscrete jl\in\junccon{\jn}{l+1}$.
These conditions give us the expression for $\degree{\control}$ in~\eqref{eq:dv},
or the maximum cardinality over all columns in $\Hu$.

If we only consider the lower triangular form of $\Hx$, then the
complexity of solving for the gradient using the forward system is
$O(\left(\nlinks\ntime\right)^{2}\ncontrols\ntime)$, where the dominating
term comes from solving~\eqref{eq:j-v}, which requires the solution
of $\ncontrols\ntime$ separate $\nlinks\ntime\times\nlinks\ntime$
lower-triangular systems. The lower-triangular system allows for forward
substitution, which can be solved in $O(\left(\nlinks\ntime\right)^{2})$
steps, giving the overall complexity $O(\left(\nlinks\ntime\right)^{2}\ncontrols\ntime)$.
The complexity of computing the gradient via the adjoint method is
$O(\left(\nlinks\ntime\right)^{2}+\left(\nlinks\ntime\right)\left(\ncontrols\ntime\right))$,
which is certainly more efficient than the forward-method, as long
as $\ncontrols\ntime>1$. The efficiency is gained by considering
that~\eqref{eq:adjoint} only requires the solution of a single $\nlinks\ntime\times\nlinks\ntime$
\emph{upper}-triangular system (via backward-substitution), followed
by the multiplication of $\lambda^{T}H_{v}$, an $\nlinks\ntime\times\nlinks\ntime$
and an $\nlinks\ntime\times\ncontrols\ntime$ matrix in~\eqref{eq:adjoint-grad},
with a complexity of $O(\left(\nlinks\ntime\right)^{2}+\left(\nlinks\ntime\right)\left(\ncontrols\ntime\right))$.

For the adjoint method, this complexity can be improved upon by considering
the sparsity of the $\Hx$ and $\Hu$ matrices, as detailed in Section~\ref{sub:Complexity-of-solving}.
For the backward-substitution step, each entry in the $\lambda$ vector
is solved by \emph{at most} $\degree{\state}$ multiplications, and
thus the complexity of solving~\eqref{eq:adjoint} is reduced to
$O(\degree{\state}\nlinks\ntime)$. Similarly, for the matrix multiplication
of $\lambda^{T}H_{v}$, while $\lambda$ is not necessarily sparse,
we know that each entry in the resulting vector requires at most $\degree{\control}$
multiplications, giving a complexity of $O(\degree{\control}\ncontrols\ntime)$. Furthermore, if a sparse implementation of the $\Hx$ and $\Hu$ matrices are used, then memory usage will also scale linearly with the number of state and control variables.
\input{ramp.tex}%!TEX root = restart.tex
\section{Numerical results for model predictive control simulations\label{sec:Numerical-results-for}}

To demonstrate the effectiveness of using the adjoint ramp metering
method to compute gradients, we implemented the algorithm in MATLAB.
The algorithm can then be used as a gradient computation subroutine
inside any descent-method optimization solver that takes advantage
of first-order gradient information. Our implementation makes use
of the open-source \emph{IpOpt} solver~\cite{Andreas2005}. To serve
as comparisons, two other metering schemes were run:
\begin{enumerate}
\item No control: the metering rate is set to 1 on all onramps at all times.
\item Alinea~\cite{Papageorgiou1991}: a popular, feedback-based ramp metering
algorithm. Alinea is computationally efficient and decentralized,
making it a popular choice for large networks, but does not take estimated
boundary flow data as input. Since Alinea has a number of tuning parameters,
we perform a \emph{modified} grid-search technique over the different
parameters that scales linearly with the number of onramps, and select
the best-performing parameters. A \emph{full} grid-search approach
scales exponentially with the number of onramps, rendering it infeasible
for moderate-size freeway networks.
\end{enumerate}
\textbf{Note: } We also implemented gradient descent using a finite differences approach to compute the gradient, similar to~\cite{Frejo2011,Ramon2013}, but it proved to be computationally infeasible for even small, synthetic
networks. Running ramp metering on even a network of 4 links over
6 time-steps for 5 gradient steps took well over 4 minutes,
rendering the method useless for real-time applications. The comparison
of running times of finite differences versus the adjoint method is given in
Figure~\ref{fig:Running-time-of}. We do not consider the finite differences in further results, due to the impractically large running times, which only becomes worse as the problem scales to larger networks and time horizons.
\begin{figure}
\begin{centering}
\includegraphics[width=0.5\columnwidth]{images/itergrad}
\par\end{centering}

\caption{Running time of metering algorithm with and without gradient computations.
Network consists of 4 links and 6 time-steps with synthetic boundary
flux data. The method using gradient computations via the adjoint
method converged well before the first step was completed with the
method that used perturbations to compute the gradient.\label{fig:Running-time-of}}
\end{figure}



\subsection{Network\label{sub:Network}}

As input into the optimization problem, we constructed a model of
a 19.4 miles
 stretch of the I15 South freeway in San Diego,
California between San Marcos and Mira Mesa.The network has $\nlinks=$
\input{fwNLinks} links, $\ncontrols=$\input{fwNRamps} onramps,
with boundary data specified for $\ntime=$ \input{fwNTime} time-steps,
for a time horizon of 120.0 minutes
 given $\Delta t=$ \input{fwdt}.
The network is shown in Figure~\ref{fig:Model-of-section}.
\begin{figure}
\begin{centering}
\includegraphics[width=0.5\columnwidth]{images/map}
\par\end{centering}

\caption{Model of section of I15 South in San Diego, California. The freeway
section spanning~19.4 miles was split into~125 links with 9 onramps.\label{fig:Model-of-section}}
\end{figure}


Link length data was obtained using the Scenario Editor software developed
as part of the Connected Corridors project, a collaboration between
UC Berkeley and PATH research institute in Berkeley, California.
Fundamental diagram parameters, split ratios, and boundary data were
also obtained using calibration techniques developed by Connected
Corridors. Densities resulting in free-flow speeds were chosen as
initial conditions on the mainline and onramps. The data used in calibration
was taken from PeMS sensor data during a morning rush hour period,
scaled to generate congested conditions. The input data was chosen
to demonstrate the effectiveness of the adjoint ramp metering method
in a real-world setting. A profile of the mainline and onramps during
a forward simulation of the network is shown in Figure~\ref{fig:Density-and-queue}
under the described boundary conditions.
\begin{figure}[b]
\subfloat[Density profile. The units are the ratio of a link's vehicle density
to a link's critical density. Values less than 1 represent free flow,
while values greater than 1 represent congestion.\label{fig:Density-profile.}]{\includegraphics[width=0.45\columnwidth]{images/ncdensity}

}\hfill{}\subfloat[Onramp queue profile in units of vehicles. The onramps are only present
at certain junctions, thus why the nonzero queue lengths are sparse
in this diagram.\label{fig:Density-profile.-2}]{\includegraphics[width=0.45\columnwidth]{images/ncqueue}

}

\caption{Density and queue profile of no-control freeway simulation. In the
first 80 minutes, congestion pockets form on the freeway and queues
form on the onramps, then eventually clear out before 120 minutes.\label{fig:Density-and-queue}}
\end{figure}



\subsection{Finite-horizon optimal control\label{sub:Finite-horizon-optimal-control}}


\paragraph{Experimental setup}

The adjoint ramp metering algorithm is compared to the reactive Alinea
scheme, where we assume perfect boundary conditions and initial conditions
are available. The actual cost used to compare the performances of
the different methods is \emph{delay}, or the total travel time minus
the free-flow total travel time. The free-flow total travel time is
computed by assuming the critical density is infinite for all links,
thus no backwards moving congestion results from high density. The
delay gives an indication of how much improvement is possible, given
that total travel time cannot be zero at optimum.


\paragraph{Results}

\begin{figure}[t]
\subfloat[Density difference profile in units of vehicles per mile.\label{fig:long-sim-density}]{\includegraphics[width=0.45\columnwidth]{images/densdiff}

}\hfill{}\subfloat[Queue difference profile in units of vehicles.\label{fig:long-sim-queue}]{\includegraphics[width=0.45\columnwidth]{images/queuediff}

}

\caption{Profile differences for mainline densities and onramp queues. Evidenced
by the mainly negative differences in the mainline densities and the
mainly positive differences in the onramp queue lengths, the adjoint
ramp metering algorithm effectively limits onramp flows in order to
reduce mainly congestion.\label{fig:long-sim}}
\end{figure}


Figure~\ref{fig:long-sim} shows a difference profile between the
no control simulation and the simulation applying the ramp metering
policy generated from the adjoint method. Negative differences in
Figures~\ref{fig:long-sim-density} and~\ref{fig:long-sim-queue}
indicate where the adjoint method resulted in fewer vehicles for the
specific link and time-step. The adjoint method was successful in
intelligently deciding which ramps should be metered in order to improve
throughput for the mainline.
\begin{figure}
\begin{centering}
\includegraphics[width=0.65\columnwidth]{images/longsim}
\par\end{centering}
\caption{Performance versus simulation time for freeway network. The results
indicate that the algorithm can run with performance better than Alinea
if given an update time around 15 minutes.}\label{fig:running-time}
\end{figure}


Running time analysis shows that the adjoint method can produce beneficial
results in real time applications. Figure~\ref{fig:running-time} details the improvement of the adjoint method as a function of the overall running time of the algorithm. After just a few gradient steps, the
adjoint method outperforms the Alinea method. Given that the time
horizon of two hours is longer than the period of time one can expect
reasonably accurate boundary flow estimates, more practical simulations
with shorter time horizons should permit more gradient steps in a
real-time setting.

While the adjoint method leads to queues with a considerable number of cars in some onramps, this can be addressed by introducing barrier terms into the cost function that limit the
maximum queue length. The Alinea method tested for the I15 network
had no prescribed maximum queue lengths as well, but was not able
to produce significant improvements, while the adjoint method was
more successful.


\subsection{Model predictive control\label{sub:Model-predictive-control}}

To study the performance of the algorithm under noisy input data,
we embed both our adjoint ramp metering algorithm and the Alinea algorithm
inside of a \emph{Model predictive control }(MPC) loop.


\paragraph{Experimental setup}

The MPC loop begins at a time $t$ by estimating the initial conditions
of the traffic on the freeway network and the predicted boundary fluxes
over a certain time horizon $T_{h}$. These values are noisy, as exact
estimation of these parameters is not possible on real freeway networks.
The estimated conditions are then passed to the ramp metering algorithm
to compute an optimal control policy over the $T_{h}$ time period.
The system is then forward simulated over an update period of $T_{u}\le T_{h}$,
using the exact initial conditions and boundary conditions, as opposed
to the noisy data used to compute control parameters. The state of
the system and boundary conditions at $t+T_{u}$ are then estimated
(with noise) and the process is repeated.

A non-negative\emph{ noise factor} is used to study how the adjoint
method and Alinea perform as the quality of estimated data decreases.
The noise factor can be summarized as follows:

\[
\bar{\discrete{}{}}=\discrete{}{}*(1+noise\_factor*R)
\]


where $R$ is a uniformly distributed random variable with mean $0$
and domain $\left[-0.5,0.5\right]$. The noise factor was applied
to both initial and boundary conditions.

Two different experiments were conducted:
\begin{enumerate}
\item \textbf{Real-time I15 South}: MPC is run for the I15 South network
with $T_{h}=27$ minutes and $T_{u}=14$ minutes. A noise factor of
2\% was chosen for the initial and boundary conditions. The number
of iterations was chosen in order to ensure that each MPC iteration
finished in the pre-determined update time $T_{u}$.
\item \textbf{Noise Robustness}: MPC is for over a synthetic network with
length 5 miles and boundary conditions over 50 minutes. The experiments
are run over a profile of noise factors between 1\% and 200\%.
\end{enumerate}

\paragraph{Results}


\subparagraph{Real-time I15 South}

The results are summarized in Figure~\ref{fig:MPC-performance-on}.
The adjoint method applied once to the entire horizon with perfect
boundary and initial condition information serves as a baseline performance
for the other simulations, which had noisy input data and limited
knowledge of predicted boundary conditions. The adjoint method still
performs well under the more realistic conditions of the MPC loop
with noise, resulting in 75\% delay as compared to the no control
scenario as compared to the 71\% delay achieved by the perfect-knowledge
adjoint control. Alinea performed worse than adjoint method, only
achieving a 95\% delay as compared to no control. The results indicate
that under a realistic assumption of a 2\% noise factor in the sensor
information, the ability to consider boundary conditions in producing
ramp metering policies as an improvement upon strictly reactive policies,
such as Alinea.

\begin{figure}
\subfloat[Percentage decrease in delay.\label{fig:MPC-performance-on}]{\includegraphics[width=0.45\columnwidth]{images/longmpc}

}\hfill{}\subfloat[Performance with increasing sensor noise.\label{fig:Ramp-metering-performance-1}]{\includegraphics[width=0.45\columnwidth]{images/noiseplot}
}
\caption{Summary of ramp metering simulations on I15 South netowrks.}
\end{figure}



\subparagraph{Robustness to noise}

Simulation results on the synthetic network with varying levels of
noise are shown in Figure~\ref{fig:Ramp-metering-performance-1}.
The adjoint method is able to outperform the Alinea method when the
noise level is less than 80\%, a reasonable assumption for data provided
by well-maintained loop detectors. As the initial and boundary condition
data deteriorates, the adjoint method becomes useless. Since Alinea
does not rely on boundary data, it is able to produce improvements,
even with severely noisy data. The results indicate that the adjoint
method will outperform Alinea under reasonable noise levels in the
sensor data.
%!TEX root =article.tex
conclusions

\bibliographystyle{plain}
\bibliography{/Users/jdr/Documents/articles/AdjointPaper}

\end{document}