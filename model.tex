%!TEX root =article.tex
\subsection{Network model} % (fold)
\label{sub:network_model}

% go over the sequential links, onramp, offramp assumptions.

% introduce in a continuous fashion, starting with PDE work to show that it's a rigorous reduction from PDE stuff.

We model a freeway stretch as a sequence of junctions $\junctions=\seq{\jn}{\jnidx},\jnidx=\left[0,\ldots,\nlinks -1\right]$.
Furthermore, each junction $\jn_{\jnidx}$ has an onramp with 
a maximum out flux $\rampMax_{\jnidx}$
and a merging priority $\priority_{\jnidx}$, and an offramp with
a throughput split ratio $\splitRatio_{\jnidx}$\footnote{The network parameters can all be time-dependent in the general case, but are presented as constant for simplicity.}. Connecting junctions $\jn_{\jnidx}$ and $\jn_{\jnidx+1}$ is road link $\link_{\jnidx}\in\links$, which has length $\length_\jnidx$, and the triangular fundamental diagram parameters:
free-flow speed $\vfd_{\jnidx}$, 
congestion wave speed $\wfd_{\jnidx}$, 
critical density $\pcfd_{\jnidx}$, 
max density $\pmfd_{\jnidx}$, and
max flux $\fmfd_{\jnidx}$. Figure~\ref{fig:freeway-network} summarizes the network topology and parameters.

\begin{figure}
\centering
\includegraphics[width=.25\columnwidth]{figures/placeholder}
\caption{Freeway stretch network and parameters}
\label{fig:freeway-network}
\end{figure}

\subsection{Dynamics} % (fold)
\label{sub:dynamics}

Since we wish to have numerical stability in our model dynamics, we use a macroscopic traffic-flow model, derived from partial differential equations.
We use a first-order LWR \cite{lighthill1955kinematic,richards1956shock} network model developed in~\cite{Monache2013}, which is particularly suitable for freeway control and can be seen as an extension of the model in~\cite{garavello2006traffic}. The model in~\cite{Monache2013} is employed for its strong boundary conditions on the onramps to ensure all flux demand passes through the network~\cite{strub2006weak} and ability to accurately model $2\times2$ junctions without blocking onramp flux.

In order to simulate the model or use it within an optimization framework, we first discretize the model using the Godunov scheme~\cite{godunov1959}, as detailed in~\cite{Monache2013}. The initial conditions of the model are set by specifying an initial density $\po$ (vehicles per unit length) for all links and an initial queue length $\lo$ (vehicles) for all onramps. The boundary conditions are specified as ramp flux demands $\rampBC$ for all ramps and all time steps, noting that the upstream mainline source is modeled as an onramp.

The system of equations for the discretized system is given in Equation~\eqref{eqn:system-of-equations}, for a given link $i \in \left[0,\ldots,\horizon\right]$ and time step $k \in \left[0,\ldots,\horizon\right]$.

\begin{equation}
\begin{aligned}
\label{eqn:system-of-equations}
% density
\idx{\density}{i}{k} & = & &
\idx{\density}{i}{k-1} + \frac{\dt}{\length_i} \left(\idx{\fIn}{i}{k-1} - \idx{\fOut}{i}{k-1}\right)
\\
% queue
\idx{\queue}{i}{k} & = & &
\idx{\queue}{i}{k-1} + \dt \left(\idx{\rampBC}{i}{k-1} - \idx{\fRamp}{i}{k-1}\right) \\
% demand
\idx{\demand}{i}{k} & = & &
\min\left( \fmfd_i, \vfd_i \idx{\density}{i}{k} \right)
\\
% supply
\idx{\supply}{i}{k} & = & &
\min\left(\fmfd_i, \wfd \left( \pmfd_i - \idx{\density}{i}{k} \right)\right)
\\
% ramp demand
\idx{\rampDemand}{i}{k} & = & &
\min\left( \rampMax_i, \idx{\queue}{i}{k} / \dt \right)
\\
% flux in
\footnotemark
\idx{\fIn}{i}{k} & = & &
\min\left( \splitRatio_{i-1} \idx{\demand}{i-1}{k} + \idx{\rampDemand}{i}{k}, \idx{\supply}{i}{k} \right)
\\
% flux out
\idx{\fOut}{i}{k} & = & &
\begin{cases}
% case 1
\idx{\demand}{i}{k} &
\frac{\idx{\supply}{i+1}{k}\priority_{i+1}}{1 + \priority_{i+1}} \ge \idx{\demand}{i}{k}\splitRatio_{i+1}{k} \\
% case 2
\left(\idx{\fIn}{i+1}{k} - \idx{\rampDemand}{i_1}{k}\right) / \splitRatio_{i +1} & 
\frac{\idx{\supply}{i+1}{k}}{1 + \priority_{i+1}} \ge \idx{\rampDemand}{i}{k} \\
% case 3
\frac{\idx{\supply}{i+1}{k}\priority_{i+1}}{\left(1 + \priority_{i+1}\right)\splitRatio_{i+1}{k}} &
\text{otherwise}
\end{cases}
\\
% ramp flux
\idx{\fRamp}{i}{k} & = & &
\idx{\fIn}{i}{k} - \splitRatio_{i} \idx{\fOut}{i-1}{k}
\end{aligned}
\end{equation}

\footnotetext{This is a simplification of~\cite{Monache2013} for the case when the ramp empties on a time step, with minimal effect on numerical results.}

The intermediate variables (those used to calculate the states variables, $\density$ and $\queue$) represent the following values; $\demand$, the demand, is the amount of flux that a link can send downstream in a time step; $\supply$, the supply, is the amount of flux that a link can receive; $\rampDemand$, the ramp demand, is the amount of flux that an onramp can send; $\fIn/\fOut$ is the flux that enters/exits upstream/downstream of a link; $\fRamp$ is the flux that exits an onramp.

By taking the system in Equation~\eqref{eqn:system-of-equations}, the initial and boundary conditions, and subtracting the right hand side of each equation in the system, then we can construct the system $\sysMat \left( \state \right) = 0$, where $x$ is the entire state vector, ordered first by time step, then by variable type (e.g. $\density, \queue, \demand,$ etc), and finally by link. More details about the ordering and its effect on the efficiency of gradient computation is given in Section~\ref{sub:efficient_solution_of_system_via_depth_first_search_system_solution}. The system of equations can be solved in linear time via forward-substitution, similar to other first-order network models such as the cell transmission model~\cite{daganzo1995cell}.

\subsection{Control parameters for ramp metering} % (fold)
\label{sub:control_parameters_for_ramp_metering}

The $\sysMat$ system in Section~\ref{sub:dynamics} expresses the relationships of all state variables, which can be solved to give a forward simulation of the system. In order to effect the system externally, we must introduce some control parameters. We denote control parameters with $u\in\Rbb^{|u|}$, and state parameters with $x\in\Rbb^{|x|}$. One method of control, known as variable speed limits~\cite{Muralidharan2009}, would set the free-flow velocity of the fundamental diagrams as control parameters. State estimation problems can be posed in this fashion by letting initial conditions become control parameters, and searching for an initial condition that drifts to a known state~\cite{Jacquet}. For this article, we study ramp metering, which can be modeled as a control parameter that limits the outflow of ramps. Thus, we say $\idx{\control}{i}{k}$ sets an upper limit on the flux out of onramp $i$ at time $k$. We modify the $\rampDemand$ equation in~\eqref{eqn:system-of-equations} to account for this new parameter:

\begin{equation}
\label{eqn:modified-ramp-demand}
\idx{\rampDemand}{i}{k} = 
\min\left( \idx{\control}{i}{k}, \rampMax_i, \idx{\queue}{i}{k} / \dt \right)
\end{equation}

% subsection control_parameters_for_ramp_metering (end)

% subsection dynamics (end)

% subsection network_model (end)