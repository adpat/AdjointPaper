
What is left to show is the model above has a unique solution, and
that the Riemann solver at the junction is self-similar, i.e.

\[
RS_{\icell}\left(\density{\icell}{b_{i},t},\rampqueue{\icell}t,\density{\icell+1}{a_{\icell+1},t}\right)=RS_{\icell}\left(RS_{\icell}\left(\density{\icell}{b_{i},t},\rampqueue{\icell}t,\density{\icell+1}{a_{\icell+1},t}\right)\right)
\]


The uniqueness is guaranteed from the fact that there is a unique
mapping from our problem's demand/supply formulation to the flux maximization problem of the model in~\cite{garavello2006traffic} which guarantees a unique
flux solution. The model also guarantees a unique mapping from initial
density and flux solution to final boundary densities, $\hat{\densitysymbol}$.
Therefore, we have a uniqueness guarantee from the Garavello model.

We demonstrate that the solution is self-similar by enumerating the
different flux conditions. We note that the condition of the existence
of an offramp ($\offrampratio{\icell}t>0$) leads to the assumption
that the priority constant $P=1$.

For the case when the flux is \emph{demand-constrained} (there is
more supply than demand), then the flux solution will be:

\begin{eqnarray*}
\flux{\icell}t & = & \celldemand{\icell}t\\
\rampflux{\icell}t & = & \rampdemand{\icell}t\\
\flux{\icell+1}t & = & \celldemand{\icell}t\left(1-\offrampratio{\icell}t\right)+\rampdemand{\icell}t
\end{eqnarray*}


and the density solutions will be:

\begin{eqnarray*}
\hat{\density{\icell}t} & = & \density{\icell}t\\
\hat{\density{\icell+1}t} & = & \tau\left(\celldemand{\icell}t\left(1-\offrampratio{\icell}t\right)+\rampdemand{\icell}t,\density{\icell+1}t\right)<\density{\icell+1}t
\end{eqnarray*}


It is clear that the Riemann solution of this new condition will have
the same demand and a supply larger than the demand, which gives a
self-similar solution.

For the case when the flux is \emph{supply-constrained}, we consider
two cases: when both the mainline and ramp have fluxes below their
demands, and the case when exactly one of the mainline or the ramp
have flux equal to demand. It is not possible to have both demands
satisfied at once, otherwise it would be demand-constrained.

For the first case, the flux values are determined by the intersection
of the equation $\rampflux{\icell}t=\cellsupply{\icell+1}t-\left(1-\offrampratio{\icell}t\right)\flux{\icell}t$
and $\rampflux{\icell}t=\frac{1-P}{P}\flux{\icell}t$, where we have
assumed the intersection occurs within feasible flow region of mainline
and ramp. The supply of the Riemann solution will not change ($\hat{\density{\icell+1}t}=\density{\icell+1}t$).
The new demands will be greater than or equal to the previous demands.
The function to determine the new intersection point (to solve for
fluxes) has the same supply term and therefore is identical to the
previous function. Therefore the intersection point is the same as
the previous. Furthermore, this point must as well be inside the feasible
region of the incoming fluxes, since the feasible region has only
grown. We have shown that the flux solution to the Riemann solution
is identical to the original flux solution, and will therefore lead
a self-similar solution.

For the second case, the important intersection point is with the
supply constraint and the demand constraint which is tight. The demand
constraint becomes tight with the supply constraint and the priority
constraint lead to an infeasible flux solution. Let us first assume
that the mainline has the tight demand constraint, then the flux solutions
are:

\begin{eqnarray*}
\flux{\icell}t & = & \celldemand{\icell}t\\
\rampflux{\icell}t & = & \cellsupply{\icell+1}t-\left(1-\offrampratio{\icell}t\right)\celldemand{\icell}t\\
\flux{\icell+1}t & = & \cellsupply{\icell+1}t
\end{eqnarray*}


which leads to solution densities identical to the initial densities.
This is clearly a self-similar solution.

Now assume that the ramp demand constraint is tight. The fluxes are
now:

\begin{eqnarray*}
\flux{\icell}t & = & \left(\cellsupply{\icell+1}t-\rampdemand{\icell}t\right)/\left(1-\offrampratio{\icell}t\right)\\
\rampflux{\icell}t & = & \rampdemand{\icell}t\\
\flux{\icell+1}t & = & \cellsupply{\icell+1}t
\end{eqnarray*}


where now the solution density on mainline $\icell$ leads to an increased
demand, while the demand on the ramp does not ever change. Since the
ramp demand is the same, the intersection point that determined the
first fluxes will be identical for the Riemann solution. Additionally,
the feasible region only increases, meaning the fluxes of the Riemann
system will be the same as the initial problem.

Since all cases have been enumerated, we have demonstrated the self-similar
property of the buffer ramp junction problem.

%-----------------------------------------------------------------------------------------------------------------------------------------------------------------
