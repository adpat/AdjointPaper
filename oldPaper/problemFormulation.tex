% !TEX root = rampMeteringViaTheAdjoint.tex

\subsection{Nomenclature}


\section*{Nomenclature}

\begin{longtable}{|c|c|c|}
\hline 
Variable  & Space  & Meaning\tabularnewline
\hline 
\hline 
$t$  & $\mathbb{R}_{+}$  & time\tabularnewline
\hline 
$x$  & $\mathbb{R}$  & space\tabularnewline
\hline 
$\nlinks$  & $\mathbb{N}$  & Number of links\tabularnewline
\hline 
$\jns$  &  & Set of junctions\tabularnewline
\hline 
$\links=\left[1,\nlinks\right]$  & $\mathbb{N}^{N}$  & Set of links\tabularnewline
\hline 
$L_{\link}$  & $\mathbb{R}_{+}$  & Length of link $\link\in\links$\tabularnewline
\hline 
$\cvar_{i}\left(t,x\right)$  & $\mathbb{R}_{+}\times\left]0,L_{i}\right[\rightarrow\mathbb{R}$  & conserved quantity for link $\link\in\links$ as function of $x$\tabularnewline
\hline 
$\initstate$  & $BV\cap L_{\text{loc}}^{1}$  & continuous intialial condition\tabularnewline
\hline 
$\dvar_{\link}^{\tind}$  & $\mathbb{R}$  & discrete conserved quantity for link $i$ at time-step $k$\tabularnewline
\hline 
$f(\cvar)$  & $f:\mathbb{R}\rightarrow\mathbb{R}$  & flux function\tabularnewline
\hline 
$\initstate$  & $\mathbb{R}$  & Riemann data\tabularnewline
\hline 
$\cvar^{-}$  & $\mathbb{R}$  & Left state of the Riemann data\tabularnewline
\hline 
$\cvar^{+}$  & $\mathbb{R}$  & Right state of the Riemann data\tabularnewline
\hline 
$\bar{x}$  & $\mathbb{R}$  & Point of discontinuity in Riemann problem\tabularnewline
\hline 
$\ssvar$  & $\mathbb{R}$  & Self-similar solution of the Riemann problem\tabularnewline
\hline 
$n_{J}$  & $\mathbb{N}$  & Number of incoming links at a junction J\tabularnewline
\hline 
$m_{J}$  & $\mathbb{N}$  & Number of outgoing links at a junction J\tabularnewline
\hline 
$\Inc\left(\jn\right)=\tuple{\jlink{\jn}1}{\jlink{\jn}{\ninc_{\jn}}}\subset\links$  &  & Set of incoming links at a junction J\tabularnewline
\hline 
$\Out\left(\jn\right)=\tuple{\jlink{\jn}{\ninc_{\jn}+1}}{\jlink{\jn}{\ninc_{\jn}+\nout_{\jn}}}\subset\links$  &  & Set of outgoing links at a junction J\tabularnewline
\hline 
$\jup{\link}\in\jns$  &  & Upstream junction for the link $\link\in\links$\tabularnewline
\hline 
$\jdown{\link}\in\jns$  &  & Downstream junction for the link $\link\in\links$\tabularnewline
\hline 
$\RS$  & $\mathbb{R}^{m+n}\rightarrow\mathbb{R}^{m+n}$  & Riemann Solver\tabularnewline
\hline 
$\trace{\cvar}_{\link}$  & $\mathbb{R}^{m+n}$  & Trace for a link $i$ at the junction\tabularnewline
\hline 
$\Delta t$  & $\mathbb{R}$  & Time grid size\tabularnewline
\hline 
$\Delta x$  & $\mathbb{R}$  & Space grid size\tabularnewline
\hline 
$t^{\tind}=k\Delta t$  & $k\in\mathbb{N}$  & Time grid points\tabularnewline
\hline 
$t^{\xind}=l\Delta x$  & $l\in\mathbb{Z}$  & Space grid points\tabularnewline
\hline 
$\lambda^{\max}$  & $\mathbb{R}$  & Wave speed\tabularnewline
\hline 
$\juncstate{\jn}{\tind}$  & $\mathbb{R}^{m_{\jn}+n_{\jn}}$  & state variables at a junction $\jn\in\jns$ at a time-step $\tind$\tabularnewline
\hline 
$T$  & $\mathbb{N}$  & Number of time steps\tabularnewline
\hline 
$\junctrace{\jn}{}$  & $\mathbb{R}^{m_{\jn}+n_{\jn}}$  & solution of $\RS$ at a junction $\jn\in\jns$ at a time-step $\tind$\tabularnewline
\hline 
$\junccon{\jn}{\tind}$  & $\mathbb{R}^{\ncontrols_{\jn}}$  & control variables at a junction $\jn\in\jns$ at a time-step $\tind$\tabularnewline
\hline 
$\syseq_{\link}^{\tind}$  & $\mathbb{R}^{\nlinks\ntime}\times\mathbb{R}^{\ncontrols\ntime}$  & update equation\tabularnewline
\hline 
$\cost$  & $\mathbb{R}^{\nlinks T}\times\mathbb{R}^{N_{\control}T}$  & cost function\tabularnewline
\hline 
$\lambda$  & $\mathbb{R}^{\nlinks\ntime}$  & adjoint variable\tabularnewline
\hline 
$D_{\state}$  & $\mathbb{N}$  & maximum junction degree on the network \tabularnewline
\hline 
$D_{\control}$  & $\mathbb{N}$  & maximum number of constraints \tabularnewline
\hline 
$\splitratio_{2\link}^{\tind}$  & $[0,1]$  & offramp split ratio \tabularnewline
\hline 
$\boundaryDemand{2\link-1}{\tind}$  &  & flux demand at the boundary of onramp $2\link-1$\tabularnewline
\hline 
$\barrierTerm$  &  & barrier penalty coefficient \tabularnewline
\hline 
$\demand_{2\left(\link-1\right)}$  &  & demand on the link $2(i-1)$ \tabularnewline
\hline 
$\rampDemand_{2\link-1}^{\tind}$  &  & demand from onramp $2\link-1$\tabularnewline
\hline 
$\supply_{2\link}^{\tind}$  &  & supply on the link $2i$ \tabularnewline
\hline 
$\ffspeed_{\link}$  & $\mathbb{R}_{+}$ & free flow speed for link $\link$\tabularnewline
\hline 
$\congspeed_{\link}$  & {[}0,1{]}  & congestion speed $\link$\tabularnewline
\hline 
$p_{2\left(\link-1\right)}$  & {[}0,1{]}  & merging parameter \tabularnewline
\hline 
\end{longtable}


%-----------------------------------------------------------------------------------------------------------------------------------------------------------------
\subsection{System description}

We consider a discretization of the continuous system described in Section~\ref{sec:continuous_problem}, using the Godunov scheme. A diagram describing the system is given in fig.~\ref{fig:system}. Discrete time is indexed by $\itime \in \{0, \dots, \ntime \}$. The mainline is divided into $\ncell$ cells, indexed by $\icell \in \{1, \dots, \ncell\}$. The density on cell~$\icell$ at time step $\itime$ is given by $\density{\icell}{\itime}$. The incoming (respectively outgoing) flux to cell~$\icell$ at time step~$\itime$ is given by $\flowin{\icell}{\itime}$ (respectively $\flowout{\icell}{\itime}$). We add a ghost cell at the entrance of the network, cell $\icell = 0$, to impose the boundary flow, or flow demand, given at time step~$\itime$ by~$\inputflux{0}{\itime}$. Each cell~$\icell \in \{1, \dots, \ncell-1\}$ is followed by a two-two junction (referred to as junction~$\icell$), that connects the mainline to an on-ramp and an off-ramp (referred to as on-ramp~$\icell$ and off-ramp~$\icell$). The flow demand from the off-ramp is determined by the control $\rampcontrol{\icell}{\itime}$ (that is the maximum flux out of the ramp) and the car count $\rampqueue{\icell}{\itime}$ on the ramp. The junction flows are determined in the same way as in the continuous system. On-ramp~$\icell$ is subject to flux input (or flux demand) given by the sequence~$(\inputflux{\icell}{\itime})_{\itime}$.

\begin{figure}[h]
\centering
\resizebox{\columnwidth}{!}{

\begin{tikzpicture}[scale=1.4]

\def \linkWidth {1cm}
\def \nodeWidth {0.3cm}

\coordinate (arrow) at (1.8,0);
\coordinate (dots) at (0.5,0);
\coordinate (unit) at (2.5,0);

\node (link0) at (0,0) [rectangle, minimum width=\linkWidth, draw] {$0$};
\node (link1) at ($(link0) + 1*(unit)$) [rectangle, minimum width=\linkWidth, draw] {$1$};
\node (linkI) at ($(link1) + 2*(unit)$) [rectangle, minimum width=\linkWidth, draw] {$\icell$};
\node (nodeI) at ($(linkI) + 1*(unit)$) [circle, minimum width=\nodeWidth, draw] {$\icell$};

\node (onrampI) at ($(linkI)-(0,0.8)$) [rectangle, minimum width=\linkWidth, draw] {on-ramp $\icell$};
\node (offrampI) at ($(linkI) + 1.4*(unit) - (0,0.5)$) {};

\node (linkI1) at ($(nodeI) + 1.45*(unit)$) [rectangle, minimum width=\linkWidth, draw] {$\icell+1$};

\node (linkN) at ($(linkI1)+1.8*(unit)$) [rectangle, minimum width=\linkWidth, draw] {$\ncell$};

% link 0
\draw[->] ($(link0)-(arrow)$) -- (link0) node[above, midway] {$\inputflux{0}{\itime}$};
\draw[->] (link0) -- (link1) node[above, midway] {$\flowout{0}{\itime}$};

% link 1
\draw[->] (link1) -- ($(link1) + (arrow)$) node[above, midway] {$\flowout{1}{\itime}$};

%dots
\draw ($(link1) + (arrow) + (dots)$) node{$\dots$} ;

% link i
\draw[->] ($(linkI) - 1.3*(arrow)$) -- (linkI)node[above, xshift=-1.1cm] {$\flowin{\icell}{\itime}$};
\draw[->] (linkI) -- (nodeI) node[above, midway] {$\flowout{\icell}{\itime}$};
\draw[->] (nodeI) -- (linkI1) node[above, xshift=-1.1cm]{$\flowin{\icell+1}{\itime}$} ;

%node i
\draw[->] (onrampI) [anchor=right]-- (nodeI) node[below, midway]{$\rampflow{\icell}{\itime}$};
\draw[->] ($(onrampI) - (arrow)$) [anchor=right]-- (onrampI) node[above, midway]{$\inputflux{\icell}{\itime}$};
\draw[->] (nodeI) -- (offrampI) node[below, xshift=-0.3cm]{$\offrampratio{\icell}{\itime} \flowout{\icell}{\itime}$};

%link i+1
\draw[->] (linkI1) -- ($(linkI1) + (arrow)$) node[above, midway] {$\flowout{\icell+1}{\itime}$};

%dots
\draw ($(linkI1)+(arrow)+(dots)$) node{$\dots$} ;

% link N
\draw[->] ($(linkN) - (arrow)$) -- (linkN) node[above, midway] {$\flowin{\ncell}{\itime}$};
\draw[->] (linkN)--($(linkN) + (arrow)$) node[above, midway] {$\flowout{\ncell}{\itime}$};

% box
\draw[dashed] ($(linkI) + (-2,1)$) rectangle ($(nodeI) + (1.8,-1.5)$) node[below]{block $\icell$};

\end{tikzpicture}
}
\caption{Flow variables and boundary flows in the system.}
\label{fig:system}
\end{figure}

\subsubsection{Choosing the correct space and time discretization}

TODO: Add the CFL conditions and discuss how to select the proper cell size. Smaller cells will violate the CFL condition and larger cells will cause numerical inaccuracies when vehicles need more that one time step to pass through the cell in free flow. 





%-----------------------------------------------------------------------------------------------------------------------------------------------------------------
\subsection{System equations}

% !TEX root = rampMeteringViaTheAdjoint.tex
Let $x$ denote the state vector of the system and let $u$ denote the vector of control variables, $u = (u(0), \dots, u(\ntime))$, where at time $\itime$, $u(\itime)$ is given by
\[
u(\itime) = (\rampcontrol{1}{\itime}, \dots, \rampcontrol{\ncell-1}{\itime})
\]
The system equations are written formally in the form $H(x,u)=0$. The discretized system can be described using eight types of constraints, given by $\H{c}{\itime}{\icell} = 0$ for $c \in \{1, \dots, 8\}$, where we index each equality constraint by time index $\itime$, and cell index $\icell$. We now give the system equations.

% Equation number stuff
%-----------------------------------------------------------------------------------------------------------------------------------------------------------------
\newcounter{oldequation}
\setcounter{oldequation}{\theequation}
\setcounter{equation}{0}
\renewcommand{\theequation}{H\arabic{equation}}
%-----------------------------------------------------------------------------------------------------------------------------------------------------------------

The mass conservation equations are given by 
\begin{subequations}
\begin{align}
\stdH{1}: && \density{\icell}{\itime} & =\density{\icell}{\itime-1}+\frac{\deltat}{\deltax}\left(\flowin{\icell}{\itime-1}-\flowout{\icell}{\itime-1}\right) & \forall\icell\in\left\{ 1,\ldots,\ncell-1\right\} ,\itime\in\left\{ 1,\ldots,\ntime\right\}
\label{eq:conservation1}
\\
\H{1}{\itime}{0}: && \density 0{\itime} & =\density 0{\itime-1}+\frac{\deltat}{\deltax}\left(\inputflux 0{\itime-1}-\flowout 0{\itime-1}\right) & \forall\itime\in\left\{ 1,\ldots,\ntime\right\}
\label{eq:conservation2}
\\
\H{1}{\itime}{\ncell}: && \density{\ncell}{\itime} & =\density{\ncell}{\itime-1}+\frac{\deltat}{\deltax}\left(\mbox{\ensuremath{\flowin{\ncell}{\itime-1}}}-\celldemand{\ncell}{\itime-1}\right) & \forall\itime\in\left\{ 1,\ldots,\ntime\right\}
\label{eq:conservation3}
\end{align}
\label{eq:conservation}
\end{subequations}
and initial condition 
\begin{align}
\H{0}{1}{\icell} : \density{\icell}0 =\densityinit{\icell} && \forall\icell\in\left\{ 0,\ldots,\ncell\right\}
\tag{I1}
\label{eq:densityInit}
\end{align}


The car count on ramp $i$ is given by 
\begin{align}
\H{2}{\itime}{\icell}: && \rampqueue{\icell}{\itime} & =\rampqueue{\icell}{\itime-1}+\deltat\left(\inputflux{\icell}{\itime-1}-\rampflow{\icell}{\itime-1}\right) & \forall\icell\in\left\{ 1,\ldots,\ncell-1\right\} ,\itime\in\left\{ 1,\ldots,\ntime\right\}
\label{eq:rampConservation}
\\
\H{2}{0}{\icell}: && \rampqueue{\icell}0 & =\rampqueueinit{\icell} & \forall\icell\in\left\{ 1,\ldots,\ncell-1\right\}
\tag{I2}\label{eq:rampInit}
\end{align}


At junctions, the flows are given by the solver described in section~\ref{sec:junction_solver}. The flows can be determined by first computing the demand function (equation~$\H{3}{\itime}{\icell}$) and the supply function (equation~$\H{4}{\itime}{\icell}$) for the mainline, the demand function for the ramp (equation~$\H{5}{\itime}{\icell}$), then the total flow through the junction (equation~$\H{6}{\itime}{\icell}$).
\begin{align}
\stdH{3}: && \celldemand{\icell}{\itime} & =\min\left(\fmax{\icell},\ffspeed{\icell}\density{\icell}{\itime}\right) & \forall\icell\in\left\{ 0,\ldots,\ncell\right\} ,\itime\in\left\{ 0,\ldots,\ntime-1\right\}
\label{eq:junctionDemand} \\
\stdH{4}: && \cellsupply{\icell}{\itime} & =\min\left(\fmax{\icell},\congspeed{\icell}\left(\jamdensity{\icell}-\density{\icell}{\itime}\right)\right) & \forall\icell\in\left\{ 1,\ldots,\ncell\right\} ,\itime\in\left\{ 0,\ldots,\ntime-1\right\} 
\label{eq:junctionSupply} \\
\stdH{5}: && \rampdemand{\icell}{\itime} & =\min\left(\rampqueue{\icell}{\itime},\rampcontrol{\icell}{\itime}\right) & \forall\icell\in\left\{ 1,\ldots,\ncell-1\right\} ,\itime\in\left\{ 0,\ldots,\ntime-1\right\} 
\label{eq:junctionRampDemand}
\end{align}
\begin{subequations}
\begin{align}
\stdH{6}: && \flowin{\icell}{\itime} & =\min\left(\celldemand{\icell-1}{\itime}\left(1-\offrampratio{\icell-1}{\itime}\right)+\rampdemand{\icell-1}{\itime},\cellsupply{\icell}{\itime}\right) & \forall\icell\in\left\{ 2,\ldots,\ncell\right\} ,\itime\in\left\{ 0,\ldots,\ntime-1\right\}
\label{eq:junctionFlowMaximization} \\
\H{6}{\itime}{1}: && \flowin{1}{\itime} & =\min\left(\celldemand{0}{\itime},\cellsupply{1}{\itime}\right) &
\forall\itime\in\left\{ 0,\ldots,\ntime-1\right\}
\label{eq:junctionFlowConservationSpecial2}	
\end{align}
\label{eq:juncmaxgroup}
\end{subequations}


When there is an actual offramp at the junction (i.e. $\offrampratio{\icell}{\itime} > 0$), the flow is uniquely determined by flow maximization across the junction (see fig.~\ref{fig:junctionFlows2}). When the split ratio $\offrampratio{\icell}{\itime} = 0$ (equivalently, when there is no off-ramp), the solution of the junction problem may not be unique. In order to guarantee uniqueness of the solution, we use a fixed%
\footnote{comment about fixed priority and proportional priority}
priority vector~\cite{garavello2006traffic} given by $\priority{\icell}$ for
junction~$i$. The unique solution is given for all cases (see Fig.~\ref{fig:junctionFlows})
\begin{subequations}
\begin{multline}
\stdH{7}: 
\flowout{\icell}{\itime}
=\begin{cases}
\flowin{\icell+1}{\itime}/\left(1-\offrampratio{\icell}{\itime}\right)
& \text{if } (\R{1}{\itime}{\icell}) : \offrampratio{\icell}{\itime} > 0 \text{ and } \flowin{\icell+1}{\itime} < \left(1 - \offrampratio{\icell}{\itime}\right) \celldemand{\icell}{\itime}
\\
\celldemand{\icell}{\itime}
& \text{if } (\R{2}{\itime}{\icell}) : \offrampratio{\icell}{\itime} > 0 \text{ and }\flowin{\icell+1}{\itime} \ge \left(1 - \offrampratio{\icell}{\itime}\right) \celldemand{\icell}{\itime}
\\
\celldemand{\icell}{\itime} 
& \text{if } (\R{3}{\itime}{\icell}) : \offrampratio{\icell}{\itime} = 0 \text{ and } \frac{\priority{\icell}}{1-\priority{\icell}} > \frac{\celldemand{\icell}{\itime}}{\flowin{\icell+1}{\itime} - \celldemand{\icell}{\itime}}
\\
\flowin{\icell+1}{\itime}-\rampdemand{\icell}{\itime}
& \text{if } (\R{4}{\itime}{\icell}) : \offrampratio{\icell}{\itime} = 0 \text{ and } \frac{\priority{\icell}}{1-\priority{\icell}} < \frac{\flowin{\icell+1}{\itime}-\rampdemand{\icell}{\itime}}{\rampdemand{\icell}{\itime}}
\\
\priority{\icell}\flowin{\icell+1}{\itime} & \text{otherwise } (\R{5}{\itime}{\icell})
\end{cases} \\
\forall\icell\in\left\{ 1,\ldots,\ncell-1\right\} ,\itime\in\left\{ 0,\ldots,\ntime-1\right\}
\label{eq:junctionPriority}
\end{multline}
\begin{align}
\H{7}{\itime}{0}: && \flowout{0}{\itime} & =\flowin{1}{\itime} &
\forall \itime \in \left \{ 0, \dots, \ntime-1 \right\}
\label{eq:junctionFlowConservationSpecial1}	
\end{align}
\end{subequations}
where Equation~\eqref{eq:junctionFlowConservationSpecial1} is a special case for the flow out at the source dummy cell.
Here we use $\R{1}{\itime}{\icell}, \dots, \R{5}{\itime}{\icell}$ to denote the sets of state vectors that satisfy the corresponding condition. They form a partition of $X$. This will be a useful notation in the adjoint system expression, where we will use the indicator function $1_{\R{1}{\itime}{\icell}}$, with implicit argument $x$, to denote the function
\[
\begin{aligned}
1_{\R{1}{\itime}{\icell}} : X &\rightarrow \{0, 1\} \\
x & \mapsto \begin{cases} 1 & \text{ if } x \in \R{1}{\itime}{\icell} \\ 0 & \text{ otherwise }\end{cases}
\end{aligned}
\]

Finally, the ramp flow is simply given by the conservation of flows:
\begin{align}
\stdH{8}: && \rampflow{\icell}{\itime} & = \flowin{\icell+1}{\itime} - \flowout{\icell}{\itime}\left(1-\offrampratio{\icell}{\itime}\right)& \forall\icell\in\left\{ 1,\ldots,\ncell-1\right\} ,\itime\in\left\{ 0,\ldots,\ntime-1\right\}
\label{eq:junctionFlowConservation}
\end{align}

\setcounter{equation}{6}

\subsubsection{Modified Piccoli model}
\label{sec:newModel}
We make two changes to the standard Piccoli 2x2 junction model to fix the two following issues:
\begin{enumerate}
\item \textbf{Loss of boundary flows}\\
The boundary conditions at the sources are only satisfied in a weak sense. This means that backwards moving shock waves passing through the sources (i.e. congestion at the sources) can result in boundary inflows being lost at the sources. To fix this problem we introduce a infinite capacity buffer at each source link of the network. See section~\ref{sec:contBufferModel} for details on the continuous time model formulation. The discrete system equations above already incorporate this change.
\item \textbf{Complete blocking of on-ramps when the mainline in congested at a 2x2 junction}\\
A major modeling deficiency of the Piccoli model is that the flow maximization condition will completely block off the on-ramp at a junction with a congested downstream link. This is due to the fact that incoming flow from the on-ramp can only be distributed to the mainline, while incoming flow from the main can be distributed to both the mainline and the off-ramp. Thus, junction flow maximization dictates that all of the demand from the upstream mainline link should be satisfied before any of the on-ramp demand is satisfied. Moreover, in any situation where the junction is supply constrained and an off-ramp exists, the mainline demand is served first. This is a bad modeling choice, since the demand allocation in reality depends on the number of lanes available for each inflow. 
To fix this problem, we modify the junction solver to maximize flow that leaves the junction on the mainline instead of maximizing the total flow out of the junction (which includes the off-ramp). This change makes the solution non-unique with regards to in which ratio the demands are allocated to the available supply in a supply constrained situation. Therefore, we reintroduce a inflow priority parameter even in the case where an off-ramp exists. Please see section~\ref{sec:contBufferModel} for an proof of the uniqueness, existence and self-similarity of the modified junction solver in the continuous case. The resulting new system equations for the junction in the discrete setting are given below: 
\end{enumerate}


\begin{subequations}
\begin{multline}
\stdH{7}: 
\flowout{\icell}{\itime}
=\begin{cases}
\celldemand{\icell}{\itime} 
& \text{if } (\R{1}{\itime}{\icell}) : \priority{\icell}\flowin{\icell+1}{\itime} > \left(1-\offrampratio{\icell}{\itime}\right)\celldemand{\icell}{\itime}
\\
\frac{\flowin{\icell+1}{\itime}-\rampdemand{\icell}{\itime}}{1-\offrampratio{\icell}{\itime}}
& \text{if } (\R{2}{\itime}{\icell}) : \left(1-\priority{\icell}\right)\flowin{\icell+1}{\itime} > \rampdemand{\icell}{\itime}
\\
\frac{\priority{\icell}\flowin{\icell+1}{\itime}}{1-\offrampratio{\icell}{\itime}} & \text{otherwise } (\R{3}{\itime}{\icell})
\end{cases} \\
\forall\icell\in\left\{ 1,\ldots,\ncell-1\right\} ,\itime\in\left\{ 0,\ldots,\ntime-1\right\}
\label{eq:junctionPriority}
\end{multline}
\begin{align}
\H{7}{\itime}{0}: && \flowout{0}{\itime} & =\flowin{1}{\itime} &
\forall \itime \in \left \{ 0, \dots, \ntime-1 \right\}
\label{eq:junctionFlowConservationSpecial1}	
\end{align}
\end{subequations}




% \subsubsection{Uniqueness of junction flows}

% Equation number stuff: resume normal numbering
%-----------------------------------------------------------------------------------------------------------------------------------------------------------------
\setcounter{equation}{\theoldequation}
\renewcommand{\theequation}{\arabic{equation}}
%-----------------------------------------------------------------------------------------------------------------------------------------------------------------

\begin{figure}[h]
\centering
\subfloat[$\offrampratio{\icell}{\itime} = 0$]{
\resizebox{.5\columnwidth}{!}{
	\documentclass[10pt]{article}
\usepackage{calc}
\usepackage[usenames,dvipsnames]{xcolor}
\definecolor{lightgray}{gray}{0.9}

\usepackage{tikz}
\usetikzlibrary{calc}
\usepackage{xparse}
\usepackage{etoolbox}
\usepackage[graphics, tightpage, active]{preview}
\usepackage{amsmath}


\usetikzlibrary{decorations.pathreplacing}


\setlength{\PreviewBorder}{2pt}
\PreviewEnvironment{tikzpicture}


\input{../commands.tex}


%============================================================================================
%Additional commands

\newcommand \trim[4]{
\renewcommand \PreviewBbAdjust{#1 #2 #3 #4}
}

\newcommand \scale{2}
\newcommand \bound[1]{\Big(1 - \frac{\maxR{#1}}{r} \Big)}
\newcommand \tick[2]{
(#2, #1) -- (#2, -#1)
}

\newcommand \smalltick[1]{\tick{0.01}{#1}}
\newcommand \largetick[1]{[line width=1pt]\tick{0.02}{#1}}


%============================================================================================
\begin{document}
%============================================================================================

\begin{tikzpicture}[scale=\scale,domain=0:1]

\def \rampDem{2}
\def \dem{2.5}
\def \priorityRat{0.7}
\def \totalFlow{3}

\coordinate (Z) at (0,0);
\coordinate (I1) at (\rampDem, 0);
\coordinate (I2) at (0, \dem);
\coordinate (I3) at (\rampDem, \dem);
\coordinate (A) at (0, {\totalFlow});
\coordinate (B) at (\totalFlow, 0);
\coordinate (C) at ({(1-\priorityRat)/\priorityRat*\dem}, \dem);
\coordinate (P1) at (intersection of I1--I3 and A--B);
\coordinate (P2) at (intersection of I2--I3 and A--B);


\draw[->] (Z) -- (3.5,0) node[below right]{$\rampflow{\icell}{\itime}$};
\draw[->] (Z) -- (0,4) node[left]{$\flowout{\icell}{\itime}$};
\draw[dashed] (I3) -- (I1) node[below]{$\rampdemand{\icell}{\itime}$};
\draw[dashed] (I3) -- (I2) node[left]{$\celldemand{\icell}{\itime}$};
\draw (A) -- (B) node[yshift=1cm, xshift=0cm]{$\flowout{\icell}{\itime} + \rampflow{\icell}{\itime} = \flowin{\icell+1}{\itime}$};
\draw (Z) -- (C) node[above, xshift=1cm]{$\flowout{\icell}{\itime} = \frac{\priority{\icell}}{1-\priority{\icell}} \rampflow{\icell}{\itime}$};
\draw (intersection of A--B and Z--C) circle (1pt);
\draw[densely dotted] (Z)--(P1);
\draw[densely dotted] (Z)--(P2);

\end{tikzpicture}


\end{document}

}
\label{fig:junctionFlows1}
}
\subfloat[$\offrampratio{\icell}{\itime} \neq 0$]{
\resizebox{.5\columnwidth}{!}{
	\documentclass[10pt]{article}
\usepackage{calc}
\usepackage[usenames,dvipsnames]{xcolor}
\definecolor{lightgray}{gray}{0.9}

\usepackage{tikz}
\usetikzlibrary{calc}
\usepackage{xparse}
\usepackage{etoolbox}
\usepackage[graphics, tightpage, active]{preview}
\usepackage{amsmath}


\usetikzlibrary{decorations.pathreplacing}


\setlength{\PreviewBorder}{2pt}
\PreviewEnvironment{tikzpicture}


\input{../commands.tex}


%============================================================================================
%Additional commands

\newcommand \trim[4]{
\renewcommand \PreviewBbAdjust{#1 #2 #3 #4}
}

\newcommand \scale{2}
\newcommand \bound[1]{\Big(1 - \frac{\maxR{#1}}{r} \Big)}
\newcommand \tick[2]{
(#2, #1) -- (#2, -#1)
}

\newcommand \smalltick[1]{\tick{0.01}{#1}}
\newcommand \largetick[1]{[line width=1pt]\tick{0.02}{#1}}

\tikzset{dashdot/.style={dash pattern=on 2pt off 3pt on 4pt off 3pt}}

\makeatletter
\newcommand{\gettikzxy}[3]{%
  \tikz@scan@one@point\pgfutil@firstofone#1\relax
  \edef#2{\the\pgf@x}%
  \edef#3{\the\pgf@y}%
}
\makeatother

%============================================================================================
\begin{document}
%============================================================================================

\begin{tikzpicture}[scale=\scale,domain=0:1]

\def \rampDem{2}
\def \dem{2.5}
\def \priorityRat{0.7}
\def \splitRat{0.4}
\def \totalFlow{2.1}

\coordinate (Z) at (0,0);
\coordinate (I1) at (\rampDem, 0);
\coordinate (I2) at (0, \dem);
\coordinate (I3) at (\rampDem, \dem);
\coordinate (A) at (0, {\totalFlow/(1-\splitRat)});
\coordinate (B) at (\totalFlow, 0);
\coordinate (C) at ({(1-\priorityRat)/\priorityRat*\dem}, \dem);
\coordinate (P) at (intersection of A--B and I2--I3);

\draw[->] (Z) -- (3.5,0) node[below right]{$\rampflow{\icell}{\itime}$};
\draw[->] (Z) -- (0,4) node[left]{$\flowout{\icell}{\itime}$};
\draw[dashed] (I3) -- (I1) node[below]{$\rampdemand{\icell}{\itime}$};
\draw[dashed] (I3) -- (I2) node[left]{$\celldemand{\icell}{\itime}$};
\draw (A) -- (B) node[yshift=1cm, xshift=1cm]{$(1-\offrampratio{\icell}{\itime})\flowout{\icell}{\itime} + \rampflow{\icell}{\itime} = \flowin{\icell+1}{\itime}$};
\draw (P) circle (1pt);

\gettikzxy{(P)}{\Px}{\Py}
\draw[dashdot, color=Green] (0, {\Px + \Py}) -- ({\Px + \Py}, 0) node [xshift=-0cm, yshift=2cm]{$\flowout{\icell}{\itime} + \rampflow{\icell}{\itime} = const$};

\end{tikzpicture}


\end{document}

}
\label{fig:junctionFlows2}
}
\caption{Junction flows}
\label{fig:junctionFlows}
\end{figure}

%-----------------------------------------------------------------------------------------------------------------------------------------------------------------

%-----------------------------------------------------------------------------------------------------------------------------------------------------------------
\subsection{Ordering of state vector and constraints}

% !TEX root = rampMeteringViaTheAdjoint.tex
\label{sec:ordering}

In order create a lower triangular $\frac{\partial H}{\partial x}$
matrix, we must have the proper ordering of the variables in the $\mathbf{x}$
vector. This process is similar to understanding the nature of the
dependencies in the forward simulation. Given the boundary and initial
conditions, the discretized system can be exactly solved by solving
the constraints in a particular order. From this consideration, we
create a ``topological ordering'' of the variables, or the order
in which the variables must be solved from the particular constraints.

\begin{align}
\density{\icell}0,\rampqueue j0 & & & \icell\in\left\{ 0,\ldots,\ncell\right\} ,j\in\left\{ 1,\ldots,\ncell-1\right\} \nonumber \\
\celldemand i0,\cellsupply j0,\rampdemand k0 & & & \icell\in\left\{ 0,\ldots,\ncell\right\} ,j\in\left\{ 1,\ldots,\ncell\right\} ,k\in\left\{ 1,\ldots,\ncell-1\right\} \nonumber \\
\flowin i0 & & & \icell\in\left\{ 1,\ldots,\ncell\right\} \nonumber \\
\flowout i0 & & & i\in\left\{ 0,\ldots,\ncell\right\} \nonumber \\
\rampflow i0 & & & i\in\left\{ 1,\ldots,\ncell-1\right\} \nonumber \\
\density{\icell}1,\rampqueue j1 & & & \icell\in\left\{ 0,\ldots,\ncell\right\} ,j\in\left\{ 1,\ldots,\ncell-1\right\} \nonumber \\
\ldots\nonumber \\
\rampflow ik & & & i\in\left\{ 1,\ldots,\ncell-1\right\} \nonumber \\
\density{\icell}{k+1},\rampqueue j{k+1} & & & \icell\in\left\{ 0,\ldots,\ncell\right\} ,j\in\left\{ 1,\ldots,\ncell-1\right\} \label{eq:topological-ordering}\\
\ldots\nonumber \\
\rampflow i{\ntime} & & & i\in\left\{ 1,\ldots,\ncell-1\right\} \nonumber 
\end{align}


The ordering has three nested loops: first over time index, then over ``variable type'', then over cell index. Each row in the set of equations only depends upon variables in above
rows, or more precisely, they will only depend on variables in rows
above \emph{and} on the same time step $\itime$ or $\itime-1$.

The constraints in the system equations section were presented in
a particular way, in order to have every variable in the system appear
alone on the LHS of exactly one equation, and have the RHS of each
equation only be a function of variables with a ``lower'' topological
order. Therefore, it is clear that if we order the $\mathbf{x}$ variables
according to Equation~\ref{eq:topological-ordering}, and order the
constraints according to when the LHS variable of the constraint appears
in $\mathbf{x}$, then the $\frac{\partial H}{\partial x}$ matrix
will have a lower-triangular structure. Additionally, there will be
all $1$'s on the diagonal.

When the adjoint system is considered, the transpose of the $\frac{\partial H}{\partial x}$
matrix is taken, thus creating an upper-triangular structure. Such
a system can be solved backwards in time, as opposed to the original
system being solved forward in time from the initial conditions.

Note that some of the constraints contain special cases for initial
conditions and boundary conditions, but it should be clear to make
the 1-to-1 mapping with the variables based on the time and space
indices being considered.


%-----------------------------------------------------------------------------------------------------------------------------------------------------------------

\subsection{Forward simulation}
% !TEX root = SO-DTA with partial compliance.tex


%-----------------------------------------------------------------------------------------------------------------------------------------------------------------
\subsection{Objective function}
TODO: generalize this to include total travel distance.\\
TODO: cars have to be flushed out of the network at the end of the $\ntime$ time steps. Discuss how this is done.

The objective function is the total travel time, given by
\[
J(x, u) = \sum_{\itime = 0}^{\ntime} \left( \sum_{\icell = 0}^{\ncell} \density{\icell}{\itime} + \sum_{\icell = 1}^{\ncell - 1} \rampqueue{\icell}{\itime} \right)
\]


%-----------------------------------------------------------------------------------------------------------------------------------------------------------------
\subsection{Adjoint system}
% !TEX root = rampMeteringViaTheAdjoint.tex

The adjoint system is given by $\frac{\partial J}{\partial x}^T + \frac{\partial H}{\partial x}^T\lambda = 0$. We observe that in our case, the forward system is affine in $x$ around any given point $(x^{(0)}, u^{(0)})$, and can be written in the form
\begin{equation}
H(x, u^{(0)}) = A(x^{(0)}, u^{(0)}) x + b(x^{(0)}, u^{(0)})
\end{equation}
where $A(x^{(0)}, u^{(0)})$ is a matrix and $b(x^{(0)}, u^{(0)})$ is a vector that are entirely determined by the state $(x^{(0)}, u^{(0)})$, and can be computed during the forward simulation step. The adjoint system is then given simply by
$A(x^{(0)}, u^{(0)})^T \lambda + \frac{\partial J}{\partial x} = 0$. Since $A(x^{(0)}, u^{(0)})$ is lower-triangular by construction (see Section~\ref{sec:ordering}), the adjoint system is upper triangular and can be solved backwards in time. Next, we explicitly give the adjoint system in equation form (not in matrix form due to the large size of the matrix)

\subsubsection{Adjoint equations}

% Equation number stuff
%-----------------------------------------------------------------------------------------------------------------------------------------------------------------
\setcounter{oldequation}{\theequation}
\setcounter{equation}{0}
\renewcommand{\theequation}{G\arabic{equation}}
%-----------------------------------------------------------------------------------------------------------------------------------------------------------------


The adjoint equations are given by $\frac{\partial J}{\partial x} + \lambda^T\frac{\partial H}{\partial x} = 0$. This can be rewritten formally as
\[
\frac{\partial J}{\partial x} + \sum_{\itime, \iconst, \icell} \sstdLambd \frac{\partial \sstdH}{\partial x} = 0
\]
where $\iconst \in \{1, \dots, \nconst\}$ indexes the type of constraint, $\icell$ and $\itime$ are in the appropriate ranges, i.e. wherever $\sstdH$ is defined. Here we defined one adjoint variable $\sstdLambd$ for each constraint $\sstdH$.

We give the equation corresponding to each variable $x$. In the formulation below, we distinguish final conditions that correspond to the final time step $\ntime$, namely equations~(F), from the rest of the equations, denoted by~(G).

% density
Taking the partial derivative with respect to the density variable $\density{\icell}{\itime}$, we obtain the following equations
\begin{subequations}
\begin{align}
\G{1}{\ntime}{\icell}: && 0 &= 1 - \Lambd{1}{\ntime}{\icell}
& \forall \icell \in \{ 0, \dots, \ncell \}, \itime = \ntime
\tag{F1}
\label{eq:}
\\
\G{1}{\itime}{0}: && 0 &= 1 - \Lambd{1}{\itime}{0} + \Lambd{1}{\itime+1}{0} + \Lambd{3}{\itime}{0} \ffspeed{0} \ind{\ffspeed{0} \density{0}{\itime} < \fmax{0} }
& \icell = 0, \forall \itime \in \{0, \ntime - 1\}
\label{eq:}
\end{align}
\begin{multline}
\stdG{1}: 0 = 1 - \Lambd{1}{\itime}{\icell} + \Lambd{1}{\itime+1}{\icell} + \stdLambd{3} \ffspeed{\icell} \ind{\ffspeed{\icell}\density{\icell}{\itime}<\fmax{\icell}} - \stdLambd{4} \congspeed{\icell} \ind{\congspeed{\icell}(\jamdensity{\icell} - \density{\icell}{\itime}) < \fmax{\icell}}
\\
\forall \icell \in \{ 1, \dots, \ncell \}, \forall \itime \in \{0, \ntime - 1\}
\label{eq:}
\end{multline}
\end{subequations}

% ramp count
The equations corresponding to the partial derivative with respect to the ramp queue $\rampqueue{\icell}{\itime}$ are
\begin{align}
\G{2}{\ntime}{\icell}: && 0 &= 1 - \Lambd{2}{\ntime}{\icell}
& \forall \icell \in \{ 1, \dots, \ncell-1 \}, \itime = \ntime
\tag{F2}
\label{eq:}
\\
\stdG{2}: && 0 &= 1 - \Lambd{2}{\itime}{\icell} + \Lambd{2}{\itime+1}{\icell} + \Lambd{5}{\itime}{\icell} \ind{\rampqueue{\icell}{\itime} < \rampcontrol{\icell}{\itime}}
& \forall \icell \in \{1, \dots, \ncell - 1\}, \forall \itime \in \{0, \ntime - 1\}
\label{eq:}
\end{align}

% demand
The equations corresponding to the partial derivative with respect to the demand $\celldemand{\icell}{\itime}$ are
\begin{subequations}
\begin{multline}
\stdG{3}: 0 = - \Lambd{3}{\itime}{\icell} + (1-\offrampratio{\icell}{\itime})\Lambd{6}{\itime}{\icell+1} 
\ind{ (1-\offrampratio{\icell}{\itime})\celldemand{\icell}{\itime} + \rampdemand{\icell}{\itime} < \cellsupply{\icell+1}{\itime} } + 
\Lambd{7}{\itime}{\icell} 1_{\R{2}{\itime}{\icell} \cup \R{3}{\itime}{\icell}}
\\
\icell \in \{1, \dots, \ncell - 1\}, \forall \itime \in \{0, \ntime - 1\}
\label{eq:}
\end{multline}
%
\begin{align}
\G{3}{\itime}{0}: && 0 &= - \Lambd{3}{\itime}{\icell} + \Lambd{6}{\itime}{1}
& \icell = 0, \forall \itime \in \{0, \ntime - 1\}
\label{eq:}
\\
\G{3}{\itime}{\ncell}: && 0 &= - \Lambd{1}{\itime+1}{\ncell} - \Lambd{3}{\itime}{\ncell}
& \icell = \ncell, \forall \itime \in \{0, \ntime - 1\}
\label{eq:}
\end{align}
\end{subequations}

%supply
The equations corresponding to the partial derivative with respect to the supply $\cellsupply{\icell}{\itime}$ are
\begin{subequations}
\begin{align}
\stdG{4}: && 0 &= - \Lambd{4}{\itime}{\icell} + \Lambd{6}{\itime}{\icell} 
\ind{ \cellsupply{\icell}{\itime} < (1-\offrampratio{\icell-1}{\itime})\celldemand{\icell-1}{\itime} + \rampdemand{\icell-1}{\itime} }
& \icell \in \{2, \dots, \ncell\}, \forall \itime \in \{0, \ntime - 1\}
\label{eq:}
\\
\G{4}{\itime}{1}: && 0 &= - \Lambd{4}{\itime}{1} + \Lambd{6}{\itime}{1} 
\ind{ \cellsupply{1}{\itime} < \celldemand{0}{\itime} }
& \icell = 1, \forall \itime \in \{0, \ntime - 1\}
\label{eq:}
\end{align}
\end{subequations}

% ramp demand
The equations corresponding to the partial derivative with respect to the ramp demand $\rampdemand{\icell}{\itime}$ are
\begin{multline}
\stdG{5}: 0 = - \Lambd{5}{\itime}{\ncell} + \Lambd{6}{\itime}{\icell+1}
\ind{ (1-\offrampratio{\icell}{\itime})\celldemand{\icell}{\itime} + \rampdemand{\icell}{\itime} < \cellsupply{\icell+1}{\itime} } - 
\Lambd{7}{\itime}{\icell} 1_{\R{4}{\itime}{\icell}}
\\ \icell \in \{1, \dots, \ncell - 1\}, \forall \itime \in \{0, \ntime - 1\}
\label{eq:}
\end{multline}

%flow in
The equations corresponding to the partial derivative with respect to the in-flux $\flowin{\icell}{\itime}$ are
%
\begin{subequations}
\begin{multline}
\stdG{6}: 0 = \frac{\deltat}{\deltax} \Lambd{1}{\itime+1}{\ncell} - \Lambd{6}{\itime}{\icell} + \Lambd{7}{\itime}{\icell-1} \left[ \frac{1}{1-\offrampratio{\icell - 1}{\itime}} 1_{\R{1}{\itime}{\icell-1}} + 1_{\R{4}{\itime}{\icell-1}} + \priority{\icell} 1_{\R{5}{\itime}{\icell-1}} \right] + \Lambd{8}{\itime}{\icell-1}
\\ \icell \in \{2, \dots, \ncell \}, \forall \itime \in \{0, \ntime - 1\}
\label{eq:}
\end{multline}
%
\begin{align}
\G{6}{\itime}{1}: && 0 = \frac{\deltat}{\deltax} \Lambd{1}{\itime+1}{1} - \Lambd{6}{\itime}{1} + \Lambd{7}{\itime}{0}
&& \icell = 1, \forall \itime \in \{0, \ntime - 1\}
\label{eq:}
\end{align}
\end{subequations}

% flow out
The equations corresponding to the partial derivative with respect to the out-flux $\flowout{\icell}{\itime}$ are
%
\begin{subequations}
\begin{align}
\stdG{7}: && 0 &= - \frac{\deltat}{\deltax} \Lambd{1}{\itime+1}{\icell} - \stdLambd{7} - \Lambd{8}{\itime}{\icell}  (1 - \offrampratio{\icell}{\itime})
& \icell \in \{1, \dots, \ncell - 1\}, \forall \itime \in \{0, \ntime - 1\}
\label{eq:}
\\
\G{7}{\itime}{0}: && 0 &= - \frac{\deltat}{\deltax} \Lambd{1}{\itime+1}{0} - \Lambd{7}{\itime}{0}
& \icell = 0, \forall \itime \in \{0, \ntime - 1\}
\label{eq:}
\end{align}
\end{subequations}

% ramp flow
The equations corresponding to the partial derivative with respect to the on-ramp flux $\rampflow{\icell}{\itime}$ are
%
\begin{align}
\stdG{8}: && 0 &= - \deltat \Lambd{2}{\itime+1}{\icell} + \stdLambd{8}
& \icell \in \{1, \dots, \ncell - 1\}, \forall \itime \in \{0, \ntime - 1\}
\label{eq:}
\end{align}

As mentioned above, the adjoint linear system is upper triangular, and can be solved backwards in time, starting from the last time step $\ntime$ (i.e. solve equations $\G{1}{\ntime}{\icell}$ and $\G{2}{\ntime}{\icell}$). At each time step, the system is solved for decreasing $h$ (starting from $h=8$) then for decreasing $\icell$.

\subsubsection{Changes for the modified Piccoli model}
In the modified Piccoli model the equations corresponding to the partial derivative with respect to the in-flux $\flowin{\icell}{\itime}$ changes as follows.
%
% adjust counter
\setcounter{equation}{5}

\begin{subequations}
\begin{multline}
\stdG{6}: 0 = \frac{\deltat}{\deltax} \Lambd{1}{\itime+1}{\ncell} - \Lambd{6}{\itime}{\icell} + \Lambd{7}{\itime}{\icell-1} \left[ \frac{1}{1-\offrampratio{\icell - 1}{\itime}} 1_{\R{2}{\itime}{\icell-1}} +  \frac{\priority{\icell}}{1-\offrampratio{\icell - 1}{\itime}} 1_{\R{3}{\itime}{\icell-1}} \right] + \Lambd{8}{\itime}{\icell-1}
\\ \icell \in \{2, \dots, \ncell \}, \forall \itime \in \{0, \ntime - 1\}
\end{multline}
%
\begin{align}
\G{6}{\itime}{1}: && 0 = \frac{\deltat}{\deltax} \Lambd{1}{\itime+1}{1} - \Lambd{6}{\itime}{1} + \Lambd{7}{\itime}{0}
&& \icell = 1, \forall \itime \in \{0, \ntime - 1\}
\end{align}
\end{subequations}

where $\R{2}{\itime}{\icell-1}$ and $\R{3}{\itime}{\icell-1}$ refer to the new conditions defined in section~\ref{sec:newModel}.



% Equation number stuff: resume normal numbering
%-----------------------------------------------------------------------------------------------------------------------------------------------------------------
\setcounter{equation}{\theoldequation}
\renewcommand{\theequation}{\arabic{equation}}
%-----------------------------------------------------------------------------------------------------------------------------------------------------------------

% !TEX root = rampMeteringViaTheAdjoint.tex







%-----------------------------------------------------------------------------------------------------------------------------------------------------------------
\subsection{Computing the gradient}
\label{sec:gradient}
After solving for $\lambda$ (adjoint system), we can compute the gradient, given formally by
\[
\nabla_u J (x, u)
= \frac{\partial J}{\partial u} + \lambda^T \frac{\partial H}{\partial u}
\]

Since the objective function $J$ does not depend on the control parameter $u$, we have
\begin{equation*}
\frac{\partial J}{\partial u} = 0
\end{equation*}

The system constraints $H$ have only one equation \ref{eq:junctionRampDemand} that depends on the control $u$. Therefore, the derivative of all other constraints with respect to the control $u$ is zero.
\begin{equation}
\frac{\partial \H{5}{\itime}{\icell} }{\partial u_i(k)} = -\ind{l_i(k) > u_i(k)}
\end{equation}

