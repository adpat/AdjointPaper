% !TEX root = rampMeteringViaTheAdjoint.tex

The adjoint system is given by $\frac{\partial J}{\partial x}^T + \frac{\partial H}{\partial x}^T\lambda = 0$. We observe that in our case, the forward system is affine in $x$ around any given point $(x^{(0)}, u^{(0)})$, and can be written in the form
\begin{equation}
H(x, u^{(0)}) = A(x^{(0)}, u^{(0)}) x + b(x^{(0)}, u^{(0)})
\end{equation}
where $A(x^{(0)}, u^{(0)})$ is a matrix and $b(x^{(0)}, u^{(0)})$ is a vector that are entirely determined by the state $(x^{(0)}, u^{(0)})$, and can be computed during the forward simulation step. The adjoint system is then given simply by
$A(x^{(0)}, u^{(0)})^T \lambda + \frac{\partial J}{\partial x} = 0$. Since $A(x^{(0)}, u^{(0)})$ is lower-triangular by construction (see Section~\ref{sec:ordering}), the adjoint system is upper triangular and can be solved backwards in time. Next, we explicitly give the adjoint system in equation form (not in matrix form due to the large size of the matrix)

\subsubsection{Adjoint equations}

% Equation number stuff
%-----------------------------------------------------------------------------------------------------------------------------------------------------------------
\setcounter{oldequation}{\theequation}
\setcounter{equation}{0}
\renewcommand{\theequation}{G\arabic{equation}}
%-----------------------------------------------------------------------------------------------------------------------------------------------------------------


The adjoint equations are given by $\frac{\partial J}{\partial x} + \lambda^T\frac{\partial H}{\partial x} = 0$. This can be rewritten formally as
\[
\frac{\partial J}{\partial x} + \sum_{\itime, \iconst, \icell} \sstdLambd \frac{\partial \sstdH}{\partial x} = 0
\]
where $\iconst \in \{1, \dots, \nconst\}$ indexes the type of constraint, $\icell$ and $\itime$ are in the appropriate ranges, i.e. wherever $\sstdH$ is defined. Here we defined one adjoint variable $\sstdLambd$ for each constraint $\sstdH$.

We give the equation corresponding to each variable $x$. In the formulation below, we distinguish final conditions that correspond to the final time step $\ntime$, namely equations~(F), from the rest of the equations, denoted by~(G).

% density
Taking the partial derivative with respect to the density variable $\density{\icell}{\itime}$, we obtain the following equations
\begin{subequations}
\begin{align}
\G{1}{\ntime}{\icell}: && 0 &= 1 - \Lambd{1}{\ntime}{\icell}
& \forall \icell \in \{ 0, \dots, \ncell \}, \itime = \ntime
\tag{F1}
\label{eq:}
\\
\G{1}{\itime}{0}: && 0 &= 1 - \Lambd{1}{\itime}{0} + \Lambd{1}{\itime+1}{0} + \Lambd{3}{\itime}{0} \ffspeed{0} \ind{\ffspeed{0} \density{0}{\itime} < \fmax{0} }
& \icell = 0, \forall \itime \in \{0, \ntime - 1\}
\label{eq:}
\end{align}
\begin{multline}
\stdG{1}: 0 = 1 - \Lambd{1}{\itime}{\icell} + \Lambd{1}{\itime+1}{\icell} + \stdLambd{3} \ffspeed{\icell} \ind{\ffspeed{\icell}\density{\icell}{\itime}<\fmax{\icell}} - \stdLambd{4} \congspeed{\icell} \ind{\congspeed{\icell}(\jamdensity{\icell} - \density{\icell}{\itime}) < \fmax{\icell}}
\\
\forall \icell \in \{ 1, \dots, \ncell \}, \forall \itime \in \{0, \ntime - 1\}
\label{eq:}
\end{multline}
\end{subequations}

% ramp count
The equations corresponding to the partial derivative with respect to the ramp queue $\rampqueue{\icell}{\itime}$ are
\begin{align}
\G{2}{\ntime}{\icell}: && 0 &= 1 - \Lambd{2}{\ntime}{\icell}
& \forall \icell \in \{ 1, \dots, \ncell-1 \}, \itime = \ntime
\tag{F2}
\label{eq:}
\\
\stdG{2}: && 0 &= 1 - \Lambd{2}{\itime}{\icell} + \Lambd{2}{\itime+1}{\icell} + \Lambd{5}{\itime}{\icell} \ind{\rampqueue{\icell}{\itime} < \rampcontrol{\icell}{\itime}}
& \forall \icell \in \{1, \dots, \ncell - 1\}, \forall \itime \in \{0, \ntime - 1\}
\label{eq:}
\end{align}

% demand
The equations corresponding to the partial derivative with respect to the demand $\celldemand{\icell}{\itime}$ are
\begin{subequations}
\begin{multline}
\stdG{3}: 0 = - \Lambd{3}{\itime}{\icell} + (1-\offrampratio{\icell}{\itime})\Lambd{6}{\itime}{\icell+1} 
\ind{ (1-\offrampratio{\icell}{\itime})\celldemand{\icell}{\itime} + \rampdemand{\icell}{\itime} < \cellsupply{\icell+1}{\itime} } + 
\Lambd{7}{\itime}{\icell} 1_{\R{2}{\itime}{\icell} \cup \R{3}{\itime}{\icell}}
\\
\icell \in \{1, \dots, \ncell - 1\}, \forall \itime \in \{0, \ntime - 1\}
\label{eq:}
\end{multline}
%
\begin{align}
\G{3}{\itime}{0}: && 0 &= - \Lambd{3}{\itime}{\icell} + \Lambd{6}{\itime}{1}
& \icell = 0, \forall \itime \in \{0, \ntime - 1\}
\label{eq:}
\\
\G{3}{\itime}{\ncell}: && 0 &= - \Lambd{1}{\itime+1}{\ncell} - \Lambd{3}{\itime}{\ncell}
& \icell = \ncell, \forall \itime \in \{0, \ntime - 1\}
\label{eq:}
\end{align}
\end{subequations}

%supply
The equations corresponding to the partial derivative with respect to the supply $\cellsupply{\icell}{\itime}$ are
\begin{subequations}
\begin{align}
\stdG{4}: && 0 &= - \Lambd{4}{\itime}{\icell} + \Lambd{6}{\itime}{\icell} 
\ind{ \cellsupply{\icell}{\itime} < (1-\offrampratio{\icell-1}{\itime})\celldemand{\icell-1}{\itime} + \rampdemand{\icell-1}{\itime} }
& \icell \in \{2, \dots, \ncell\}, \forall \itime \in \{0, \ntime - 1\}
\label{eq:}
\\
\G{4}{\itime}{1}: && 0 &= - \Lambd{4}{\itime}{1} + \Lambd{6}{\itime}{1} 
\ind{ \cellsupply{1}{\itime} < \celldemand{0}{\itime} }
& \icell = 1, \forall \itime \in \{0, \ntime - 1\}
\label{eq:}
\end{align}
\end{subequations}

% ramp demand
The equations corresponding to the partial derivative with respect to the ramp demand $\rampdemand{\icell}{\itime}$ are
\begin{multline}
\stdG{5}: 0 = - \Lambd{5}{\itime}{\ncell} + \Lambd{6}{\itime}{\icell+1}
\ind{ (1-\offrampratio{\icell}{\itime})\celldemand{\icell}{\itime} + \rampdemand{\icell}{\itime} < \cellsupply{\icell+1}{\itime} } - 
\Lambd{7}{\itime}{\icell} 1_{\R{4}{\itime}{\icell}}
\\ \icell \in \{1, \dots, \ncell - 1\}, \forall \itime \in \{0, \ntime - 1\}
\label{eq:}
\end{multline}

%flow in
The equations corresponding to the partial derivative with respect to the in-flux $\flowin{\icell}{\itime}$ are
%
\begin{subequations}
\begin{multline}
\stdG{6}: 0 = \frac{\deltat}{\deltax} \Lambd{1}{\itime+1}{\ncell} - \Lambd{6}{\itime}{\icell} + \Lambd{7}{\itime}{\icell-1} \left[ \frac{1}{1-\offrampratio{\icell - 1}{\itime}} 1_{\R{1}{\itime}{\icell-1}} + 1_{\R{4}{\itime}{\icell-1}} + \priority{\icell} 1_{\R{5}{\itime}{\icell-1}} \right] + \Lambd{8}{\itime}{\icell-1}
\\ \icell \in \{2, \dots, \ncell \}, \forall \itime \in \{0, \ntime - 1\}
\label{eq:}
\end{multline}
%
\begin{align}
\G{6}{\itime}{1}: && 0 = \frac{\deltat}{\deltax} \Lambd{1}{\itime+1}{1} - \Lambd{6}{\itime}{1} + \Lambd{7}{\itime}{0}
&& \icell = 1, \forall \itime \in \{0, \ntime - 1\}
\label{eq:}
\end{align}
\end{subequations}

% flow out
The equations corresponding to the partial derivative with respect to the out-flux $\flowout{\icell}{\itime}$ are
%
\begin{subequations}
\begin{align}
\stdG{7}: && 0 &= - \frac{\deltat}{\deltax} \Lambd{1}{\itime+1}{\icell} - \stdLambd{7} - \Lambd{8}{\itime}{\icell}  (1 - \offrampratio{\icell}{\itime})
& \icell \in \{1, \dots, \ncell - 1\}, \forall \itime \in \{0, \ntime - 1\}
\label{eq:}
\\
\G{7}{\itime}{0}: && 0 &= - \frac{\deltat}{\deltax} \Lambd{1}{\itime+1}{0} - \Lambd{7}{\itime}{0}
& \icell = 0, \forall \itime \in \{0, \ntime - 1\}
\label{eq:}
\end{align}
\end{subequations}

% ramp flow
The equations corresponding to the partial derivative with respect to the on-ramp flux $\rampflow{\icell}{\itime}$ are
%
\begin{align}
\stdG{8}: && 0 &= - \deltat \Lambd{2}{\itime+1}{\icell} + \stdLambd{8}
& \icell \in \{1, \dots, \ncell - 1\}, \forall \itime \in \{0, \ntime - 1\}
\label{eq:}
\end{align}

As mentioned above, the adjoint linear system is upper triangular, and can be solved backwards in time, starting from the last time step $\ntime$ (i.e. solve equations $\G{1}{\ntime}{\icell}$ and $\G{2}{\ntime}{\icell}$). At each time step, the system is solved for decreasing $h$ (starting from $h=8$) then for decreasing $\icell$.

\subsubsection{Changes for the modified Piccoli model}
In the modified Piccoli model the equations corresponding to the partial derivative with respect to the in-flux $\flowin{\icell}{\itime}$ changes as follows.
%
% adjust counter
\setcounter{equation}{5}

\begin{subequations}
\begin{multline}
\stdG{6}: 0 = \frac{\deltat}{\deltax} \Lambd{1}{\itime+1}{\ncell} - \Lambd{6}{\itime}{\icell} + \Lambd{7}{\itime}{\icell-1} \left[ \frac{1}{1-\offrampratio{\icell - 1}{\itime}} 1_{\R{2}{\itime}{\icell-1}} +  \frac{\priority{\icell}}{1-\offrampratio{\icell - 1}{\itime}} 1_{\R{3}{\itime}{\icell-1}} \right] + \Lambd{8}{\itime}{\icell-1}
\\ \icell \in \{2, \dots, \ncell \}, \forall \itime \in \{0, \ntime - 1\}
\end{multline}
%
\begin{align}
\G{6}{\itime}{1}: && 0 = \frac{\deltat}{\deltax} \Lambd{1}{\itime+1}{1} - \Lambd{6}{\itime}{1} + \Lambd{7}{\itime}{0}
&& \icell = 1, \forall \itime \in \{0, \ntime - 1\}
\end{align}
\end{subequations}

where $\R{2}{\itime}{\icell-1}$ and $\R{3}{\itime}{\icell-1}$ refer to the new conditions defined in section~\ref{sec:newModel}.



% Equation number stuff: resume normal numbering
%-----------------------------------------------------------------------------------------------------------------------------------------------------------------
\setcounter{equation}{\theoldequation}
\renewcommand{\theequation}{\arabic{equation}}
%-----------------------------------------------------------------------------------------------------------------------------------------------------------------
