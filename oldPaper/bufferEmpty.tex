\subsection{\label{sub:Buffer-remains-empty}Buffer remains empty after emptying}

First, we give the necessary and sufficient condition for the buffer
to empty in a single time step:
\begin{lem}
\label{lem:QE}The buffer $\rampqueuesymbol$ empties during the time
period $\deltat=1$ if and only if $0\le\totalrampflowsymbol<\rampqueueinit{}+\totalrampflowsymbol<\rampflowsymbol\le\rampcontrolsymbol$\end{lem}
\begin{proof}
After $\deltat$, the total flux into the ramp is $\rampqueueinit{}+\totalrampflowsymbol\deltat=\rampqueueinit{}+\totalrampflowsymbol$.
Then, if the flux out of the ramp is greater than this quantity, the
ramp will empty. A similar argument shows the other direction.
\end{proof}
Next, we show that the buffer must remain empty once it empties.
\begin{lem}
\label{lem:remain-empty}The solution of the buffer flux once the
queue empties is $\totalrampflowsymbol$.\end{lem}
\begin{proof}
Let us denote with the prime symbol the input into the second Riemann
problem (when the buffer empties). Properties of the junction solver
give us the condition $\max\left(\frac{\cellsupplysymbol'}{\prioritysymbol+1},\cellsupplysymbol'-\celldemandsymbol'\left(1-\offrampratiosymbol\right)\right)\ge\totalrampflowsymbol\implies\rampflowsymbol'=\totalrampflowsymbol$.
So, we need to only show that the left hand condition always holds
when the buffer empties, or we show $\frac{\cellsupplysymbol'}{\prioritysymbol+1}<\totalrampflowsymbol\implies\cellsupplysymbol'-\celldemandsymbol'\left(1-\offrampratiosymbol\right)\ge\totalrampflowsymbol$.
When $\frac{\cellsupplysymbol'}{\prioritysymbol+1}<\totalrampflowsymbol$
holds, then we also have $\cellsupplysymbol'\ge\cellsupplysymbol$,
$\celldemandsymbol'=\celldemandsymbol$, and $\cellsupplysymbol'\ge\cellsupplysymbol\ge\celldemandsymbol\left(1-\offrampratiosymbol\right)+\rampqueueinit{}+\totalrampflowsymbol$
from Lemma~\ref{lem:QE}. We then have:

\begin{eqnarray*}
\cellsupplysymbol'-\celldemandsymbol'\left(1-\offrampratiosymbol\right) & \ge & \cellsupplysymbol-\celldemandsymbol\left(1-\offrampratiosymbol\right)\\
 & \ge & \celldemandsymbol\left(1-\offrampratiosymbol\right)+\rampqueueinit{}+\totalrampflowsymbol-\celldemandsymbol\left(1-\offrampratiosymbol\right)\\
 & = & \rampqueueinit{}+\totalrampflowsymbol\\
 & \ge & \totalrampflowsymbol
\end{eqnarray*}

\end{proof}

\subsection{\label{sub:Difference-between-discrete}Difference between discrete
and continuous model}

The discrete model includes the condition:

\[
\rampdemandsymbol=\min\left(\rampqueuesymbol,\rampcontrolsymbol\right),
\]


while the continuous model is:

\[
\rampdemandsymbol=\begin{cases}
\rampcontrolsymbol & \rampqueuesymbol>0\\
\min\left(\rampcontrolsymbol,\totalrampflowsymbol\right) & \text{otherwise}
\end{cases}
\]


The difference sometimes ends up being equivalent when considering
the Godunov discretization (i.e. the cumulative fluxes are the same
over the time step). We have that the models may result in different
flux solutions only if the buffer empties.
\begin{lem}
\label{lem:must-empty}If the buffer does not empty during the simulation
time step for either the continuous or discrete case, then the cumulative
fluxes are the same.
\end{lem}
\begin{proof}
We consider the case where the buffer \emph{may} empty, as the other case is trivial. If the solution of the junction problem for the continuous case results in a ramp flux $\le \rampqueueinit{} + \totalrampflowsymbol$, then the same maximal point will be feasible for the discrete problem.
\end{proof}
Furthermore, there are scenarios where the cumulative fluxes are the
same, even when the buffer empties.
\begin{lem}
\label{dl-sl-cool}If the continuous model begins as $\demandlimited$
or ends as $\supplylimited$, then the cumulative fluxes are equivalent
between the continuous model and discrete model.\end{lem}
\begin{proof}
One can show that if the first Riemann problem is $\demandlimited$,
the second Riemann problem will also be $\demandlimited$ and thus
$\fluxsymbol_{\linkmlone}=\celldemandsymbol_{\linkmlone}$. From Lemma~\ref{lem:remain-empty},
$\rampfluxsymbol=\rampqueuesymbol$. One can also show that if the
continuous model is $\demandlimited$ then so is the discrete model
($\fluxsymbol_{\linkmlone}=\celldemandsymbol_{\linkmlone}$). Therefore,
from Lemma~\ref{lem:must-empty}, we also know the buffer must empty
($\rampfluxsymbol=\rampqueuesymbol$).

One can show that if the second Riemann problem is $\supplylimited$,
the first Riemann problem will also be $\supplylimited$. Therefore
$\fluxsymbol_{\linkmltwo}=\cellsupplysymbol_{\linkmltwo}$ for the
continuous case. One case also show that if the second Riemann problem
is $\supplylimited$, then the discrete problem must be $\supplylimited$
as well, again giving $\fluxsymbol_{\linkmltwo}=\cellsupplysymbol_{\linkmltwo}$
for the discrete case. Lemma~\ref{lem:remain-empty} and Lemma~\ref{lem:must-empty}
says that the buffer fluxes will be $\rampfluxsymbol=\rampqueuesymbol$
for both cases again.
\end{proof}
Finally, by using counter-examples, one can show that for a $\supplylimited-\demandlimited$
transition in the continuous problem, the cumulative fluxes need not
be the same.
\begin{lem}
\label{lem:dl-to-sl-not-cool}Let $\celldemandsymbol_{\linkmlone}'$
be the demand for the second Riemann problem. Then the continuous
problem and the discrete problem may have different cumulative fluxes
if the buffer empties and $\celldemandsymbol_{\linkmlone}'+\totalrampflowsymbol<\cellsupplysymbol_{\linkmltwo}$.\end{lem}
\begin{proof}
By example. The condition that the buffer empties and $\celldemandsymbol_{\linkmlone}'+\totalrampflowsymbol<\cellsupplysymbol_{\linkmltwo}$
is equivalent to transitioning from $\supplylimited$ to $\demandlimited$.
In this case, the flux into the downstream cell is $\cellsupplysymbol_{\linkmltwo}\deltat_{1}+\left(1-\deltat_{1}\right)\left(\celldemandsymbol_{\linkmlone}'+\totalrampflowsymbol\right)<\cellsupplysymbol_{\linkmltwo}$.
With these conditions, it is possible that the discrete case is $\supplylimited$,
and has a flux of $\cellsupplysymbol_{\linkmltwo}$ into the downstream
cell.\end{proof}
\begin{thm}
\label{thm:check-fluxes-sometiems}The continuous problem and the
discrete problem may have different cumulative fluxes \textbf{only}
if the buffer empties and $\celldemandsymbol_{\linkmlone}'+\totalrampflowsymbol<\cellsupplysymbol_{\linkmltwo}$.\end{thm}
\begin{proof}
Theorems~\ref{dl-sl-cool}~and~\ref{lem:dl-to-sl-not-cool} enumerate
all possibilities for Riemann problems when the buffer empties.
\end{proof}

\subsection{\label{sub:Godunov-solution-of}Godunov solution of case when buffer
empties}

As stated in Theorem~\ref{thm:check-fluxes-sometiems}, we only need
to modify the Godunov solution of a very particular case: when the
buffer empties and transitions from $\supplylimited$ to $\demandlimited$
when the buffer empties. Additionally, all variables may stay the
same, except for $\flowout{\icell}{\itime},\flowin{\icell+1}{\itime},$
and $\rampflow{\icell}{\itime}$.

To solve for this value, we first solve the first Riemann problem
for the continuous problem ($\fluxsymbol,\rsind{\densitysymbol}$),
and check if the conditions in Theorem~\ref{thm:check-fluxes-sometiems}
hold. If this is not the case, then we take the solution obtained
from the previous discretized system. If the condition holds, then
we modify our solution as follows:

\begin{eqnarray*}
t & = & \mbox{\ensuremath{\frac{\rampqueue{\icell}{\itime}-\totalrampflow{\icell}{\itime}}{\rampflux{\icell}{\itime}-\totalrampflow{\icell}{\itime}}}}\\
\flowin{\icell+1}{\itime} & = & t\cellsupply{\icell+1}{\itime}+\left(\deltat-t\right)\left(\totalrampflow{\icell}{\itime}+\celldemand{\icell}{\rsind{\densitysymbol_{\linkmlone}}}\left(1-\offrampratio{\icell}{\itime}\right)\right)\\
\rampflow{\icell}{\itime} & = & \rampqueue{\icell}{\itime}\\
\flowout{\icell}{\itime} & = & \frac{\flowin{\icell+1}{\itime}-\rampflow{\icell}{\itime}}{1-\offrampratio{\icell}{\itime}}
\end{eqnarray*}


A complication with the new formulation is that the system is no longer
piecewise-affine. The flux across the junction is a function of the
moment that the queue empties, which is in turn a function of the
queue length, upstream demand, and downstream supply. The coupling
is introduced due to the fact that the ODE describing the buffer need
not be self-similar, and sometimes introduces a second shock in the
Riemann problem. Normally, only a single shock is considered, which
results in constant flux terms to be integrated over, while in our
case, the discontinuity in the flux causes some nonlinear coupling.
