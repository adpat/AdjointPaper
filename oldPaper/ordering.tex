% !TEX root = rampMeteringViaTheAdjoint.tex
\label{sec:ordering}

In order create a lower triangular $\frac{\partial H}{\partial x}$
matrix, we must have the proper ordering of the variables in the $\mathbf{x}$
vector. This process is similar to understanding the nature of the
dependencies in the forward simulation. Given the boundary and initial
conditions, the discretized system can be exactly solved by solving
the constraints in a particular order. From this consideration, we
create a ``topological ordering'' of the variables, or the order
in which the variables must be solved from the particular constraints.

\begin{align}
\density{\icell}0,\rampqueue j0 & & & \icell\in\left\{ 0,\ldots,\ncell\right\} ,j\in\left\{ 1,\ldots,\ncell-1\right\} \nonumber \\
\celldemand i0,\cellsupply j0,\rampdemand k0 & & & \icell\in\left\{ 0,\ldots,\ncell\right\} ,j\in\left\{ 1,\ldots,\ncell\right\} ,k\in\left\{ 1,\ldots,\ncell-1\right\} \nonumber \\
\flowin i0 & & & \icell\in\left\{ 1,\ldots,\ncell\right\} \nonumber \\
\flowout i0 & & & i\in\left\{ 0,\ldots,\ncell\right\} \nonumber \\
\rampflow i0 & & & i\in\left\{ 1,\ldots,\ncell-1\right\} \nonumber \\
\density{\icell}1,\rampqueue j1 & & & \icell\in\left\{ 0,\ldots,\ncell\right\} ,j\in\left\{ 1,\ldots,\ncell-1\right\} \nonumber \\
\ldots\nonumber \\
\rampflow ik & & & i\in\left\{ 1,\ldots,\ncell-1\right\} \nonumber \\
\density{\icell}{k+1},\rampqueue j{k+1} & & & \icell\in\left\{ 0,\ldots,\ncell\right\} ,j\in\left\{ 1,\ldots,\ncell-1\right\} \label{eq:topological-ordering}\\
\ldots\nonumber \\
\rampflow i{\ntime} & & & i\in\left\{ 1,\ldots,\ncell-1\right\} \nonumber 
\end{align}


The ordering has three nested loops: first over time index, then over ``variable type'', then over cell index. Each row in the set of equations only depends upon variables in above
rows, or more precisely, they will only depend on variables in rows
above \emph{and} on the same time step $\itime$ or $\itime-1$.

The constraints in the system equations section were presented in
a particular way, in order to have every variable in the system appear
alone on the LHS of exactly one equation, and have the RHS of each
equation only be a function of variables with a ``lower'' topological
order. Therefore, it is clear that if we order the $\mathbf{x}$ variables
according to Equation~\ref{eq:topological-ordering}, and order the
constraints according to when the LHS variable of the constraint appears
in $\mathbf{x}$, then the $\frac{\partial H}{\partial x}$ matrix
will have a lower-triangular structure. Additionally, there will be
all $1$'s on the diagonal.

When the adjoint system is considered, the transpose of the $\frac{\partial H}{\partial x}$
matrix is taken, thus creating an upper-triangular structure. Such
a system can be solved backwards in time, as opposed to the original
system being solved forward in time from the initial conditions.

Note that some of the constraints contain special cases for initial
conditions and boundary conditions, but it should be clear to make
the 1-to-1 mapping with the variables based on the time and space
indices being considered.
